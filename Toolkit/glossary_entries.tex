% Combined Entries ----------------------------------------------------------------------------------------------------
% Argument order: id, acronym, acronym written out, description

% Create both, acronym and glossary entry with linking:
\newcommand{\newglossaryacronym}[4]{%
%  \newglossaryentry{#1}{
%     type=\acronymtype, 
%     name={#2}, 
%     description={#3}, 
%     first={#3~(#2)\glsadd{#1gls}}, 
%     firstplural={#3\glspluralsuffix~(#2\glspluralsuffix)},
%     see=[See:]\gls{#1gls}
%     }
 \newacronym{#1acr}{#2}{\acrlong{#1}}
 \newglossaryentry{#1}{
        name={#2 (#3)},
        text={#2},
        % symbol={#2},
        short={#2},
        long={#3},
        first={#3 (#2)},
        firstplural={#3\glspluralsuffix~(#2\glspluralsuffix)},
        description={#4\glsadd{#1acr}}
 }
}


% Alternative: All-in-one Glossary entry (https://tex.stackexchange.com/a/410534/213738)
% \newcommand{\newglossaryacronym}[4]{
%     \newglossaryentry{#1}{
%         text={#2},
%         % short={#2},
%         long={#3},
%         name={\glsentrylong{#1} (\glsentrytext{#1})},
%         first={\glsentryname{#1}},
%         firstplural={\glsentrylong{#1}\glspluralsuffix (\glsentryname{#1}\glspluralsuffix )},
%         description={#4}
%     }
% }


% ------------------------------

\newglossaryacronym{ip}{IP}{Interaction Point}{
    Point at which the two counter-rotating \glspl{beam} interact, i.e. their particles collide. 
    In the \acrfull{lhc} these are located in the center of 
    \glsfirstplural{ir} 1, 2, 5 and 8 inside the detectors (``\Glspl{experiment}'')%
}

\newglossaryacronym{ir}{IR}{Insertion Region}{
    Straight section between the arcs of a synchrotron, housing larger facilities, 
    such as detectors (``\Glspl{experiment}''), acceleration (``\acrshort{rf}''), etc%
}

\newglossaryacronym{rf}{RF}{Radio Frequency}{
    Short for the acceleartion part of a synchrotron, as 
    the frequency of the accelerating electric field
    is usually in the radio-frequency range (at the \acrshort{lhc} $\approx \qty{400}{\mega\hertz}$)%
}

\newglossaryacronym{atlas}{ATLAS}{A Toroidal LHC Apparatus}{
    is the larger of two general-purpose detectors at the \acrfull{lhc}. 
    It investigates a wide range of physics, 
    from the search for the Higgs boson to extra dimensions and particles that could make up dark matter. 
    Although it has the same scientific goals as the \acrshort{cms} \glstext{experiment}, 
    it uses different technical solutions and a different magnet-system design.
    See: \href{https://atlas.cern/}{atlas.cern}%
}

\newglossaryacronym{cms}{CMS}{Compact Muon Solenoid}{
    a general-purpose detector at the \acrfull{lhc}. 
    It has a broad physics programme ranging from studying the Standard Model (including the Higgs boson) 
    to searching for extra dimensions and particles that could make up dark matter. 
    Although it has the same scientific goals as the \acrshort{atlas} \glstext{experiment}, 
    it uses different technical solutions and a different magnet-system design.
    See: \href{https://cms.cern}{cms.cern}%
}

\newglossaryacronym{alice}{ALICE}{A Large Ion Collider Experiment}{
    An \glstext{experiment} optimized to study heavy-ion collisions. 
    See: \href{https://lhcb.cern}{lhcb.cern}%
}

\newglossaryacronym{lhcb}{LHCb}{LHC-beauty}{
    A specialized b-physics \glstext{experiment}, 
    designed primarily to measure the parameters of CP violation in the interactions of 
    b-\glspl{hadron} (heavy particles containing a bottom quark).
    See: \href{https://lhcb.cern}{lhcb.cern}%
}

\newglossaryacronym{md}{MD}{Machine Development}{
    Studies dedicated to improving the machine, 
    e.g. by trying to reveal error-sources or attempting their correction,
    or testing possible future procedures and machine configurations.
    As they usually involve \gls{beam}-time, i.e. measurements in the \glsfirst{ccc},
    a proposal needs to be written and they need to be approved by the \acrfull{lmc}.
}

% \newglossaryacronym{}{}{}{
    
% }



% Acronym Entries ----------------------------------------------------------------------------------------------------
\newacronym{bbq}{BBQ}{Base Band \Gls{tune}}
\newacronym{bpm}{BPM}{\Gls{beam} Position Monitor}
\newacronym{blm}{BLM}{\Gls{beam} Loss Monitor}
\newacronym{bctfr}{BCTFR}{Fast \Gls{beam} Current Transformer}
\newacronym{linac}{LINAC}{Linear Accelerator}
\newacronym{ps}{PS}{Proton Synchrotron}
\newacronym{psb}{PSB}{\acrlong{ps} Booster}
\newacronym{sps}{SPS}{Super \acrlong{ps}}
\newacronym{lhc}{LHC}{Large \Gls{hadron} Collider}
\newacronym{fcc}{FCC}{Future Circular Collider}
\newacronym{hllhc}{HL-LHC}{High-\Gls{luminosity} Large \Gls{hadron} Collider}
\newacronym{rhic}{RHIC}{Relativistic Heavy Ion Collider}
\newacronym{da}{DA}{Dynamic Aperture}
\newacronym{omc}{OMC}{\Gls{optics} Measurements and Corrections}
% \newacronym[see=\cref{sec:BackgroundNFandRDTs}]{rdt}{RDT}{Resonance Driving Term}
\newacronym{svd}{SVD}{Singluar Value Decomposition}
\newacronym{yets}{YETS}{Year-End Technical Stop}
\newacronym{wise}{WISE}{Windows Interface to Simulation Errors}
\newacronym{fidel}{FiDeL}{Field Model of the \acrshort{lhc}}
\newacronym{mad}{MAD}{Methodical Accelerator Design}
\newacronym[see=\cite{ForestIntroductionPolymorphicTracking2002}]{ptc}{PTC}{Polymorphic Tracking Code}
\newacronym{be}{BE}{\Glspl{beam} Department}
\newacronym{abp}{ABP}{Accelearator \Gls{beam} Physics Group}
\newacronym{lno}{LNO}{Linear and Nonlinear \Gls{optics} Section}
\newacronym{cern}{CERN}{Conseil Europ\'een pour la Recherche Nucl\'eaire \\ European Organization for Nuclear Research}
\newacronym{ccc}{CCC}{\acrshort{cern} Control Center}
\newacronym{lmc}{LMC}{\acrshort{lhc} Machine Committee}
\newacronym{prab}{PRAB}{Physical Review Accelerators and Beams}
\newacronym[first={the \acrfull{ipac}}]{ipac}{IPAC}{International Particle Accelerator Conference}
\newacronym[first=h.o.t]{hot}{h.o.t}{High(er) Order Terms}
% \newacronym{}{}{}


% Glossary Entries ----------------------------------------------------------------------------------------------------
\newglossaryentry{ampdet}
{
        name=amplitude detuning,
        description={
            Tune change with the transversal amplitude of a particle.
            See~\cref{sec:BackgroundAmplitudeDetuning}%
        }
}

\newglossaryentry{acdipole}
{
        name={AC-Dipole},
        description={Dipole magnet attached to an AC power source with variable frequency and strength,
        thus allowing to impose forced oscillations on the \gls{beam}. See~\cite{SerranoLHCACDipole2010}}
}


\newglossaryentry{betafunction}
{
        name={beta-function},
        text=\ensuremath{\beta\text{-function}},
        description={Value of the \gls{twiss} $\beta$ as a function of longitudinal location.
        This value is closely related to the amplitude $\mathcal{A}$ of the betatron-oscillations
        and hence the size of the \gls{beam} at that location via the \gls{action} \glssymbol{action} 
        by $\mathcal{A} = \sqrt{2J\beta}$%
        }
}

\newglossaryentry{betabeat}
{
        name={beta-beating},
        text=\ensuremath{\beta\text{-beating}},
        description={Relative difference of the \gls{betafunction}
        between measurement and model: $\sfrac{\left(\beta_\text{Measured} - \beta_\text{Model}\right)}{\beta_\text{Model}}$
        }
}


\newglossaryentry{twiss}
{
        name={twiss parameters},
        text={twiss parameter},
        description={See \gls{courantsnyderparameter}}
}

\newglossaryentry{courantsnyderparameter}
{
        name={Courant-Snyder parameters},
        text={Courant-Snyder parameter},
        user1={Courant-Snyder},
        description={
            Set of quantities describing the distribution of 
            positions and momenta of the particles in a \gls{beam}.
            Also known as \glspl{twiss}. See also \gls{betafunction}
        }
}

\newglossaryentry{beam}
{
        name=beam,
        description={Short for ``Particle Beam''. 
        The name for the collection of all particles in an accelerator traveling in the same direction.
        In the \acrfull{lhc} there are two beams (Beam~1 and Beam~2) traversing in opposite directions
        and colliding at the \glsfirstplural{ip}}
}

\newglossaryentry{bunch}
{
        name=bunch,
        description={Short for ``Particle Bunch''.
        Due to the \acrshort{rf} acceleration scheme, a continuous \gls{beam} is not possible in a
        synchrotron and particles become bunched during acceleration}
}


\newglossaryentry{experiment}
{
        name=experiment,
        description={
            In the context of the \acrfull{lhc}, an ``experiment'' denotes 
            one of the eight particle physics detector experiments   
            and its accompanying data-collecting research structure.
            The four largest detectors, \acrshort{atlas}, \acrshort{cms}, \acrshort{alice}, \acrshort{lhcb} 
            are located at the four \glsfirstplural{ip} of the \acrshort{lhc}%
        }
}

\newglossaryentry{flatoptics}
{
        name=flat-optics,
        description={\Gls{optics} in which \glstext{betastar} is different for both planes. 
        Usually much smaller in one than in the other. 
        See also: \glstext{roundoptics}}
}

\newglossaryentry{roundoptics}
{
        name=round-optics,
        description={\Gls{optics} in which \glstext{betastar} is identical for both planes.
        See also: \glstext{flatoptics}} 
}

\newglossaryentry{flatorbit}
{
        name=flat-orbit,
        description={Machine setup without any crossing bumps in the \glsfirstplural{ir}}
}

\newglossaryentry{feeddown}
{
        name=feed-down,
        description={Particles passing off-center through a multipole field experience effects 
        akin to the influence of lower-order multipoles due to the orbit offset.
        See~\cref{sec:FeedDown}%
        }
}

\newglossaryentry{hadron}
{
        name=hadron,
        description={Composite subatomic particles consisting of two or more quarks,
        held together by strong interactions}
}

\newglossaryentry{luminosity}
{
        name=luminosity,
        description={Is the ratio of the number of events, i.e. collisions, detected per cross section,
        either per time interval (``instantaneous luminosity'') or in total 
        (``integrated luminosity'', e.g. since the beginning of a year, a \gls{run} or of operation).
        Colloquially it is shortened to ``lumi''}
}

\newglossaryentry{svdclean}
{
        name={SVD cleaning},
        description={Noise cleaning method used in the \glstext{omc3} python package. 
        Based on an \acrfull{svd}, using the strongest modes to reconstruct the signal}
}

\newglossaryentry{run}
{
        name={run},
        description={Consecutive years of \acrshort{lhc}-operation, separated by 
        long shutdowns. Run~1 has been from 2008-2013, run~2 from 2015-2018 and run~3 started in 2022}
}

\newglossaryentry{online}
{
        name={online analysis},
        text=online,
        user1={online analysis},
        description={Any analysis that is being done in the \glsfirst{ccc}, 
        i.e. at the same time as the measurements are performed during machine commissioning or \glsfirst{md}.
        In contrast, see: \glsuseri{offline}}
}

\newglossaryentry{offline}
{
        name={offline analysis},
        text=offline,
        user1={offline analysis},
        description={Any analysis that is being done after the measurements are completed.
        In contrast, see: \glsuseri{online}}
}

\newglossaryentry{normal}
{
        name={normal magnets},
        text=normal,
        description={Magnet with a ``normal'' oriented multipole field with pole-tips off the horizontal axis. 
        Opposed to a ``\gls{skew}'' magnet of the same order $n$, for which the field is rotated by $\sfrac{\pi}{(2n)}$\,\si{\radian}}
}

\newglossaryentry{skew}
{
        name={skew magnets},
        text=skew,
        description={Magnet with a ``skew'' oriented multipole field, with pole-tips on the horizontal axis. 
        Opposed to a ``\gls{normal}'' magnet of the same order $n$, for which the field is rotated by $\sfrac{-\pi}{(2n)}$\,\si{\radian}}
}

\newglossaryentry{omc3}
{
        name={omc3},
        text={\texttt{omc3}},
        user1={\glstext{\glslabel}~\cite{OMC-TeamOMC3}},
        user2={\glstext{\glslabel}-package~\cite{OMC-TeamOMC3}},
        description={Python package maintained and applied by the \glsfirst{omc}-Team to analyse 
        measurements from different accelerators and calculate corrections. See~\cite{OMC-TeamOMC3}}
}

\newglossaryentry{optics}
{
        name={optics},
        description={Here ``optics'' refers to ``accelerator \gls{beam} optics'',
        describing the particle motion through an accelerator as defined by the elements
        (i.e. magnets) of the machine. As the behaviour of a particle \gls{beam} in magnetic fields shows 
        similarities to a light beam propagating through lenses and can be described with similar equations,
         a lot of theory and nomenclature has been borrowed from ray optics%
        }
}

\newglossaryentry{triplet}
{
        name=triplet,
        description={Name for the final focusing quadrupoles in the 
        \glsfirst{ir} left and right of an \glsfirst{ip}.
        In principle these consist of three magnets Q1--3 (starting with Q1 closest to the 
        \acrshort{ip}), where Q1 and Q3 have the same orientation (e.g. focusing) and are opposite
        to Q2 (e.g. defocusing), depending on \gls{beam}, plane and side.
        In the \acrfull{lhc} Q2 is split into two magnets, 
        while in the \acrfull{hllhc} also Q1 and Q3 will each be split in two%
        }
}

\newglossaryentry{tune}
{
        name=tune,
        description={Number of betatron oscillations per turn in a circular accelerator.
        See also: \gls{betafunction}%
        }
}

\newglossaryentry{d1d2}
{
        name={D1\&D2},
        text={D1/D2},
        description={D1 and D2 are the shorthands for the separation/recombination dipoles in the 
        \glsfirstplural{ir} of the \glsfirst{lhc} and \glsfirst{hllhc}, diverting the \glspl{beam} from their respective
        beam-lines into the common aperture region around the \glsfirstplural{ip} and 
        vice-versa%
        }
}

\newglossaryentry{longshutdown}
{
        name={long shutdown},
        description={Planned shutdown periods of the \acrfull{lhc} between \glspl{run} spanning multiple years,
        as opposed to the \acrfull{yets}, which only lasts a few months.
        This time is used for repairs and upgrades of the machine, as well as general maintenance.
        }
}

\newglossaryentry{madx}
{
        name={MAD-X},
        text={\texttt{MAD-X}},
        description={Current version of the \acrfull{mad} framework developed in \acrshort{be}-\acrshort{abp}
        at \acrshort{cern}%
        }
}


% Symbols ----------------------------------------------------------------------------------------------------

\newglossaryentry{betastar}
{
    type=\symboltype,
    sort={beta},
    name={$\bm \beta^*$},
    text={\ensuremath{\beta^*}},
    symbol={\ensuremath{\beta^*}},
    description={\gls{betafunction} at the \acrshort{ip}. 
    Usually subscripted with $x$ or $y$ giving the transversal plane.
    \textit{Unit: $m$}%
    },
    first={$\beta^*$ (the \gls{betafunction} at the \acrshort{ip})}
}

\newglossaryentry{action}
{
    type=\symboltype,
    sort={action},
    name={$\bm J$},
    text={action},
    symbol={\ensuremath{J}},
    description={\Glstext*{action}.
    One of the phase-space coordinates in the Courant-Snyder normalization
    and closely related to the invariant of (linear) motion $\epsilon = 2J$. 
    Usually subscripted with $x$ or $y$ giving the transversal plane.
    \textit{Unit: $m$}%
    }
}

\newglossaryentry{beam-rigidity}
{
    type=\symboltype,
    sort={beam-rigidity},
    name={B$\bm \rho$},
    text={beam rigidity},
    symbol={\ensuremath{\mathrm{B}\rho}},
    description={\Glstext*{beam-rigidity}.
    Used as a normalization factor for \glstext{K} and \glstext{J}. 
    See \cref{sec:FieldNormalization}.
    \textit{Unit: $Tm$}%
    }
}

\newglossaryentry{J}
{
    type=\symboltype,
    sort={strengthJ},
    name={$\bm J_n$},
    text={skew magnetic field strength},
    symbol={\ensuremath{J}},
    description={\Glstext*{J}.
    Skew field component normalized to the \glstext{beam-rigidity}.
    Usually subscripted with an integer $n$ giving the field order.
    \textit{Unit: $m^{-n}$}.\\
    See \cref{sec:FieldNormalization}%
    }
}

\newglossaryentry{K}
{
    type=\symboltype,
    sort={strengthK},
    name={$\bm K_n$},
    text={normal magnetic field strength},
    symbol={\ensuremath{K}},
    description={\Glstext*{K}.
    Normal field component normalized to the \glstext{beam-rigidity}.
    Usually subscripted with an integer $n$ giving the field order.
    \textit{Unit: $m^{-n}$}.\\
    See \cref{sec:FieldNormalization}%
    }
}
