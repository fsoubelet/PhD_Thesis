\begin{abstract}

In order to further expand our knowledge of the structure of matter and the workings of our universe, scientists are constantly seeking to collide particles at ever-increasing energies and with higher \gls{luminosity}.
So is the task of the \gls{LHC} at \acrshort{CERN}, the highest energy particle accelerator and collider to date, and the goal of its future upgrade into the \gls{HL-LHC}.
This constant progress in performance requires more intense beams and smaller beam sizes at collisions points as well as a tight control of these parameters.
Thus, successful operation of large-scale particle colliders heavily depends on the precise correction of magnet field or alignment errors present in the machine.

In the \gls{LHC}, transverse \gls{betatron_coupling} has been shown to have a significant impact on both the beam dynamics and luminosity production due to uncompensated sources close to the \glspl{IP}.
However, current measurement methods are not sufficient for precise local coupling measurement at the \gls{IP}, and the impact of these sources has so far been left uncompensated.
This thesis covers work done in an effort to determine and correct Interaction Region (\acrshort{IR}) local coupling.

A key tool presented in this document is the designed \gls{RWS}, a new optics configuration which allows the determination of local coupling corrections based on correlated global variables such as the closest tune approach \glssymbol{Cminus}.
The validity of this new method has been demonstrated through simulations and experimental measurements taken during the \gls{LHC} \Gls{run}~\num{3} commissioning in~\num{2022}, where determined corrections were applied and led to a measured luminosity increase of \qty{9.7}{\percent} and \qty{3.5}{\percent} at the \acrshort{ATLAS} and \acrshort{CMS} detectors, respectively.
Additionally, the application of machine learning techniques for high complexity problems such as the detection of coupling sources in the \gls{LHC} \acrshortpl{IR} has been explored, yielding promising results but requiring some more improvements to be operationally viable.
Finally, optics studies which revealed avenues for improvements in the optics measurements done at the LHC are also presented.

\end{abstract}

\glsresetall                                     % reset glossary entries counts for the next chapter
