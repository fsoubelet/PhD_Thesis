\begin{acknowledgements}
% \addchaptertocentry{\acknowledgementname} % Add the acknowledgements to the table of contents
\vspace{0.8cm}

% Somehow, this section turned out to be the hardest to write out of this thesis.
Now that the main results of several years of work have been summarized in this little document and, as this writing comes to an end, it is time to look back and thank the many people that have contributed to this work and supported me along the way.
\newline

First and foremost, I wish to thank Tobias Persson with all my heart for his kind supervision and unwavering support during those years.
I consider this PhD to be divided into a before and an after, and you to be the turning point.
I can never thank you enough for your patience, advice, guidance, encouragements and enthusiasm, without which I know I would not have crossed the finish line.
My heartfelt thanks go to Rogelio Tomás as well for keeping my best interest in mind, and for your continuous help throughout.
\newline

A few special people now need to be thanked here.
They have all, each in their own way, significantly contributed to helping me through this long journey.
\newline

A big thanks to Joschua Dilly for your enthusiasm for programming, "clean code" and scientific computing.
Thanks for the reviews, for the late night pull requests, the scientific discussions, the coffees, the control room banter and for your \LaTeX \ black magic knowledge.
This document would not have looked as good without you.

To Félix Carlier, or "Félix number \num{1}", thank you for taking the time to explain so much of the non-linear dynamics theory to me, and for providing advice on the writing of this document.
Your thesis, that you so kindly gave me a physical copy of, has been a great source of inspiration and has set a high standard of quality.

Thank you to Michael Hofer for the countless questions you have answered, for being the first to guide me through the perilous land of Resonance Driving Terms, and for being such an encyclopedia of knowledge and Indico links.

To Elena Fol, thank you for showing me how to teach the machines, for the coffee talks and for being such a little ray of sunshine every day (except when it's cold).

To Axel Poyet, thank you for being you and being there every step of the way, from close and afar.
Thanks for the friendship, the jokes, the sarcasm, the shared struggle and all of the silliest moments: it has been great.

Thank you to Konstantinos Paraschou for being a great flatmate through part of this journey, and getting through the pandemic lockdowns with me.
Shared mental breakdowns are strangely cathartic.
As you already know, "book boom".

Thanks also go to my family for their support, patience and encouragements.
\newline

Though not necessarily personally named, many other people have contributed one way or another to this adventure, and a few group mentions are in order.
\newline

My thanks go to all the past and current members of the OMC team for the great work environment, and for their cheerful company during all these hours - mostly nights and weekends - spent in the control room.
I am equally grateful to the LHC Operations team for their collaboration, for accepting me in their midst for a short time, and in particular to Michi Hostettler for being such a source of knowledge and always eager to provide help and advice.

To all the people I have walked the PhD road with - Thibault, Marie, Eva, Anaïs, Floriane, Guillaume, Mathieu, Nico, Tirsi, Natalia, Michalis, Maël, Seb, Giulia, - a thousand thanks for your comradeship and the myriad of great moments we have shared.
May you find success and happiness in your journey.

As a final word, a big "thank you!" to all scientists who have advanced this beautiful field of physics, have led to the design, creation and operation of the LHC, and overall have made my PhD possible.
As was put so well into words by Sir Isaac Newton: "If I have seen further, it is by standing on the shoulders of giants."

\end{acknowledgements}