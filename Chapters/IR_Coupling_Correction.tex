\chapter{Interaction Region Local Coupling Correction in the LHC} % Main chapter title
\label{chapter:IR_Local_Coupling} % For referencing the chapter elsewhere, use \cref{chapter:IR_Local_Coupling}

\todo{Some paragraph before the first section.}
An overview of the various experimental measurements taken for this work can be found in \cref{appendix:measurement_fills}.

%----------------------------------------------------------------------------------------

\section{Local Betatron Coupling in the LHC Interaction Regions}
\label{section:local_ir_coupling}

In the LHC, corrections of local IR linear coupling are of importance to keep a good control of beam sizes at the IPs and hence the luminosity performance, as well as to prevent a significant impact on the beam dynamics.

In \cref{equation:coupling_rdts_from_skew_quads} the contribution of elements to the linear coupling RDTs is given, where contributing elements are typically skew quadrupoles and solenoids.
The amplitude of the contribution is dominated by the integrated strength of the magnet \(k L\) as well as the \(\sqrt{\beta_x \beta_y}\) term at the location of the magnet.
Given that a tilted quadrupole interacts with the beam as a straight quadrupole with an additional skew quadrupolar component, once can see from \cref{figure:ir5_and_around} that any tilt in the triplet quadrupoles would generate a massive contribution to the coupling RDTs due to the very high \(\beta\)-functions in these magnets.

\Cref{figure:triplet_tilts_to_rdts} shows the coupling RDTs' amplitudes from tilts in triplet quadrupoles, with the \(\beta^{\ast} =\)~\qty{30}{\centi\meter} \num{2022} optics.
In this MAD-X simulation triplets around IP\num{1} were assigned random tilts within \(\pm\)~\qty{1.5}{\milli\radian}, and these were the only contribution to coupling in the machine.
Nevertheless, this contribution amounted to a \(\abs{C^{-}}\) of \num{3.84e-2}, too high for machine operation.

\begin{figure}[!htb]
    \centering
    \includegraphics*[width=0.99\columnwidth]{Figures/IR_Coupling_Correction/triplet_tilts_to_rdts.pdf}
    \caption{Amplitudes of the coupling RDTs (bottom) \(f_{1001}\) (\textcolor{mplblue}{blue}) and \(f_{1010}\) (\textcolor{mplorange}{orange}) from tilts in the triplet quadrupoles around IP\num{1}. The top plot shows the magnets' powering while the middle plot shows the assigned tilts in each element.}
    \label{figure:triplet_tilts_to_rdts}
\end{figure}

As a consequence, the IR contribution - mainly the triplets - to global coupling needs to be compensated.
For this, in the LHC the local coupling correction is done by measuring the RDTs in the vicinity of the IP and using the MQSX skew quadrupole correctors introduced in \cref{subsection:lhc_eirs} and showcased in \cref{figure:ir5_and_around,figure:lhc_ir_corrector_layout} for correction.
The corrections are determined with the Segment-by-Segment technique described in \cref{subsection:correction_principles}, and try to cancel the triplets' contribution as well as possible.
Though this compensation is rarely perfect, the residual contribution is usually small enough to be handled by the skew quadrupole correctors present in the LHC arcs (see \cref{subsection:lhc_arcs}).
This correction is essential in order to reach low \(\beta^{\ast}\) with good optics control: at \(\beta^{*}=\) \qty{30}{\centi\meter} the local errors compensated in Run~\num{2}~\cite{CERN:Persson:LHCOpticsCorrectionsEvian2019} would contribute to the \(\abs{C^{-}}\) by the amount of \num{0.33}, too high for the arc correctors to handle.
While such coupling in the machine is not inherently unstable in itself, it leads to other effects causing the machine to be unstable such as transverse instabilities from loss of Landau damping~\cite{MASTERS:Soubelet:Optics_Octupole_PyHEADTAIL,PRAB:Carver:Transverse_Instabilities_With_Coupling}.

Due to their location the MQSX correctors are single aperture magnets, meaning that both beams pass through a single cavity in the element and feel the same magnetic field.
This means finding a correction has the additional constraint that it is applied to both beam~\num{1} and beam~\num{2} simultaneously, and should be a compromise that works for both beams.
As the triplet quadrupoles - also single aperture magnets - are expected to be most of the contribution to local coupling, the local error to be corrected should be the same for both beams and such an arrangement of correctors is manageable.

During the late \num{2018} ion run in the LHC Run~\num{2}, it was observed that while global coupling was well corrected, a local coupling bump in the IR\num{2} had a significant impact on collisions and led to a reduction of the luminosity at the affected IP by up to \qty{50}{\percent}~\cite{IPAC:Jowett:LHC_2018_Heavy_Ion_Run, IPAC:Tomas:Run2_Experience_View_LHC_HLLHC, CERN:Persson:LHCOpticsCorrectionsEvian2019}.
Investigations revealed that a coupling bump around IP\num{2} led to a strong increase in beam size and a drop in collisions.
Importantly, the incident highlighted that no method existed to correct for the coupling specifically at the IP location.

\Cref{figure:lhc_vs_hllhc_beam_size_and_lumi_growths} shows the expected beam size growth and luminosity decrease from various strengths of such a local coupling bump at IP\num{1} or IP\num{5}, for the LHC at \(\beta^{*}=\) \qty{30}{\centi\meter} and for the HL-LHC at \(\beta^{*}=\) \qty{15}{\centi\meter}.
These results highlight the necessity of a proper handling of local coupling for the Run~\num{3}, as well as for the HL-LHC where it should be about a factor \num{4} more accurate.

\begin{figure}[!htb]
    \centering
    \includegraphics*[width=0.99\columnwidth]{Figures/IR_Coupling_Correction/lhc_vs_hllhc_combined.pdf}
    \caption{Relative values of the RMS beam size at IP\num{1} (\textcolor{mplblue}{blue}) as well as luminosity (\textcolor{mplorange}{orange}) for different strengths of a local coupling bump around the IP made with skew quadrupoles, for the LHC (filled) and HL-LHC v1.5 (dashed) collision optics.}% Beam sizes are calculated according to \cref{equation:lebedev_beam_size} and luminosities according to \cref{equation:luminosity_double_beams}. In the case of the HL-LHC, the relative beam size increase and subsequent luminosity loss are greater due to the much larger \(\beta\)-functions in the triplet.}
    \label{figure:lhc_vs_hllhc_beam_size_and_lumi_growths}
\end{figure}

In the studies presented in this thesis, various calculations rely heavily on the Ripken parameters mentioned in \cref{subsection:parametrization_of_betatron_coupling}.
For instance, beam sizes are calculated from the \(\beta_{kj}\) terms according to~\cite{IOP:Lebedev:Betatron_Motion_Coupling}:

\begin{equation}
    \langle z \rangle = \sqrt{\varepsilon_1 \beta_{1z} + \varepsilon_2 \beta_{2z}}, \quad z \in\{x, y\} ,
    \label{equation:lebedev_beam_size}
\end{equation}
where \(\varepsilon_1\) and \(\varepsilon_2\) are the horizontal and vertical emittances, respectively.

The validity of this calculation has been verified in simulations by comparing its results to those obtained from other means.
\Cref{figure:lebedev_vs_tracking} shows the relative deviation between computed beam sizes at IP\num{5}, calculated either according to \cref{equation:lebedev_beam_size} or from tracking a particle distribution, under various strengths of local coupling.
In all cases the relative deviation is below \qty{0.25}{\percent}.

\begin{figure}[!htb]
    \centering
    \includegraphics*[width=0.99\columnwidth]{Figures/IR_Coupling_Correction/lebedev_vs_tracking.pdf}
    \caption{Relative deviation between beam sizes calculated from Ripken parameters according to \cref{equation:lebedev_beam_size} and from tracking a particle distribution, at an IP affected by coupling for the horizontal (\textcolor{mplblue}{blue}) and vertical (\textcolor{mplorange}{orange}) planes.}
    \label{figure:lebedev_vs_tracking}
\end{figure}

At the LHC IPs with round beams, as is the case in Run~\num{3}, the effect of the beam's tilt induced by linear coupling is negligible and its impact manifests as an increase in the beam size, as was the case at IP\num{2} in \num{2018}.
\Cref{figure:ip_ellipses_from_coupling} shows a reconstruction of transverse beam sizes at IP\num{1} (similar for IP\num{5}) under various strengths of a local coupling bump.
While the beam ellipses show a \(\gg\) \qty{99}{\percent} overlap indicating a negligible tilt effect, the beam size in the most affected case is about \qty{250}{\percent} of the uncoupled case.

\begin{figure}[!htb]
    \centering
    \includegraphics*[width=0.85\columnwidth]{Figures/IR_Coupling_Correction/ellipses_various_coupling_bumps.pdf}
    \caption{Transverse beam sizes at IP\num{5} at \qty{6.8}{\tera\electronvolt} and \(\beta^{\ast}=\)~\qty{30}{\centi\meter} with normalized emittances \(\varepsilon_x = \varepsilon_y =\)~\qty{3.75}{\micro\meter} and for different strengths of a local coupling bump around the IP. The ellipses are reconstructed through the \(\sigma_{11}\), \(\sigma_{13}\) and \(\sigma_{33}\) terms of the sigma matrix, obtained from MAD-X.}
    \label{figure:ip_ellipses_from_coupling}
\end{figure}

Instantaneous luminosities calculated for~\cref{figure:lhc_vs_hllhc_beam_size_and_lumi_growths}, in the absence of crossing angles, are done so according to~\cite{CERN:Herr:Concept_Luminosity}:

\begin{equation}
    \mathcal{L} = \frac{N_1 N_2 f_{rev} N_b}{2 \pi \sqrt{\left( \sigma_{x, 1}^2 + \sigma_{x, 2}^2 \right)} \sqrt{\left( \sigma_{y, 1}^2 + \sigma_{y, 2}^2 \right)}} ,
    \label{equation:luminosity_double_beams}
\end{equation}
where \(N_{n}\) is the number of protons per bunch in beam \(n\), \(f_{rev}\) the revolution frequency of particles, \(N_b\) the number of bunches per beam and \(\sigma_{z, n}\) is the size at the IP of beam \(n\) in the transverse plane \(z\), calculated according to \cref{equation:lebedev_beam_size}.

\todo{What else to add here? I should say a bit that we want to look into getting coupling control at the IP, for the reasons. This transitions into the next section.}
\todo{Lorem ipsum dolor sit amet, consectetur adipiscing elit. Nulla feugiat diam at elit consectetur, non auctor lorem fringilla. Nullam nec dolor nec leo maximus ultrices ut eu dui. Quisque egestas orci sed ante tempor, sed bibendum quam placerat. Pellentesque vel vulputate risus, nec laoreet nulla. Phasellus consectetur nulla nibh, at malesuada lectus consequat non. Praesent luctus, tellus a ornare vulputate, lorem mauris sollicitudin ante, et iaculis arcu risus ac mi. Fusce ac diam vel justo volutpat pretium in imperdiet diam. Vivamus efficitur ante eget arcu sollicitudin efficitur. Cras leo massa, tincidunt nec placerat in, lobortis et est. Praesent ultricies nunc eget ullamcorper sodales. Aenean lobortis tellus eu purus elementum euismod. Cras tempus mattis massa, in sollicitudin nulla. Cras eleifend maximus consequat. Morbi ut nibh eu ligula scelerisque consequat. Donec vel feugiat dolor.}

\section{Current Correction Methods and Their Limitations}
\label{section:current_correction_methods_and_their_limitations}

While the coupling RDTs contain information on the coupling situation in the machine, looking at their patterns along the ring is not a good indicator of the situation in a specific location.
For instance, \cref{figure:guess_rdts} shows the reconstructed coupling RDTs from two measurements taken during the LHC \num{2022} commissioning.
One of these measurements corresponds to a \qty{20}{\percent} lower measured luminosity at IP\num{1} compared to the other, but it is not possible to tell which is which from looking at the RDTs alone.

\begin{figure}
    \centering
    \includegraphics*[width=0.99\linewidth]{Figures/IR_Coupling_Correction/similar_rdts_different_ip1_lumi.pdf}
    \caption{Similarly looking coupling RDTs from two measurements (top and bottom) taken during the LHC \num{2022} commissioning. One scenario leads to a \qty{20}{\percent} instantaneous luminosity decrease at IP\num{1} compared to the other. Can you tell which is which?}
    \label{figure:guess_rdts}
\end{figure}

The Segment-by-Segment technique mentioned in \cref{section:local_ir_coupling} and used to implement local corrections in the LHC IRs suffers from inherent limitations making it unsuitable for the correction of coupling at the IP.
Firstly, due to unfavorable phase advances in between BPMs in the IRs, it is difficult to get a good measurement of the coupling RDTs in these regions.
As these are reconstructed from the \(h_z^\pm\) coordinates they require reconstruction of the momentum (see \cref{subsection:reconstruction_linear_coupling_rdts}).
Knowing that the phase advance in the IRs is \(\sim 0\) from BPM to BPM, and \(\sim \pi\) from one side of the IP to the other, one can see through \cref{equation:momentum_from_two_bpms} why the momentum reconstruction is difficult at BPMs around an IP.

As a consequence, the reconstruction of coupling RDTs in close proximity to the IPs is inaccurate.
\Cref{figure:beamtest_vs_2022_sbs_abs_f1001_ir1} shows the amplitude of the coupling RDTs propagated with the SbS technique in the IR\num{1} segment, from measurements taken during the LHC \num{2021} beam tests and \num{2022} commissioning.
Not only can large error bars can be noticed on the reconstructed data points, but no given case appears to be specifically better than the other while the orange line (\num{2022} commissioning) corresponds to a better correction than the blue one (\num{2021} beam tests).
Given the data patterns and the fact that the orange case corresponds to a better correction, it does not appear that the SbS technique gives a good indication of the local coupling at the IP.

\begin{figure}[!htb]
    \centering
    \includegraphics*[width=0.9\columnwidth]{Figures/IR_Coupling_Correction/sbs_coupling_b1_ir1_compare_2021_2022_colin_delta_minus4.pdf}
    \caption{Propagation of the measured \(|f_{1001}|\) and \(|f_{1010}|\) for beam \num{1} around IP\num{1} (dashed vertical line), measured with two different correction settings. The \num{2022} measurement (\textcolor{mplorange}{orange}) leads to a beam size smaller by \qty{9.2}{\percent} than the \num{2021} one (\textcolor{mplblue}{blue}).}
    \label{figure:beamtest_vs_2022_sbs_abs_f1001_ir1}
\end{figure}

A candidate for a better observable that was considered are the \intro{combined coupling RDTs}~\cite{PRAB:Franchi:First_Simultaneous}, or simply Combined RDTs (CRDTs), \(|\hat{F}_{XY}|\) and \(|\hat{F}_{YX}|\).
These can be expressed from the coupling RDTs, here with a scaling factor, as:

\begin{equation}
    \begin{aligned}
        \hat{F}_{xy} &= \frac{\sinh{2 \mathcal{P}}}{\mathcal{P}} \left( f^{\ast}_{1001} - f^{\ast}_{1010} \right)  \text{ ,} \\
        \hat{F}_{yx} &= \frac{\sinh{2 \mathcal{P}}}{\mathcal{P}} \left( f_{1001} + f^{\ast}_{1010} \right)         \text{ ,}
    \end{aligned}
    \label{equation:combined_coupling_rdts}
\end{equation}
where \(2 \mathcal{P} = \sqrt{\abs{2 f_{1010}}^2 - \abs{2 f_{1001}}^2}\) and \(^{\ast}\) denotes the complex conjugate.
These have the advantage that they can be reconstructed directly from the particle coordinates (\(x,y\)) without the need for momentum reconstruction~\cite{PRAB:Hofer:Coupling_Local_Observables}.
Although the CRDTs seemed to work well in straightforward simulations, they were found to not be useful when it came to applying them to more realistic cases or real measurement data.
\Cref{figure:crdt_fxy_nominal_vs_colin_md_2018} shows the reconstructed CRDT \(|\hat{F}_{XY}|\) from measurements done in a late \num{2018} MD.
During the MD the first investigations were made on local coupling at the IP, and while little data was collected overall due to a dump of beam~\num{1} it provides a good test bed for a new observable candidate.
Information about the MD fill can be found in \cref{appendix:measurement_fills}.

\begin{figure}[!htb]
    \centering
    \includegraphics*[width=0.95\columnwidth]{Figures/IR_Coupling_Correction/crdt_fxy_nominal_vs_colin_md_2018.pdf}
    \caption{Reconstructed CRDT \(|\hat{F}_{XY}|\) around IP\num{2} from measurements at \(\beta^{\ast} =\)~\qty{50}{\centi\meter} during a \num{2018} MD. The \textcolor{mplblue}{blue} lines correspond to \num{2} measurements with the nominal optics, and \textcolor{mplorange}{orange} lines correspond to \num{3} measurements with a local coupling bump implemented around IP\num{2}. The percentages indicate the strength of the AC dipole kicks.}
    \label{figure:crdt_fxy_nominal_vs_colin_md_2018}
\end{figure}

While the error bars are barely visible compared to the large ones in \cref{figure:beamtest_vs_2022_sbs_abs_f1001_ir1}, the main observed issue was a lack of reproducibility between different measurements: different measurements done with identical settings give different results.
For this reason, the CRDTs were not considered further.

Furthermore, the SbS technique does not allow one to differentiate the contribution of one individual corrector from the other in the LHC IRs, making it difficult to find the correct balance of left and right powering settings.
Additionally, as the coupling RDTs are reconstructed at BPMs the method cannot provide a measurement to estimate the coupling at the IP as there are no BPMs at the location.

The usual technique to get a measurement of \(\beta\)-functions at the IP is k-modulation.
Unfortunately, previous studies have shown that k-modulation measurements are robust against the presence of local coupling, both analytically~\cite{PRAB:Hofer:Coupling_Local_Observables, PRAB:Carlier:KModulation_HiLumi} and experimentally in a dedicated Machine Development (MD)~\cite{CERN:Persson:Local_Coupling_IP}.
This prevents the possibility of directly measuring the beam size variation at an IP from local coupling to drive a correction. 

To summarize, existing methods allow compensating for the IR's contribution to global coupling but do not provide a reliable way to drive a minimization of coupling at the IP.
To achieve this goal, two new things are needed: a way to adjust for the coupling at the IP specifically, and a good observable to rely on to drive the correction.
The former can be achieved with a knob described below, while the latter was achieved with a new optics configuration presented in \cref{section:rigid_waist_shift_for_local_coupling_correction}.

\subsubsection*{The Colinearity Knob}

The \intro{colinearity knob} is a powering setting convention for the IR skew quadrupole correctors.
Originally designed for a flat optics MD~\cite{CERN:Fartoukh:First_LHC_Flat_Optics_High_Intensity}, the knob acts anti-symmetrically on the left and right corrector magnets.
Since for round optics in the LHC the \(\beta\)-functions are by design identical at the MQSX position left and right of the IP (see \cref{figure:lhc_ir5_zoomed}) and the phase advance between the two magnets is \(\pi\), assuming a good \(\beta\)-beating correction the knob induces a closed coupling bump going from corrector to corrector around the IP without impacting the machine's global coupling.
The definition of the knob is given in \cref{table:colinearity_knob}.
A full definition of the knobs as implemented and used in the LHC can be found in \cref{section:colinearity_knobs_lsa}.

\begin{table}[!hbt]
    \centering
    \begin{tblr}{colspec={ccccc}}
        \hline
        \textbf{Magnet}                     &  \textbf{K\(_{1S}\) [m\(^{-2}\)]}  \\
        \hline
        MQSX.3R[IP] \(\rightarrow K_{1S}\)  &  \num{1E-4}                        \\
        MQSX.3L[IP] \(\rightarrow K_{1S}\)  &  \num{-1E-4}                       \\
        \hline
    \end{tblr}
    \caption{Definition of one unit of the colinearity knob, a powering setting of the IR skew quadrupole correctors.}
    \label{table:colinearity_knob}
\end{table}

\todo{\Cref{figure:colinearity_knob_effect} bla bla bla. A similar plot can be done for \(f_{1010}\).}

\begin{figure}[!htb]
    \centering
    \includegraphics*[width=0.95\columnwidth]{Figures/IR_Coupling_Correction/colinearity_knob_effect.pdf}
    \caption{Amplitude of the \(f_{1001}\) RDT in the vicinity of IP\num{1} under various strengths of the colinearity knob in the absence (\textcolor{mplblue}{blue}, \textcolor{mplorange}{orange} and \textcolor{mplpurple}{purple}) and presence (\textcolor{mplred}{red}) of global coupling in the machine. The location of the MQSX magnets are highlighted as \textcolor{mqsx_green}{green} vertical lines.}
    \label{figure:colinearity_knob_effect}
\end{figure}

\todo{Say thanks to it we can change the coupling in between the MQSXs without changing the compensation of IR coupling, which is a way to adjust for coupling specifically at the IP.}

\section{Rigid Waist Shift for Local Coupling Correction}
\label{section:rigid_waist_shift_for_local_coupling_correction}

\subsection{Rigid Waist Shift}

\subsection{Concept and Simulations}

\begin{figure}
    \centering
    \includegraphics[width=0.9\textwidth]{Figures/IR_Coupling_Correction/rigid_waist_shift_effect_combined.pdf}
    \caption{Simulated effect of the designed rigid waist shift knob on the \(\beta^{\ast} =\) \qty{30}{\centi\metre} collision optics. The \textcolor{mplblue}{blue} line represents the waist displacement from the design location and is almost completely linear with the knob setting. The \textcolor{mplorange}{orange} and \textcolor{mplgreen}{green} lines represent the horizontal and vertical \(\beta\)-functions change at the IP point itself as the waist is displaced, respectively. The minima of the parabolas are not found at the zero knob setting as the design optics include a very small waist.}
    \label{figure:rigid_waist_shift_knob_effect1}
\end{figure}

\begin{figure}[!htb]
    \centering
    \includegraphics*[width=\columnwidth]{Figures/IR_Coupling_Correction/waist_shift_leaks_rdts.pdf}
    \caption{Linear coupling RDTs in the vicinity of IP\num{1} under a coupling bump, with (\textcolor{mplr}{red}) and without (\textcolor{mplb}{blue}) an RWS. The vertical \textcolor{mqsx_green}{green} lines represent the positions of the skew quadrupoles correctors (MQSX.\num{3}[RL]\num{1}) used to implement the coupling bump. A colinearity knob setting of \num{10} and a rigidity waist shift knob setting of \num{1} were used, which results in a \qty{0.5}{\percent} change in the triplet powering knob and a \qty{43.5}{\centi\meter} shift of the beam waists.}
    \label{figure:rdt_leak}
\end{figure}

\begin{figure}[!htb]
    \centering
    \includegraphics*[width=0.99\columnwidth]{Figures/IR_Coupling_Correction/colin_knob_vs_waist_shift.pdf}
    \caption{Impact of the colinearity knob on the global \(\abs{C^{-}}\), calculated according to \cref{equation:deltaqmin_from_f1001}, with and without applying an RWS.}
    \label{figure:knob_to_cminus_with_waist}
\end{figure}

\Cref{figure:cminus_colin_vs_tilt_with_waist} shows the values of the resulting \(\abs{C^{-}}\) across the parameter space when an RWS is applied.

\begin{figure}[!htb]
    \centering
    \includegraphics*[width=0.99\columnwidth]{Figures/IR_Coupling_Correction/cminus_colin_tilt_compensation_with_waist.pdf}
    \caption{Resulting \(\abs{C^{-}}\) (\cref{equation:deltaqmin_from_f1001}) for various combinations of tilt error and colinearity knob settings, when applying an RWS.}
    \label{figure:cminus_colin_vs_tilt_with_waist}
\end{figure}

\Cref{figure:beam_size_colin_vs_tilt_no_waist} shows the resulting beam size increase as a ratio to the nominal beam size across the same parameter space, highlighting that minimization of the growth is possible though a wrong setting would enhance the phenomenon.

\begin{figure}[!htb]
    \centering
    \includegraphics*[width=0.99\columnwidth]{Figures/IR_Coupling_Correction/ip_beam_size_growth_colin_tilt_compensation_no_waist.pdf}
    \caption{Resulting beam size (\cref{equation:lebedev_beam_size}) increase for identical settings of tilt error and colinearity knob settings as \cref{figure:cminus_colin_vs_tilt_with_waist}, but without an RWS.}
    \label{figure:beam_size_colin_vs_tilt_no_waist}
\end{figure}

\subsection{Determining Corrections}

Blah.

\subsubsection*{Rigid Waist Shift Procedure}

Bleh.

%----------------------------------------------------------------------------------------

\section{Experimental Results from the LHC \num{2022} Commissioning}
\label{section:rws_experimental_results}

Can refer to appendix \Cref{appendix:experimental_knobs} for the fills used.

\begin{table}[!htb]
    \centering
    \caption{Luminosity gains observed at the main experiments ATLAS and CMS from the method's suggested corrections.}
    \begin{tblr}{colspec={ccc}}
        \hline
        \SetCell[r=2,c=1]{m,c} \textbf{Experiment} & \SetCell[c=2]{c} \textbf{Luminosity Gain [\unit{\percent}]}                     \\
        \cline{2,3}                                &    \(\beta^{\ast} = \) \qty{30}{cm}    &    \(\beta^{\ast} = \) \qty{42}{cm}    \\
        \hline
        \textbf{ATLAS (IP\num{1})}                 &    \num{9.7}                           &     \num{5.2}                          \\
        \textbf{CMS (IP\num{5})}                   &    \num{3.5}                           &     \num{1.5}                          \\
        \hline
    \end{tblr}
    \label{table:rws_lumi_gains}
\end{table}

\section{Operation with Limited Correctors Availability}
\label{section:limited_correctors_availability}

Mitigation Options in Case of MQSX Failures
\todo{Uncomment text when I get here, take mostly the structure of the IPAC23 article and project update V. Go deeper in details (see commented out subsections).}

% \subsection{Lifetime Considerations of MQSX Elements}

% Explain that there is a real risk that some of our MQSX magnets die, especially the ones in IR1 (ATLAS).
% In this case, we will need a containment plan, as not only are they used for the but the local corrections they are a part of are a baseline for us to compute higher order terms corrections.

% \Cref{table:correctors_peak_dose} reproduced from slide 23~\cite{Evian21:Cerutti:TripletLifetime}:

% \begin{table}[!htb]
%     \centering
%     \caption{Expected total received dose of the corrector magnets located in the triplets for the main IRs. Table reproduced from~\cite{Evian21:Cerutti:TripletLifetime}. The entries marked with \asterisk assume an IR\num{1} polarity inversion in the middle of \num{2025}.}
%     \begin{tblr}{colspec={ccc}}
%         \hline
%         \SetCell[r=3,c=1]{m,c} \textbf{Magnets}             &  \SetCell[c=2]{c} \textbf{Peak Dose [\unit{\mega\gray}]}                                                                                     \\
%         \cline{2,3}                                         &  With ATLAS Variable (Fixed) Angle                        &    +\num{2025} (as \num{2023}/\num{2024})                                        \\
%         \cline{2,3}                                         &  After \qty{395}{\femto\barn^{-1}}                        &    After \qty{480}{\femto\barn^{-1}}                                             \\
%         \hline
%         \textcolor{red}{\textbf{MCBX\num{1} (IR\num{1})}}   &    \textcolor{red}{\num{8.5} (\num{8.5})}                 &     \textcolor{red}{\num{11} (\num{11}) $/$ \asterisk \num{10.5} (\num{10.5})}   \\
%         \textcolor{red}{\textbf{MCBX\num{1} (IR\num{5})}}   &    \num{6}                                                &     \textcolor{red}{\num{7.5}}                                                   \\
%         \textbf{MCBX\num{2} (IR\num{1})}                    &    \num{3.5} (\num{3.5})                                  &     \num{4} (\num{4}) $/$ \asterisk \num{4} (\num{4})                            \\
%         \textbf{MCBX\num{2} (IR\num{5})}                    &    \num{2}                                                &     \num{2.5}                                                                    \\
%         \textcolor{red}{\textbf{MQSX (IR\num{1})}}          &    \textcolor{red}{\num{7.5} (\num{7.5})}                 &     \textcolor{red}{\num{9} (\num{9}) $/$ \asterisk \num{9} (\num{9})}           \\
%         \textcolor{red}{\textbf{MQSX (IR\num{5})}}          &    \textcolor{red}{\num{8} (\num{8})}                     &     \textcolor{red}{\num{9.5} (\num{9.5})}                                       \\
%         \textbf{MCBX\num{3} (IR\num{1})}                    &    \num{5} (\num{5})                                      &     \num{6} (\num{6.5}) $/$ \asterisk \num{6} (\num{6})                          \\
%         \textbf{MCBX\num{3} (IR\num{5})}                    &    \num{3}                                                &     \num{3.5}                                                                    \\
%         \hline
%     \end{tblr}
%     \label{table:correctors_peak_dose}
% \end{table}

% \todo{Important quote from the end of the presentation:}
% "Assuming a limit of \qty{6}{\mega\gray} for the corrector magnets in the triplet, this is expected to be reached in the four \(\mathrm{MCBX.1}\) and four \MQSX by the end of \num{2024}."

% This means in simple terms that MQSX dying will drastically impact the LHC's operations, and potentially shut the machine down.

% \subsection{Tilt of Triplet Elements}

% Talk about what we want to do (tilt Q3 or Q2) to induce a skew component + simulation results.
% Found settings of the Q3 or Q2 that would negate the MQSX one determined in beam test / commissioning.
% Show some plots, and come up with the tilt values that would be needed to do the compensation.

% Also show we have a very minimal beta-beating from this.
% Show we have had a look at different optics (30cm and 1m betastar) and it works for both.

% \subsubsection*{Operational Constraints}

% The LHC systems are not meant for this, but for vertical alignment of these magnets!
% System relies on bellows: 2 pieds IP side and 1 pied other side for Q2 for instance.
% This means that inducing rotation is not only not the design purpose, but also imperfect (on move les 2 pieds pour essayer de mimer une rotation mais c'est pas parfait).
% Would be very good to have a plot of the assemblies here to show what I mean. Ask MP people? See in the LHC design report?

% It is considered by MP people to be quite dangerous to do this unless forced to (read an MQSX dies), especially in cold mass, as if we damage the belows then we're in for 1 year of shutdown to repair it.
% Say that for these reasons we decided not to test this concept in the machine, unfortunately.

% Could show plots here (see "Living with Local Coupling" section of my project update V) about the effect on luminosity: what reduction are we looking at 

% \subsection{Warm Skew Quadrupole Replacement}

% Here talk about how there is space between D1 and TAN for a magnet, and we could put a warm skew quadrupole there.
% Show some schematic.
% Show some calculations of how the gradient should be, how long the element should be etc.

% It has considerations such as being imbalanced (asymmetric) with the remaining MQSX, limiting the available aperture etc.

% \subsection{Feed-Down from High Order Correctors}

% Mention here that we consider using sextupolar and octupolar (MCSSX and MCSX) magnets to generate feed-down to coupling.
% However, these elements are not strong enough to generate the same effect that the MQSX do, so this is not really an option.

% \subsection{Adapting the Optics Squeezing Scheme}

% We could change things in the squeeze to compensate for the absence of an MQSX.
% Potentially squeeze harder on one IP (the one with a missing element).
% Potentially squeeze similarly for both but when the unaffected IP stops the squeeze, the other one keeps going to lower \betastar, and the lost luminosity at beginning of fill is made up for starting this moment.
% Not cool because LHCb prefers long fill + impact on BBLR?

% Plot to show the ratio of luminosity as in PowerPoint?

% \subsection{Potential MDs / Exp. Results}

% Talk about how this could be tested.
% If we do get MD blocks for this in September, include the data here.

% %----------------------------------------------------------------------------------------

\section{Conclusions}

%----------------------------------------------------------------------------------------