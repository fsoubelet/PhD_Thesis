\chapter{Optics Measurements and Corrections at the LHC}
\label{chapter:lhc_omc}

The \gls{LHC} is a \qty{26.659}{\kilo\metre} synchrotron collider located at the \gls{CERN}, on the French-Swiss border.
It is part of \gls{CERN}'s Accelerator Complex, illustrated in \cref{figure:cern_accelerator_complex}, a chain of particle accelerators progressively bringing protons and heavy ions up to an energy of \qty{6.8}{\tera\electronvolt} per beam as of \num{2023}.

\begin{figure}[!htb]
  \centering
  \includegraphics*[width=0.9\linewidth]{Figures/Optics_Measurements_Corrections_at_LHC/cern_accelerator_complex.png}
  \caption{The CERN Accelerator Complex in \num{2022}, not to scale~\cite{Website:CERN_Accelerator_Complex_Resource}. For typical LHC operation, a proton beam is produced in LINAC\num{4} and follows the chain: LINAC\num{4} \(\rightarrow\) PSB \(\rightarrow\) PS \(\rightarrow\) SPS \(\rightarrow\) LHC.}
  \label{figure:cern_accelerator_complex}
\end{figure}

Particles go through a chain of different particle accelerators before reaching their experimental destinations.
For protons colliding in the \gls{LHC}, the first step is a linear accelerator, \acrshort{LINAC}\num{4}, which accelerates them up to a kinetic energy of \qty{160}{\mega\electronvolt}.
Next, the protons are injected into the \gls{PSB}, where they are accelerated to an energy of \qty{1.4}{\giga\electronvolt}.
The next stages are the \gls{PS}, in which they will reach \qty{25}{\giga\electronvolt}; then the \gls{SPS} where they are accelerated to \qty{450}{\giga\electronvolt} before being finally injected into the \gls{LHC}.

The \gls{LHC} circulates two counter-rotating \gls{hadron} \glspl{beam}, each in their ring, which are made to collide at four \glspl{IP} to provide data for \gls{HEP} \glspl{experiment}.
The main data-taking experiments on the \gls{LHC} are \acrshort{ATLAS}~\cite{ATLAS_Paper,Website:ATLAS,Website:ATLAS_CDS}, \acrshort{ALICE}~\cite{ALICE_Paper,Website:ALICE,Website:ALICE_CDS}, \acrshort{CMS}~\cite{CMS_Paper,Website:CMS,Website:CMS_CDS}, and \acrshort{LHCb}~\cite{LHCb_Paper,Website:LHCb,Website:LHCb_CDS}; with other notable experiments being LHCf~\cite{LHCf_Paper,Website:LHCf,Website:LHCf_CDS}, MATHUSLA~\cite{MATHUSLA_Paper,Website:MATHUSLA,Website:MATHUSLA_CDS}, FASER~\cite{FASER_Paper,Website:FASER,Website:FASER_CDS}, SND~\cite{SND_Paper,Website:SND,Website:SND_CDS}, TOTEM~\cite{TOTEM_Paper,Website:TOTEM,Website:TOTEM_CDS}, and MoEDAL~\cite{MoEDAL_Paper,Website:MOEDAL,Website:MOEDAL_CDS}.
The \gls{LHC} is currently the world's highest energy particle accelerator, colliding beams at \qty{13.6}{\tera\electronvolt} center-of-mass energy as of \Gls{run}~\num{3}, \num{2023}.

%----------------------------------------------------------------------------------------

\section{The LHC Lattice}
\label{section:lhc_lattice}

The \gls{LHC} lattice consists of eight \concept{octants} each intersected by an \gls{IR}.
Conventionally, the segment between two \glspl{IR} is called an \concept{arc} and the arc between IR1 and IR2 is named Arc12, and similarly for other arcs.
An octant is defined as going from mid-arc to mid-arc around a given \gls{IR} which is located at its center.
Each octant is named according to the \gls{IR} at its center: the octant with IR\num{1} at its center is named Octant1, and similarly for other octants.
An illustration and a detailed description on naming conventions can be found in \cref{appendix:naming_conventions}.

\begin{figure}[!htb]
  \centering
  \includegraphics*[width=0.81\linewidth]{Figures/Optics_Measurements_Corrections_at_LHC/lhc_schematic.pdf}
  \caption{Schematic of the LHC layout, adapted from~\cite{PHD:Poyet}.}
  \label{figure:lhc_schematic_layout}
\end{figure}

Beam~\num{1} rotates clockwise in its ring when viewing the \gls{LHC} from above, and Beam~\num{2} rotates counter-clockwise as viewed from above.
The beams occupy separate apertures, or beam pipes, except in the \glspl{IR} where they are made to collide.
The layout of the \gls{LHC} can be seen in simplified schematic form in~\cref{figure:lhc_schematic_layout}, and full details can be found in the \gls{LHC} Design Report~\cite{BOOK:Bruning:LHC_Design_Report_Main_Ring,BOOK:Bruning:LHC_Design_Report_Infrastructure,BOOK:Benedikt:LHC_Design_Report_Injector_Chain}.

\subsection{The LHC Arcs}
\label{subsection:lhc_arcs}

Each arc in the \gls{LHC} is made up of 23 cells and is approximately \qty{2.45}{\kilo\meter} long.
The layout of an \gls{LHC} arc cell is given in~\cref{figure:lhc_schematic_arc_cell}, and a clearer schematic representation can be found in~\cite{MASTERS:Keintzel:Arc_Cell_Options_HELHC}.
The cell is based on an FBDB (FODO with Bends) layout alternating focusing and defocusing quadrupoles interspaced with dipoles.
These elements are all superconducting and are commonly labeled \concept{MQF}, \concept{MQD} and \concept{MB}, respectively.

\begin{figure}[!hbt]
  \centering
  \includegraphics*[width=0.99\linewidth]{Figures/Optics_Measurements_Corrections_at_LHC/lhc_schematic_arc_cell.pdf}
  \caption{Schematic of an LHC arc cell~\cite{BOOK:Bruning:LHC_Design_Report_Main_Ring}.}
  \label{figure:lhc_schematic_arc_cell}
\end{figure}

Each cell contains two \concept{MQ} (one MQF, one MQD) with three MB in between, for a total of 6 MBs per cell.
The MBs are all powered in series and, for size constraint reasons~\cite{BOOK:Bruning:LHC_Design_Report_Main_Ring}, are of a dual bore design.
The MQs are themselves also powered in series but split in two families: one power circuit is dedicated to MQF magnets and another circuit for the MQD magnets, where each arc holds a circuit for each family.
As a consequence, these elements can only be trimmed in groups.

As a part of the main assemblies are superconducting \concept{spool piece magnets}, correctors used for the local compensation of magnetic errors in the main arc magnets~\cite{BOOK:Bruning:LHC_Design_Report_Main_Ring}.
These include sextupole correctors, sextupolar spool pieces named \concept{MCS} and mounted on the ends of every main dipole, used to correct \(b_3\) errors of the MBs.
Similarly, octupole and decapole spool pieces are included and used for the compensation of \(b_4\) and \(b_5\) errors in the main arc magnets, respectively.
The octupole correctors are named \concept{MCO} while the decapole correctors are named \concept{MCD}, and both are nested together in an assembly named \concept{MCDO} which is mounted on the end of every second MB.
The spool piece magnets in the LHC are single aperture and powered in series similarly to the MQs, with one circuit assigned for each magnet family.

In addition to spool piece magnets, linear and non-linear \concept{lattice correctors} are mounted on the main arc quadrupoles MQs.
These lattice correctors are powered in series per family, and independently for each beam.
Horizontal and vertical orbit correctors, respectively \concept{MCBH} and \concept{MCBV}, are installed at each focusing and defocusing MQ.
Normal \gls{trim} quadrupoles, named \concept{MQT}, are primarily used for tune correction.
In each arc four MQTs are rotated by \qty{45}{\degree} to form \gls{skew} quadrupoles, named \concept{MQS}, used for \gls{betatron_coupling} correction.
\Gls{normal} and \gls{skew} sextupoles \concept{MS} and \concept{MSS}, used for natural chromaticity and chromatic coupling correction respectively, are mounted on the MQs.
Landau octupoles \concept{MO} provide damping of coherent oscillations, and are split into two families (focusing and defocusing) powered in series, such that there are two families per arc and per beam.
\break

\Cref{figure:lhc_arc_cell_latwiss} shows a simplified layout of an LHC arc cell's elements as well as \(\beta\) and dispersion functions for \num{2022} optics at \(\beta^{\ast} =\) \qty{30}{\centi\meter}.
In the layout (top) part of the plot element powerings are indicated, with MBs in \textcolor{mplblue}{blue}, MQs in \textcolor{latwiss_red}{red}, MSs in \textcolor{latwiss_yellow}{yellow}, MOs in \textcolor{latwiss_green}{green} and beam position monitors (BPMs) are indicated as grey patches.
Note that not all elements are indicated there.
\Cref{figure:lhc_arc23_latwiss} shows a similar plot but across LHC arc\num{23} for the same optics.

\begin{figure}[!hbt]
  \centering
  \includegraphics*[width=\linewidth]{Figures/Optics_Measurements_Corrections_at_LHC/lhc_arc_cell.pdf}
  \caption{Simplified layout (top) and optics functions (bottom) of an LHC arc cell for the \(\beta^{\ast} =\) \qty{30}{\centi\meter} optics.}
  \label{figure:lhc_arc_cell_latwiss}
\end{figure}

\begin{figure}[!hbt]
  \centering
  \includegraphics*[width=\linewidth]{Figures/Optics_Measurements_Corrections_at_LHC/lhc_arc23.pdf}
  \caption{Simplified layout (top) and optics functions (bottom) of the LHC arc\num{23} for the \(\beta^{\ast} =\) \qty{30}{\centi\meter} optics.}
  \label{figure:lhc_arc23_latwiss}
\end{figure}

The purpose of the arcs is that of beam transport to the more purpose-specific parts of the machine located in the \glspl{IR}.

\subsection{The LHC Experimental Interaction Regions}
\label{subsection:lhc_eirs}

In the middle of each octant, in between arcs, the \gls{LHC} hosts \concept{long straight sections} (see \cref{appendix:naming_conventions} for details) with specific purposes.
Each of these is centered around an \acrfull{IR} where a dedicated layout is in place to fulfill the section's purpose.
The purpose of each straight section is briefly stated on \cref{figure:lhc_schematic_layout} and detailed in \cref{table:lhc_straight_sections}.
\\

\begin{table}[!hbt]
  \centering
  \begin{tblr}{colspec={cc}}
      \hline
      \text{Straight Section} & \text{Description}                                   \\
      \hline
      IR\num{1}               & ATLAS Experiment                                     \\
      IR\num{2}               & ALICE Experiment and B\num{1} Injection              \\
      IR\num{3}               & Momentum Cleaning (Collimation)                      \\
      IR\num{4}               & RF Systems and LHC Instrumentation                   \\
      IR\num{5}               & CMS Experiment                                       \\
      IR\num{6}               & Beam Dump System                                     \\
      IR\num{7}               & Betatron Cleaning (Collimation)                      \\
      IR\num{8}               & LHCb Experiment, and B\num{2} Injection              \\
      \hline
  \end{tblr}
  \caption{Description and purpose of the straight sections in the LHC. Out of the experiments only the four major ones are mentioned.}
  \label{table:lhc_straight_sections}
\end{table}

Of interest to this thesis are the \concept{experimental insertions}, located in IR\num{1}, IR\num{2}, IR\num{5}, and IR\num{8}, where the beams are made to collide.
An insertion region in which beams are made to collide is called an \concept{Interaction Region}, or sometimes Experimental Interaction Region.

\begin{noteblock}
  Though in the strict sense the short form \gls{IR} stands for Insertion Region, it is commonly used to refer to an Interaction Region.
  It is the case in this document, where when used \gls{IR} should be taken as Interaction Region.
\end{noteblock}

At the center of the \gls{IR}, beams are made to collide at the \acrfull{IP}.
In order to achieve high luminosity during collisions, and as shown in \cref{section:luminosity}, the \glspl{beta-function} at the \glspl{IP} are squeezed to very small values.
\Cref{figure:ir5_and_around} shows the \(\beta\)-functions in the LHC around IP5 at both injection and collision optics, where the effect of the squeeze is apparent.

\begin{figure}[!hbt]
  \centering
  \includegraphics*[width=\linewidth]{Figures/Optics_Measurements_Corrections_at_LHC/ir5_surroundings_optics_2.pdf}
  \caption{The horizontal and vertical \(\beta\)-functions in the LHC around IP5 at injection optics (top) and collision optics (bottom). Notice the drastically different scales on the vertical axes.}
  \label{figure:ir5_and_around}
\end{figure}

During normal operation for collisions the \glssymbol{betastar} at \acrshort{ATLAS} and \acrshort{CMS} is squeezed down to \(\beta^{\ast} =\) \qty{30}{\centi\metre}.
In this configuration, at \acrshort{ALICE} and \acrshort{LHCb} the \(\beta^{\ast}\) are only squeezed to higher values, \qty{10}{\meter} and \qty{2}{\meter} respectively in \num{2023}.
During collisions involving ions (Pb-Pb and p-Pb) the \(\beta^{\ast}\) is reduced at \acrshort{ALICE} and \acrshort{LHCb}.
\Cref{table:lhc_betastars_configurations} summarizes the \(\beta^{\ast}\) values for the different experiments and configurations.

\begin{table}[!htb]
  \centering
  $\begin{tblr}{colspec={cccc}}
      \hline
      \SetCell[r=2,c=1]{m,c} \text{IP} & \SetCell[c=3]{c} \text{Lowest } \beta^{\ast}                                     \\
      \cline{2-4}
                                       &  \text{Injection Optics}  &  \text{Proton Collisions} &  \text{Ion Collisions}   \\
      \hline
      \text{IP\num{1}}                 &  \qty{11}{\meter}         &  \qty{30}{\centi\meter}   &  \qty{50}{\centi\meter}  \\
      \text{IP\num{2}}                 &  \qty{11}{\meter}         &  \qty{10}{\meter}         &  \qty{50}{\centi\meter}  \\
      \text{IP\num{5}}                 &  \qty{11}{\meter}         &  \qty{30}{\centi\meter}   &  \qty{50}{\centi\meter}  \\
      \text{IP\num{8}}                 &  \qty{11}{\meter}         &  \qty{2}{\meter}          &  \qty{50}{\centi\meter}  \\
      \hline
  \end{tblr}$
  \caption{Value of the \(\beta^{\ast}_{x,y}\) at different IPs for different optics configurations as of Run~\num{3}.}
  \label{table:lhc_betastars_configurations}
\end{table}

In order to achieve a small \(\beta^{\ast}\) at the IPs, the beams are focused using a superconducting \concept{triplet} of quadrupoles just before the IP, on either side~\cite{CERN:Ostojic:Improved_Optical_System_LHC_Triplet}.
The triplet is optimized to be symmetric~\cite{CERN:DAmico:Analysis_Generic_Insertions}, with Q\num{1} and Q\num{3} being the same length at \qty{6.3}{\meter} and Q\num{2} split into two sub-magnets Q\num{2}a and Q\num{2}b of \qty{5.5}{\meter} each.
All three magnets are powered in series but can be adjusted individually using dedicated \gls{trim} converters~\cite{PAC:Bordry:LHC_Inner_Triplet_Powering}.

This arrangement of three quadrupoles allows for a strong focusing of the \(\beta\)-functions in both transverse planes.
However, such an arrangement leads to high \(\beta\)-functions in the triplet quadrupoles themselves and neighboring elements.
An illustration of the area close to IP\num{5} for collision optics with \(\beta^{\ast} =\) \qty{30}{\centi\metre} is given in \cref{figure:lhc_ir5_zoomed}.

\begin{figure}[!hbt]
  \centering
  \includegraphics*[width=\linewidth]{Figures/Optics_Measurements_Corrections_at_LHC/lhc_ir5_zoomed.pdf}
  \caption{The simplified element layout (top) and \(\beta\)-functions (bottom) in the close vicinity of IP\num{5} at \(\beta^{\ast} =\) \qty{30}{\centi\metre} collision optics, without crossing angles.}
  \label{figure:lhc_ir5_zoomed}
\end{figure}

On the layout plot the four \textcolor{latwiss_red}{red patches} closest to the \gls{IP} location correspond to Q\num{1}, Q\num{2}a, Q\num{2}b and Q\num{3} respectively, the triplet quadrupoles.
The \textcolor{latwiss_blue}{blue patches} correspond to D\num{1} (first batch closest to the IP) and D\num{2} (the furthest dipole), the \concept{separation/recombination dipoles} responsible for bringing the beams together/apart in the common region from/to their separate apertures in the arcs.
The separation dipole D\num{1} is made of six \qty{3.4}{\meter} long normal conducting magnets while D\num{2} is a superconducting twin aperture magnet \qty{9.45}{\meter} long.
Further quadrupoles after the triplet are matching quadrupoles and will be discussed later.
The grey lines correspond to the location of \glspl{BPM}, measurement instrumentation.
\break

Due to the large \(\beta\)-functions in the triplet quadrupoles, as can be seen in \cref{figure:lhc_ir5_zoomed}, any magnetic error in the elements of the IR would have a strong impact on the beam dynamics.
To enable correction of these errors, linear and non-linear corrector magnets are installed along the IR, distributed symmetrically around the IP: every corrector magnet on one side of the IP has a counterpart on the other side.
Of interest to this thesis are the \(a_2\) \gls{skew} quadrupole correctors installed just before Q\num{3} on each side of the \gls{IP}, the locations of which are highlighted in \cref{figure:lhc_ir5_zoomed} by \textcolor{mqsx_green}{green vertical lines}.
A schematic of the corrector layout is shown in \cref{figure:lhc_ir_corrector_layout}.
All IR correctors are individually powered magnets.

\begin{figure}[!hbt]
  \centering
  \includegraphics*[width=\linewidth]{Figures/Optics_Measurements_Corrections_at_LHC/corrector_package.png}
  \caption{Layout of the triplet magnets and the linear and non-linear correctors in the LHC experimental insertions~\cite{CERN:Bruning:Dynap_Studies}, showing common aperture magnets. The skew quadrupole correctors correspond to order \(a_2\) and are located in the \textcolor{mplb}{C\num{2}} package.}
  \label{figure:lhc_ir_corrector_layout}
\end{figure}

In order to prevent parasitic crossings between the two beams' bunches around the IP during collisions, \concept{separation bumps} are implemented in a single transverse plane for each IP, in the form of closed orbit bumps.
Due to the presence of these bumps, in order to reach collisions a \concept{crossing angle} is introduced.

The optics in IR\num{1} and IR\num{5} are identical except for the crossing schemes.
In IR\num{1} the crossing angle is in the vertical plane while it is in the horizontal plane in IR\num{5}.
Respectively, the separation bumps are in the horizontal plane in IR\num{1} and in the vertical plane in IR\num{5}.
On top of the opposite planes for the separation bump and crossing angles, the schemes themselves are slightly different.

\Cref{figure:lhc_crossing_schemes_ip15} shows the crossing schemes for both IR\num{1} and IR\num{5} for the \(\beta^{\ast} =\) \qty{30}{\centi\metre} collision optics, with the location of the triplets highlighted in grey and that of the separation dipoles in yellow.

\begin{figure}[!hbt]
  \centering
  \includegraphics*[width=\linewidth]{Figures/Optics_Measurements_Corrections_at_LHC/lhc_crossing_schemes_ip15.pdf}
  \caption{Crossing schemes for IR\num{1} and IR\num{5} at collision optics.}
  \label{figure:lhc_crossing_schemes_ip15}
\end{figure}

Other IRs, of lower interest to this thesis, have significantly different layouts which can be found in details in~\cite{BOOK:Bruning:LHC_Design_Report_Main_Ring,PHD:Vanbavinckhove}.

\subsection{Matching Sections and Dispersion Suppressors}
\label{subsection:matching_sections_dispersion_suppressors}

Assuring the transition between the arcs and the specific \gls{optics} conditions of the \glspl{IR} are \concept{matching sections} and \concept{dispersion suppressors}, the location of which is highlighted on \cref{figure:ir5_and_around}.
Note that the area designated as matching section on this figure also includes the dispersion suppressor.
Together, the two segments are responsible for matching the \gls{Twiss_parameters} between the arcs and the IRs, and for reducing the dispersion to near-zero value at the IP, respectively.

The dispersion suppressor is made of two arc cells containing two instead of the regular three dipoles.
The quadrupoles in these cells, Q\num{7} to Q\num{10}, are powered individually.
The dispersion suppressor leading to IP\num{5} can be seen on \cref{figure:lhc_dispersion_suppressor}, where the beam travels from left to right.

\begin{figure}[!hbt]
  \centering
  \includegraphics*[width=0.99\linewidth]{Figures/Optics_Measurements_Corrections_at_LHC/lhc_dispersion_suppressor.pdf}
  \caption{Simplified layout and optics functions in the dispersion suppressor leading beam~\num{1} to IP\num{5}, for the \(\beta^{\ast} =\) \qty{30}{\centi\meter} optics.}
  \label{figure:lhc_dispersion_suppressor}
\end{figure}

The matching section is made of individually powered superconducting quadrupoles Q\num{4} to Q\num{6}.
These are used to match the Twiss parameters from their values out of the arcs to that at the entrance of the triplets.
In order to help the matching to the arcs the trim quadrupoles QT\num{11} to QT\num{13}, adjacent to the FODO quadrupoles Q\num{11} to Q\num{13}, are also individually powered and used for the matching.
The full segment from the start of the dispersion suppressor to just before separation dipole D\num{2} is shown in \cref{figure:lhc_matching_section}, where again the beam travels from left to right.

\begin{figure}[!hbt]
  \centering
  \includegraphics*[width=0.99\linewidth]{Figures/Optics_Measurements_Corrections_at_LHC/lhc_matching_section.pdf}
  \caption{Simplified layout and optics functions in the matching section and dispersion suppressor leading beam~\num{1} to IP\num{5}, for the \(\beta^{\ast} =\) \qty{30}{\centi\meter} optics.}
  \label{figure:lhc_matching_section}
\end{figure}

\subsection{The ATS Optics Scheme}
\label{subsection:lhc_ats_optics_scheme}

When pushing the \glssymbol{betastar} to smaller values, and therefore the \glspl{beta-function} in the triplets to higher ones, the chromatic effects produced by the triplet quadrupoles~(\cref{equation:natural_chromaticity_approximation}) increase drastically and need to be corrected.
As the beam energy reaches its maximum, the beam size gets smaller and an aperture margin that allows to increase the \(\beta\)-functions appears in the arcs.

The \gls{ATS} optics scheme~\cite{CERN:Fartoukh:ATS_Report,PRAB:Fartoukh:Achromatic_Telescopic_Squeeze,IPAC:Pojer:LHC_ATS_Experience} consists in splitting the reduction of the \glssymbol{betastar} - the squeeze - in two stages.
In the first one, the \concept{pre-squeeze}, the \(\beta^{\ast}\) is reduced using the matching quadrupoles around the affected IP.
As the use of these quadrupoles has several limits (magnet strength, chromaticity correction, orbit control) a second stage is necessary.
In this second stage, the \concept{tele-squeeze}, the \(\beta^{\ast}\) is reduced by using the matching quadrupoles in the nearby IRs: IR\num{2} and IR\num{8} for the tele-squeeze of IR\num{1}, and IR\num{4} and IR\num{6} for the tele-squeeze of IR\num{5}.
Sectors \numlist{81;12;45;56} are therefore called ATS sectors.
This modulation in the second stage sends \(\beta\)-beating waves down the arcs, which make the \glspl{beta-function} peak at the location of sextupoles and octupoles in those arcs, enhancing their efficiency.

The ratio between the peak \(\beta\)-functions induced in the arcs is called \concept{telescopic index}, or tele-index, and is denoted \(r_{Tele}\)~\cite{CERN:Fartoukh:Round_Telescopic_Optics_LHC_Large_Telescopic_Index}.
It is defined as:

\begin{equation}
  r_{Tele} = \beta^{\mathrm{peak}}_{\mathrm{ats}} / \beta^{\mathrm{peak}}_{\mathrm{non-ats}}
  \label{equation:tele_index}
\end{equation}

\Cref{figure:lhc_ats_scheme} shows the \(\beta\)-functions around IP\num{5}, at two different steps in the squeeze, where one can see the \(\beta\)-beating waves in the neighboring ATS sectors \num{45} and \num{56}.

\begin{figure}[!hbt]
  \centering
  \includegraphics*[width=0.96\linewidth]{Figures/Optics_Measurements_Corrections_at_LHC/lhc_ats_wave.pdf}
  \caption{The \(\beta\)-functions in sectors \num{45} and \num{56} at different points in the squeeze for the \num{2022} optics: at the end of the pre-squeeze (top) and at the end of the tele-squeeze (bottom).}
  \label{figure:lhc_ats_scheme}
\end{figure}

This ATS optics scheme has been used in the LHC starting \Gls{run}~\num{2} and allowed reducing the collision optics \glssymbol{betastar} from its design value of \qty{55}{\centi\meter} to \qty{30}{\centi\meter}.
It is the operational baseline of Run~\num{3}.

\section{The Operational Cycle of the LHC}
\label{section:lhc_operational_cycle}

The \gls{LHC} operational cycle~\cite{Report:LHCModes}, illustrated in \cref{figure:lhc_cycle}, begins with a \concept{pre-cycle} of certain magnetic elements~\cite{Report:LHCMagnetsPreCycles}.
A full pre-cycle is only done after an interruption of the machine operations, such as when a technical intervention is performed.
During the pre-cycle no beams are present in the machine and the respective element currents are increased up to several \unit{\tera\electronvolt} beam energy configuration, to ensure the reproducibility of the magnetic fields over successive fills.
The exact nature of the pre-cycle depends on the magnetic elements.%, and a pre-cycle is not necessarily performed before each fill.

After the pre-cycle comes the \concept{injection} stage: beams are injected from the \gls{SPS} at an energy of \qty{450}{\giga\electronvolt}.
First a probe beam consisting of just a single bunch is injected to check the validity of several systems (injection interlock, orbit, tune, chromaticity and coupling control), then a \num{12}-bunches beam, and finally a physics beam meant for collisions is injected.
At injection optics the \(\beta^{\ast}_{x,y}\) at the main colliding \glspl{IP} (IP\num{1} and IP\num{5}) is \qty{11}{\metre}.
The number of bunches, their intensity and their filling pattern~\cite{Report:LHCStandardFillingSchemes} depends strongly on the experimental demands.
For \gls{optics} measurements for instance, between one and three low intensity, non-colliding bunches of about \num{e10} protons per bunch are injected for each beam.
For luminosity production a larger number of high intensity bunches is injected: in the order of a few \num{e3} bunches, with \(\ge\) \num{e11} protons per bunch.

\begin{figure}[!hbt]
  \centering
  \includegraphics*[width=\linewidth]{Figures/Optics_Measurements_Corrections_at_LHC/lhc_cycle.pdf}
  \caption{Simplified illustration of the LHC nominal cycle.}
  \label{figure:lhc_cycle}
\end{figure}

After injection, the beam energy is increased up to collision energy (\qty{6.8}{\tera\electronvolt} in Run~3) while the beams are squeezed and the \(\beta^{\ast}\) reduced.
This process, called \concept{combined ramp and squeeze}, has been used in the LHC since 2017~\cite{IPAC:Camillocci:CombinedRampAndSqueeze}.
Before then the squeezing process only started once the energy had reached collision value.

After reaching top energy, a configuration known as \concept{flat-top}, another \concept{squeeze} is performed to bring the \(\beta^{\ast}\) to collision value.
This is when the \acrshort{ATS} scheme mentioned in \cref{subsection:lhc_ats_optics_scheme} happens.

In a final step before luminosity production, called \concept{adjust}, the last few needed parameters are adjusted to bring the beams into collision: tunes, crossing angles, collapse of the separation bumps.
The resulting configuration, called \concept{stable beams}, is kept throughout the fill for luminosity production.
The fill ends when the beams are extracted from the machine, a.k.a. \concept{beam dump}, after which the cycle ends by a \concept{ramp down} of the magnets' currents.
Some magnets pre-cycle during the ramp down.

The working point is changed several times along the cycle for stability reasons.
As of \num{2022}, at injection the transverse tunes are (\(62.275, 60.293\)).
The working point is brought to (\(62.28, 60.31\)) during the ramp and squeeze, at the end of which it is moved to (\(62.311, 60.318\)).
A final change is made in the adjust step, where the tunes are brought to (\(62.314, 60.319\)) before going into collisions.
This last setting may be changed by machine operators during stable beams in order to optimize for beam lifetime.

Starting in \Gls{run}~\num{3}, some additional complexities were added to the cycle that are not shown in \cref{figure:lhc_cycle}.
In \num{2022} a \concept{\(\beta^{\ast}\)-leveling} was introduced, were collisions start at \(\beta^{\ast} =\)~\qty{60}{\centi\meter} and the \glssymbol{betastar} is progressively reduced to \qty{30}{\centi\meter} during stable beams.
This is done in order to limit pile-up for the experiments (at around \num{52} events per bunch crossing for the main \glspl{IP} in \num{2022}) and the impact on the triplets' cryogenics capacity~\cite{CERN:Fartoukh:LHC_Config_Run3, CERN:Ferlin:Cryogenics}.
This \(\beta^{\ast}\)-leveling is moved to start at \(\beta^{\ast} =\)~\qty{1.2}{\meter} in \num{2023} and \num{2024}, with a higher pile-up value.
Starting in \num{2023} an \concept{anti-telesqueeze} is performed in the ramp to allow this earlier leveling, and a crossing-angle rotation at \acrshort{LHCb} (IP\num{8}) is done when reaching flat-top in order to maintain physics conditions at the IP regardless of the LHCb spectrometer polarity~\cite{CERN:Fartoukh:LHC_Config_Run3}.

\section{Optics Measurements and Corrections}
\label{section:optics_measurements_and_corrections}

The quality of the \gls{LHC} optics has a significant impact on the machine's performance.
For instance, the luminosity achieved by the machine is directly determined by the \glspl{beta-function} at the \glspl{IP}, as seen in \cref{section:luminosity}.
Furthermore, a good control of the \(\beta\)-functions is essential for safe beam operations due to the destructive power of the LHC beams, and the machine is subject to strict limits on the deviation from model values~\cite{CERN:Bruning:Field_Quality_Spec_LHC_Main_Dipoles}.
One can then define the \gls{beta-beating}, a good indicator of the quality of the linear \gls{optics}, as the relative deviation of the machine's \(\beta\)-functions from that of the designed values.
It is defined as:

\begin{equation}
  \frac{\Delta \beta_z(s)}{\beta_z(s)} = \frac{\beta_z(s)_{\mathrm{measured}} - \beta_z(s)_{\mathrm{model}}}{\beta_z(s)_{\mathrm{model}}} \quad \text { where } z = x, y \text{ .}
  \label{equation:beta_beating_definition}
\end{equation}
\vspace{0.3mm}

In order to verify the machine's beam optics and find any potential faults, or deviations from the model values, beam measurements are necessary.
From these, comparisons to model values are made which allow for an assessment and understanding of the errors in the machine; and corrections can be calculated and applied to bring back the optics as close to the nominal scenario as possible.
As the linear optics functions impact the non-linear phenomenology of an accelerator, a well understood and corrected linear optics is a pre-requisite to study of the non-linear dynamics.

\Cref{figure:virgin_vs_corrected_lhcb2} shows the \(\beta\)-beating for beam~\num{2} of the LHC in its \num{2022} virgin\footnote{The term \textit{virgin} refers to the state of the machine without any corrections.} state and with all determined corrections trimmed in, at the end of the commissioning phase.
Correction of the linear optics functions towards their nominal values also leads to an enhanced rms closed orbit around the ring since the orbit feedback algorithms in the LHC assume the nominal LHC model~\cite{PRAB:Tomas:Record_Low_Beta_Beating_in_the_LHC, PRAB:Persson:LHC_Optics_Commissioning_OnePercent}.

\begin{figure}[!hbt]
  \centering
  \includegraphics*[width=\linewidth]{Figures/Optics_Measurements_Corrections_at_LHC/virgin_vs_commissionned_lhcb2.pdf}
  \caption{The measured \(\beta\)-beating at the beginning (\textcolor{mplblue}{blue}) and end (\textcolor{mplorange}{orange}) of the LHC \num{2022} commissioning, for beam~\num{2}.}
  \label{figure:virgin_vs_corrected_lhcb2}
\end{figure}

\subsection{Beam Instrumentation for Optics Measurements}
\label{subsection:beam_instrumentation_for_optics_measurements}

Circular machines such as the LHC include a variety of beam instrumentation devices which serve various purposes, from injection kickers and feedback systems used in regular operations to dedicated devices for optics measurements.
Below is an introduction to those of interest for optics measurements.

\subsubsection*{Beam Position Monitors and Tune Measurement}

\acrfullpl{BPM} are one of the most crucial devices for beam diagnostics.
They measure the transverse center of charge of circulating bunches, either in a given plane for single plane BPMs, or in both planes simultaneously for dual plane BPMs.
In the \gls{LHC}, this centroid beam position can be measured on a turn-by-turn and bunch-by-bunch basis by around \num{500} dual plane \glspl{BPM} across the machine.
So-called stripline BPMs are employed in the common apertures as they can discriminate between counter-rotating bunches, while button BPMs are used in the remaining portions of the machine~\cite{BOOK:Bruning:LHC_Design_Report_Main_Ring}. 
The location of BPMs in the lattice can be seen as vertical grey lines, in an insertion region such as IR\num{5} in \cref{figure:lhc_ir5_zoomed,figure:lhc_matching_section} and in the arcs in \cref{figure:lhc_arc_cell_latwiss}.

In the LHC, the \gls{BBQ} system~\cite{CERN:Boccardi:LHC_Transverse_Diagnostics_Systems,CERN:Boccardi:LHC_BBQ_Tune_Chromaticity_Systems} provides continuous, passive monitoring of the tune by performing spectral analysis of the turn-by-turn data at a specific location in IR\num{4}.
The \gls{BBQ} is also capable of measuring an estimation of the linear coupling at the measurement location, which can be used as a rough first estimate for the coupling in the machine.
While it is possible to assess tune and coupling with the BBQ without external excitation of the beam, the LHC chirp can generate small transverse oscillations to improve the quality of these measurements.

\subsubsection*{Experimental Kickers}

For the study of beam dynamics, measurements are done by inducing large transverse oscillations of the beam to be picked up by the \glspl{BPM}, typically much larger than natural beam size.
Large oscillation amplitudes are required to provide a good signal-to-noise ratio.
The spectral analysis of measured turn-by-turn positions provides valuable insights in all the modes contained in the particle motion, at each BPM location.
In the \gls{LHC} kicker dipole magnets are available for both beams, and are located in IR\num{4}.
These magnets can operate in three possible modes, referred to as the \concept{Tune Kicker} (MKQ), \concept{Aperture Kicker} (MKA), and the \concept{AC dipole}.

The tune and aperture kickers~\cite{CERN:Barlow:Control_MKQA_LHC,IPMS:Carlier:Kicker_Pulse_Generator_Measurement_Tune_Dynamic_Aperture_LHC} operate as traditional kicker magnets: ramping up and down in a single turn, applying a transverse kick to the beam and then allowing free motion.
The name only refers to the amplitude of induced oscillations: lower strength to measure the tune and higher strength to measure the available dynamic aperture.
Unfortunately, at top energy the amplitude of oscillations achievable with the kickers is considerably reduced.
Additionally the beam will decohere after being kicked: the momentum distribution of particles in the bunch will cause the observed centroid of the beam to show a decaying oscillation~\cite{Report:Meller:Decoherence_Kicked_Beams}.
As a consequence a beam can only be kicked a certain number of times before needing to be replaced, and it can take up to several hours to reach the same machine configuration again, as the cycle detailed in \cref{section:lhc_operational_cycle} needs to be respected.

For \gls{optics} measurements lasting oscillations are preffered, as they increase the spectral resolution and reduce noise floor in the spectral analysis of turn-by-turn data.
Furthermore, a non-destructive excitation method is preferred in order not to alter the beam state, which allows for repeated measurements.

Such a non-destructive, sustained excitation of the beams can be achieved with the \gls{AC_dipole}.
It is a fast oscillating dipole magnet which can generate forced driven oscillations with large amplitudes by exciting the beam at frequencies close to the natural tunes~\cite{PHD:Miyamoto,CERN:Serrano:LHC_ACDipole_Introduction}.
Moreover, the AC dipole strength can be ramped up and down adiabatically~\cite{PRAB:Tomas:Adiabaticity_Ramping_Process_AC_Dipole} and kept constant at high amplitudes, allowing for a long lasting coherent oscillations of the beam.
These properties, fullfilling the aforementioned requirements, make the AC dipole the most important tool for optics measurements in the LHC.
A comparison of the turn-by-turn data obtained from beam excitation with a single free kick and an AC dipole is shown in \cref{figure:kick_vs_acdipole_tbt}.

\begin{figure}[!htb]
  \centering
  \includegraphics*[width=\linewidth]{Figures/Optics_Measurements_Corrections_at_LHC/kick_vs_acdipole.pdf}
  \caption{Comparison of turn-by-turn data obtained from a single free kick (top) and an AC dipole excitation (bottom).}
  \label{figure:kick_vs_acdipole_tbt}
\end{figure}

\subsection{Optics Measurements}
\label{subsection:optics_measurements}

\Gls{optics} measurements are performed by exciting the beam and analyzing the resulting betatron oscillations.
The various steps of the measurement and analysis process are described below.

\subsubsection*{Beam Excitation}

The first step of optics measurements in the LHC is to excite the beams with the \gls{AC_dipole}.
As excitation of the beams to large amplitudes can represent a risk to the safety of the machine, such measurements are only performed with one to three low intensity bunches of about \num{e10} protons each.
In the LHC, the AC dipole has a ramp-up and ramp-down time of \num{2000} turns, and is able to drive excitations at maximum strength for \num{6600} turns before ramping down, as can be seen on \cref{figure:kick_vs_acdipole_tbt}.
When exciting the beam, the turn-by-turn position of the bunch is measured by the \glspl{BPM} across the machine and only the turns corresponding to the maximum AC dipole strength are used for analysis.

It is important to note now that the forced oscillations introduce a perturbation on the optics.
In the case of free oscillations, it was shown in \cref{chapter:theory} that the transverse motion goes according to \cref{equation:hill_solution}.
For the transverse plane \(z\), by considering \(\phi_{z,0} = 0\) at the start of machine one gets:

\begin{equation}
  z(s) = \sqrt{2 J_z \beta_z(s)} \cos \left( \phi_z \right)  \text{ ,} \quad z = x, y \text{ ,}
  \label{equation:free_transverse_motion}
\end{equation}
where \(\phi_z\) and \(J_z\) are the action and angle variables introduced in \cref{subsection:phase_space_ellipse}, respectively.

When the AC dipole is driving the beam, an equivalent parametrization exists, and noting \(z_D(s)\) the transverse driven coordinate one can express it as~\cite{PHD:Miyamoto,PRAB:Tomas:Normal_Form_Particle_Motion_AC_Dipole}:

\begin{equation}
  z_D(s)=\sqrt{2 J_z \beta_z(s)} \cos \left( \phi_z \right) + \sqrt{2 A \beta_{D,z}(s)} \cos \left( \phi_D \right) \text{ ,}
  \label{equation:driven_transverse_motion}
\end{equation}
where \(A\) and \(\phi_D\) are respectively the action and angle variables of the forced oscillations, and \(\beta_{D,z}(s)\) is the \(\beta\)-function modified by the AC dipole.
The form of \cref{equation:driven_transverse_motion} makes the assumption that the forced oscillations term only depends on \(\phi_D\).

The \(\beta\)-function modified by the AC dipole, corresponding to the perturbed optics under forced oscillations, is given as~\cite{PRAB:Miyamoto:Parametrization_Driven_Betatron_Oscillation}:

\begin{equation}
  \beta_{D,z}(s) = \frac{1 + \lambda_D^2 - 2 \lambda_D \cos \left(\psi_D(s)\right)}{1 - \lambda_D^2} \beta_z(s) \text{ ,}
  \label{equation:driven_beta_function}
\end{equation}
where \(\psi_D(s)\) is the phase advance with respect to the AC dipole location.
The \(\lambda_D\) term is dependent on the gap between the driven and natural tune, and is defined by~\cite{PRAB:Miyamoto:Parametrization_Driven_Betatron_Oscillation}:

\begin{equation}
  \lambda_D = \frac{\sin \left(\pi \delta_D \right)}{\sin \left(2 \pi Q_z + \pi \delta_D \right)} \text{ ,}
  \label{equation:driven_oscillations_lambda_D}
\end{equation}
in which \(\delta_D = Q_D - Q_z\) corresponds to the aforementioned gap, often called tune delta.
In the LHC the AC dipole is usually driven with \(\delta_{D,x} = -0.01\) and \(\delta_{D,y} = 0.012\).

One can see in \cref{figure:acdipole_beta_functions_vs_nominal} the impact of an AC dipole on the vertical \(\beta\)-function in a simple FODO lattice, where the difference between the free and forced cases is apparent.

\begin{figure}[!htb]
  \centering
  \includegraphics*[width=0.99\linewidth]{Figures/Optics_Measurements_Corrections_at_LHC/betas_nominal_vs_driven.pdf}
  \caption{Resulting vertical \(\beta\)-functions in a FODO lattice in the case of free (\textcolor{mplorange}{orange}) and driven (\textcolor{mplblue}{blue}) oscillations.}
  \label{figure:acdipole_beta_functions_vs_nominal}
\end{figure}

One can then determine the \(\beta\)-beating induced by the AC dipole, as defined in \cref{equation:beta_beating_definition}.
From \cref{equation:driven_beta_function} and \cref{equation:driven_oscillations_lambda_D} one can express this beating according to:

\begin{equation}
  \frac{\beta_{D,z} - \beta_z}{\beta_z} = \frac{1 + \lambda_D^2 - \lambda_D \cos \left(2 \psi_D - 2 \pi Q_z\right)}{1 - \lambda_D^2} - 1 \text{ ,}
  \label{equation:ac_dipole_beta_beating}
\end{equation}
with \(Q_z\) the natural transverse tune of the machine.
\Cref{figure:ac_dipole_induced_beta_beating} shows this \(\beta\)-beating for various phase advances between the AC dipole and an element where the observation would be made.
The vertical black lines correspond to the usual tune deltas used in LHC measurements, and the values are displayed for the two fractional tunes of a common working point for LHC measurements: \((Q_x, Q_y) = (0.28, 0.31)\).
For these usual settings the AC dipole induced \(\beta\)-beating reaches no more than \qty{9}{\percent}.

\begin{figure}[!htb]
  \centering
  \includegraphics*[width=0.99\linewidth]{Figures/Optics_Measurements_Corrections_at_LHC/bbeatings_acdipole.pdf}
  \caption{AC dipole induced \(\beta\)-beating as a function of \(\delta_D\), for various phase advances between the AC dipole and a given location in the ring where the observation would be made. The results are shown for the horizontal (left) and vertical (right) planes with a common working point used for measurements.}
  \label{figure:ac_dipole_induced_beta_beating}
\end{figure}

As a consequence, the measured oscillations, associated optics functions and \glspl{RDT} will not be that of the natural machine.
This impact of the AC dipole on the optics is taken into account and compensated in analysis steps~\cite{IPAC:Miyamoto:Measurement_Coupling_RDTs_LHC_AC_Dipole}, and this compensation is expanded on later on in this chapter.

\subsubsection*{Spectral Analysis}

As a first step the recorded raw turn-by-turn data is cleaned of noise using a \gls{SVD} approach~\cite{PRAB:Calaga:Statistical_Analysis_RHIC_BPMs}.
An interpolated Fourier Transform is then performed on the cleaned data~\cite{PHD:Malina,IPAC:Malina:Harpy_Fast_Simple}, which provides information about the phase and the measured amplitude at each BPM.
Automatic outlier detection is available based on the spectra of BPMs, both as an automated step~\cite{PRAB:Fol:Detection_Faulty_BPMs} and a manual step (removing BPMs with exact-zero signals, wrong tune lines etc).
\Cref{figure:example_spectrum} shows a horizontal spectrum obtained from a measurement during the Run~\num{2} commissioning in \num{2022}, with the main tune lines identified.
The frequency space is zoomed for clarity.

\begin{figure}[!htb]
  \centering
  \includegraphics*[width=\linewidth]{Figures/Optics_Measurements_Corrections_at_LHC/example_bpm_spectrum.pdf}
  \caption{Horizontal spectrum from the frequency analysis of a measurement taken during the Run~\num{2} commissioning in \num{2022}. The main (\textcolor{mplorange}{orange}) and natural (\textcolor{mplgreen}{green}) tune lines are indicated, where for instance \((0, 1)\) corresponds to the \(0 \cdot Q_x + 1 \cdot Q_y = Q_y\) tune in the horizontal spectrum.}
  \label{figure:example_spectrum}
\end{figure}

\subsubsection*{Optics Reconstruction}

With knowledge of the phase information at each measuring \gls{BPM}, and with knowledge of the model machine, the transverse optics functions can be determined.
The model knowledge traditionally comes from the \gls{MADX} code~\cite{CODE:MADX_guide} for LHC measurements.
The \(\beta\)-function at a given BPM can be calculated from the measured phases of \num{3} BPMs \((i, j, k)\), according to~\cite{PHD:Castro,BOOK:Minty:Measurements_Control_Charged_Particle_Beams}:

\begin{equation}
  \beta_z^{\mathrm{phase}}(s_i) = \frac{\cot \left(\phi_z(i \rightarrow j)\right) + \cot \left(\phi_z(i \rightarrow k)\right)}{\cot \left(\phi^m_z(i \rightarrow j)\right) + \cot \left(\phi^m_z(i \rightarrow k)\right)} \beta^m_z(s_i)  \text{ ,} \quad z = x, y \text{ ,}
  \label{equation:beta_from_phase}
\end{equation}
\vspace{0.3mm}

\noindent
where the superscript \(^m\) denotes the model values and \(\phi_z(i \rightarrow j)\) is the phase advance between the \(i^{\mathrm{th}}\) and \(j^{\mathrm{th}}\) BPMs.
The BPMs do not necessarily need to be consecutive, and specific phase advances between the chosen BPMs increase the precision of the measurement.
This method has also been extended to use specifically chosen combinations of \(N\) BPMs~\cite{PRAB:Langner:N_BPM_Method,PRAB:Wegscheider:Analytical_N_BPM_Method}, which allows to improve its precision as it then only depends on the measured phase advances and is fully independent of BPM calibration~\cite{PRAB:Langner:Optics_Measurement_Algorithms_Error_Analysis_Proton_Energy_Frontier}.

From \cref{equation:single_particle_action} one infers that the \(\beta\)-function can also be determined directly from the amplitude \(A\) of the oscillations recorded at BPMs:

\begin{equation}
  \beta_z^{\mathrm{amp}}(s_i) = \frac{A_z^2(s_i)}{2 J_z} \text{ .}
  \label{equation:beta_from_amplitude}
\end{equation}

Using the peak-to-peak oscillation amplitude over \(n\) BPMs, one can determine the calibration-dependent~\cite{PRAB:GarciaTabares:BPM_Calibration} term \(J_z\) according to:

\begin{equation}
  2 J_z = \frac{\sum_n \left(\frac{\mathrm{peak-to-peak}}{2}\right)^2 / \beta^m_z}{n} \text{ .}
  \label{equation:J_from_peak_to_peak}
\end{equation}

The reconstructed \gls{beta-function} from amplitude is dependent on the calibration of measuring BPMs.
\Cref{figure:betabeating_phase_vs_amp} shows the \(\beta\)-beating reconstructed from phase and from amplitude for a \num{2022} LHC measurement at \(\beta^{\ast} =\) \qty{30}{\centi\meter}, where the lower precision from reconstructing from amplitude can be seen.

\begin{figure}[!htb]
  \centering
  \includegraphics*[width=\linewidth]{Figures/Optics_Measurements_Corrections_at_LHC/betabeat_phase_vs_amp.pdf}
  \caption{The \(\beta\)-beatings obtained when reconstructing the \(\beta\)-functions from phase (\textcolor{mplblue}{blue}) and from amplitude (\textcolor{mplorange}{orange}) in an LHC measurement in \num{2022}.}
  \label{figure:betabeating_phase_vs_amp}
\end{figure}

Other Twiss parameters can be reconstructed from the phases and \(\beta\)-functions.
Chromatic parameters are reconstructed by performing measurements at different momentum settings.
By adjusting the frequency of the accelerating cavities, typically between \qty{-100}{\hertz} and \qty{+100}{\hertz} of the nominal setting, the energy of the beam is changed and the dispersion is then determined from the mean orbit recorded at different energies.
Recent improvements have allowed doing so in a single measurement by modulating the RF frequency while transversely exciting the beam~\cite{PHD:Malina, IPAC:Malina:3D_Excitation}.

One can also compute the normalized dispersion \(D / \sqrt{\beta}\) which is independent from model values and BPM calibration~\cite{PAC:Calaga:BPM_Calibration_Independent_LHC_Optics_Correction}.

\subsubsection*{K-Modulation}

It is possible to directly measure the \glssymbol{betastar} at the \glspl{IP} without beam excitation using the \concept{\(k\)-modulation} technique~\cite{IPAC:Calaga:LHC_K_Modulation}.
By modifying the strength of individual quadrupoles and recording the resulting tune changes, one can determine the average \(\beta\)-functions at the modulated quadrupole according to~\cite{BOOK:Minty:Measurements_Control_Charged_Particle_Beams}:

\begin{equation}
  \begin{aligned}
    \beta^{\mathrm{avg}}_{z} & = \pm \left[ \cot \left(2 \pi Q_z \right) \left( 1 - \cos \left(2 \pi \Delta Q_z \right) \right) + \sin \left( 2 \pi \Delta Q_z \right) \right] \frac{2}{\Delta k L} \text{ ,}  \\
                               & \approx \pm 4 \pi \frac{\Delta Q_z}{\Delta k L} \text{ ,}
  \end{aligned}
  \label{equation:k_modulation_average_beta}
\end{equation}
where \(\Delta k\) is the quadrupole strength variation, \(L\) its length and \(\Delta Q_z\) the resulting tune variation in the machine.
\Cref{figure:kmod_mqxa1l1} shows an example of the data retrieved from the \(k\)-modulation.

\begin{figure}[!htb]
  \centering
  \includegraphics*[width=\linewidth]{Figures/Optics_Measurements_Corrections_at_LHC/k_modulation.pdf}
  \caption{Example data from the \(k\)-modulation of the first quadrupole left of IP\num{1}. The top plot shows the quadrupole strength variation over time and the bottom plot the resulting horizontal (\textcolor{mplblue}{blue}) and vertical (\textcolor{mplorange}{orange}) tune variations.}
  \label{figure:kmod_mqxa1l1}
\end{figure}

By modulating both innermost quadrupoles near the IP, one determines the average \(\beta\)-functions in these quadrupoles, which can then be propagated through the drift space in between to determine both the \(\beta^{\ast}\) value as well as any potential waist shift~\cite{PRAB:Carlier:K_Modulation_HL_LHC}.
In the LHC the \(k\)-modulation technique is the main method to determine the \(\beta^{\ast}\) at the \glspl{IP}.
Its results are also used as a better reconstruction of the \(\beta\)-function values in the inner BPMs.

\subsection{Reconstruction of Linear Coupling RDTs}
\label{subsection:reconstruction_linear_coupling_rdts}

Of particular relevance to this thesis are the \gls{betatron_coupling} \glspl{RDT} \(f_{1001}\) and \(f_{1010}\), and it is worth spending some time detailing their reconstruction.
As mentioned in \cref{subsec:spectral_contribution}, the \(f_{jklm}\) RDTs can be determined from the specific \concept{spectral lines} arising during the spectral analysis of turn-by-turn data.
By the term spectral lines one refers here to the Fourier transform of the complex Courant-Snyder coordinate, which corresponds to:

\begin{equation}
  \begin{aligned}
    H^\pm(n_x, n_y) &= \mathcal{F}\{h_x^\pm\}(n_x Q_x + n_y Q_y)  \text{ ,}  \\
    V^\pm(n_x, n_y) &= \mathcal{F}\{h_y^\pm\}(n_x Q_x + n_y Q_y)  \text{ ,}
  \end{aligned}
  \label{equation:spectral_lines}
\end{equation}
\vspace{0.1mm}

\noindent
in which \(H\) indicates a line in the spectrum of horizontal turn-by-turn data, and \(V\) a line in the spectrum of vertical turn-by-turn data.

As seen in \cref{equation:spectral_lines} the Courant-Snyder variables defined in \cref{equation:resonance_basis_definition} are needed.
Since the momentum present in the expression of \(h_z^\pm\) is not a quantity directly measurable from a single BPM, it is reconstructed using two successive BPMs:

\begin{equation}
  \hat{p}_{z_n} = \frac{\hat{z}_{n+1} - \hat{z}_n \cos \left(\Delta \phi_z\right)}{\sin \left(\Delta \phi_z\right)}  \text{ ,} \quad z = x, y \text{ ,}
  \label{equation:momentum_from_two_bpms}
\end{equation}
\vspace{1pt}

\noindent
with \(\Delta \phi_z\) the phase advance between the \(n^{\mathrm{th}}\) and \((n+1)^{\mathrm{th}}\) BPMs in the transverse \(z\) plane.
In \cref{equation:momentum_from_two_bpms} it is assumed that the region between the two BPMs is free of coupling sources as well as non-linearities contributing to the main line and the coupling line in each spectrum.

With the information from two BPMs, one can compute the spectral lines according to \cref{equation:spectral_lines}.
Starting from \cref{equation:rdt_contribution_to_spectrum_line}, one can for instance subsitute \(j = 1\), \(k = 0\), \(l = 0\), \(m = 1\) and obtain \(H(0, 1) = \abs{f_{1001}} \left(2 I_y \right)^{\frac{1}{2}}\), the amplitude of the line at \(Q_y\) in the horizontal spectrum.

With the proper values for \(j,k,l,m\) and using lines from the vertical spectrum one can cancel out the dependence on the action and look to obtain \(\abs{f_{1001}}\).
However, it is possible - and likely - that the BPM calibrations aren't perfect, in which cases the measured and real coordinates differ:

\begin{equation}
  \begin{aligned}
    x^{\mathrm{meas}} &= C_x x^{\mathrm{real}}  \text{ ,}  \\
    y^{\mathrm{meas}} &= C_y y^{\mathrm{real}}  \text{ .}
  \end{aligned}
\end{equation}
\vspace{1pt}

One can cancel out the \(C_{x,y}\) factors by dividing any line amplitude by that of the main line, to create normalized spectral lines.
Of interest for the coupling \glspl{RDT} are the following:

\begin{equation}
  \begin{aligned}
    A_{0,1} &= \frac{H(0,1)}{\abs{H(1,0)}}    \text{ ,}  \\
    B_{1,0} &= \frac{V(1,0)}{\abs{V(0,1)}}    \text{ ,}  \\
    A_{0,-1} &= \frac{H(0,-1)}{\abs{H(1,0)}}  \text{ ,}  \\
    B_{-1,0} &= \frac{V(-1,0)}{\abs{V(0,1)}}  \text{ ,}
  \end{aligned}
  \label{equation:normalized_spectral_lines}
\end{equation}
\vspace{1pt}

\noindent
where the \(A_{0,1}\) and \(B_{1,0}\) normalized spectral lines contain information on \(f_{1001}\) whereas the \(A_{0,-1}\) and \(B_{-1,0}\) ones contain \(f_{1010}\).

By combining these normalized spectral lines, it has been shown that the amplitudes of the coupling \glspl{RDT} may be expressed as~\cite{CERN:Franchi:Computation_Coupling_Resonance_Driving_Term_Single_BPM,PRAB:Tomas:CERN_LHC_OMC}:

\begin{equation}
  \begin{gathered}
    \abs{f_{1001}} = \frac{1}{2} \sqrt{\abs{A_{0,1} B_{1,0}}}   = \frac{1}{2} \sqrt{\frac{H(0,1) V(1,0)}{V(0,1) H(1,0)}}       \text{ ,}  \\
    \abs{f_{1010}} = \frac{1}{2} \sqrt{\abs{A_{0,-1} B_{-1,0}}} = \frac{1}{2} \sqrt{\frac{H(0,-1) V(0,-1)}{V(0,1), H(1,0)}}  \text{ ,}
  \end{gathered}
  \label{equation:coupling_rdts_amplitude_from_spectral_lines}
\end{equation}
\vspace{1pt}

\noindent
and the phases as:

\begin{equation}
  \begin{gathered}
    q_{1001} = \phi_{V(1,0)} -\phi_{H(1,0)} +\frac{\pi}{2} = \phi_{H(0,1)} - \phi_{V(0,1)} + \frac{\pi}{2}      \text{ ,}  \\
    q_{1010} = \phi_{H(0,-1)} -\phi_{V(0,1)} +\frac{\pi}{2} = \phi_{V(-1,0)} - \phi_{H(1,0)} + \frac{\pi}{2}    \text{ .}
  \end{gathered}
  \label{equation:coupling_rdts_phase_from_spectral_lines}
\end{equation}
\vspace{1pt}

Here \(H(1,0)\) corresponds to the line in the horizontal spectrum at \(Q_x\) while \(H(0, 1)\) corresponds to the line at \(Q_y\) in the same spectrum.
In \cite{PRAB:Franchi:First_Simultaneous} a table is given that relates various spectral lines to amplitudes and phases of the corresponding resonances and RDTs.
From the amplitude and phase information of \cref{equation:coupling_rdts_amplitude_from_spectral_lines,equation:coupling_rdts_phase_from_spectral_lines} the complex coupling RDTs are reconstructed as:

\begin{equation}
  \begin{aligned}
    f_{1001} &= \abs{f_{1001}} e^{i q_{1001}}  \text{ ,}  \\
    f_{1010} &= \abs{f_{1010}} e^{i q_{1010}}  \text{ .}  \\
  \end{aligned}
  \label{equation:complex_coupling_rdts_from_phase_and_amplitude}
\end{equation}
\vspace{1pt}

In \cref{figure:example_spectrum}, which shows a spectrum from an LHC measurement during the \num{2022} commissioning, the coupling lines are visible and highlighted.
This reconstruction method has successfully been used to measure linear coupling RDTs in the LHC since Run~\num{1}~\cite{PRAB:Benedikt:Driving_Term_Experiments_CERN,IPAC:Persson:Automatic_Coupling_Correction_LHC_Injection_Oscillations,IPAC:Miyamoto:Measurement_Coupling_RDTs_LHC_AC_Dipole}.

\subsubsection*{AC dipole Compensation}

Mentioned in \cref{subsection:optics_measurements} is the effect of \gls{AC_dipole} excitation on the observables, which differ from that of the natural machine.
This effect is also present in the reconstructed coupling \glspl{RDT}, and must be compensated for in order to obtain the natural machine's values.
Several compensation methods exist, with the most straightforward being the application of a rescaling factor to the reconstructed \glspl{RDT}.

Using this rescaling one can express the driven \glspl{RDT} as~\cite{IPAC:Wegscheider:Forced_Coupling_Resonance_Driving_Terms}:

\begin{equation}
  \begin{aligned}
    f_{\pm, x}^{\mathrm{driven}} &= \frac{\sin \left( Q_x \pm Q_y \right)}{\sin \left( Q_{D,x} \pm Q_y \right)} f_{\pm} \text{ ,} \\
    f_{\pm, y}^{\mathrm{driven}} &= \frac{\sin \left( Q_x \pm Q_y \right)}{\sin \left( Q_x \pm Q_{D,y} \right)} f_{\pm} \text{ ,}
  \end{aligned}
  \label{equation:rescaling_coupling_rdts}
\end{equation}
\vspace{1pt}

\noindent
where \(f_{+} = f_{1010}\) corresponds to the sum resonance, \(f_{-} = f_{1001}\) corresponds to the difference resonance, \(f_{\pm, x}^{\mathrm{driven}}\) denotes the driven \gls{RDT} as measured from horizontal turn-by-turn data, \(Q_{D, z}\) is the driven tune in the transverse plane \(z\) and \(Q_z\) the natural tune in the same plane.
Here the \(f_{\pm}\) notation is used for concisiveness.

A fully analytical compensation of the AC dipole effect can be found in~\cite{REPORT:Miyamoto:Measurement_Coupling_Resonance_Driving_Terms,IPAC:Miyamoto:Measurement_Coupling_RDTs_LHC_AC_Dipole} but will not be detailed in this document.
A comparison of these methods' efficiency can be found in~\cite{IPAC:Wegscheider:NBPM_Momentum_Reconstruction_for_Linear_Coupling_RDTs}.

\subsubsection*{Single BPM}

It is possible to approximate the coupling \glspl{RDT} from the real coordinates using a single \gls{BPM}, and doing the reconstruction without the momentum information.
In such case where only the position data is used, however, the spectal analysis will mix up lines \(Z(a,b)\) and \(Z(-a, -b)\) as it cannot distinguish one from the other.
For the linear coupling RDTs this means that the \(f_{1001}\) and \(f_{1010}\) will contribute to the same resonance lines and cannot be separated~\cite{PHD:Persson}.
It may be, however, possible to neglect one of them depending on their relative strengths.

Additionally, from the method described above it becomes clear that dual plane BPMs are required for coupling measurements.
In case such instrumentation is not available, one can numerically construct a pseudo dual plane BPM by virtually shifting a vertical plane monitor towards the location of the nearest horizontal one, or inversely.
To do so, the phases of the real spectral lines are shifted in accordance with the phase advance between the monitors, which assumes that the segment between these monitors is free of non-linear sources~\cite{PHD:Vanbavinckhove}.

\subsection{Correction Principles}
\label{subsection:correction_principles}

Corrections of the linear optics in the LHC are based on two different approaches.
Global corrections are better suited to compensate for widely distributed sources, while local corrections are focused towards the identification and compensation of strong, highly localized errors, and are mostly used around the IPs.

\subsubsection*{Local Corrections}

In the LHC local optics errors are determined and corrected using the \concept{Segment-by-Segment} (SbS) technique~\cite{PRAB:Tomas:CERN_LHC_OMC,PRAB:Tomas:Review_Linear_Optics_Measurements}.
The technique treats a section - or segment - of the accelerator as an independent beam line and propagates optics parameters measured at the start of the segment through the line using the MAD-X code~\cite{CODE:MADX_guide}.
The propagated optics parameters may be compared with the observation and one then tries to find correction settings - powering changes of selected magnets - that would best reproduce these propagated optics.
Thereby, inverting the settings found and applying the inverted values in the machine corrects the measured errors.

This method is mostly used in the LHC IRs, where the \(\beta\)-beating is corrected by compensating the discrepancies in the betatron phase, which has the same impact as correcting the \(\beta\)-beating directly but proved to be a more precise and local observable~\cite{PRAB:Tomas:CERN_LHC_OMC}.
For this one looks at the \(\Delta \phi\) quantity thought the segment and tries to minimze it:

\begin{equation}
  \Delta \phi = \phi_{\mathrm{model}} - \phi_{\mathrm{measurement}} \text{ .}
  \label{equation:sbs_delta_phi}
\end{equation}

An example of a local correction of the phase advance around IP\num{5} from the LHC \num{2022} commissioning is shown in \cref{figure:example_sbs_correction}, where the phase deviation from the model values are shown across the segment together with the effect of the reconstructed errors.

\begin{figure}[!hbt]
  \centering
  \includegraphics*[width=0.99\linewidth]{Figures/Optics_Measurements_Corrections_at_LHC/sbs_phase_ip5_example.pdf}
  \caption{Local phase correction in IR\num{5} (vertical line indicates IP\num{5}) from the LHC \num{2022} commissioning at flat-top. The \textcolor{mplblue}{blue} line shows the measured phase deviation from model values, while the \textcolor{mplorange}{orange} line shows the effect of the reconstructed errors on the model. The \textcolor{mplgreen}{green} line shows the expected phase deviation after applying the correction.}
  \label{figure:example_sbs_correction}
\end{figure}

The concept from \cref{equation:sbs_delta_phi} can be applied to other observables such as the \(\beta\)-functions or the coupling \acrshortpl{RDT}.
In the case of the coupling \glspl{RDT}, due to the lower number of magnets available for adjustment one usually tries to compensate for the \glspl{RDT} at the edges of the segment.
One can also attempt to match for specific components of the \glspl{RDT} specifically (real and imaginary parts).
\Cref{figure:example_sbs_correction_coupling} shows an example of a local correction of the coupling \glspl{RDT} around IP\num{2} from the LHC \num{2021} beam test.
In this case, a good rematching at the edges of the segment means the contribution of the \gls{IR} to the rest of the machine is well compensated.]
More details are given in the next chapter.

\begin{figure}[!hbt]
  \centering
  \includegraphics*[width=0.99\linewidth]{Figures/Optics_Measurements_Corrections_at_LHC/sbs_coupling_ip2_example.pdf}
  \caption{Local coupling RDTs correction in IR\num{2} (vertical line indicates IP\num{2}) from the LHC \num{2021} beam test. The \textcolor{mplblue}{blue} line shows the measured RDTs, while the \textcolor{mplorange}{orange} line shows the attempt at canceling the contribution at the edges of the segment with the two available correctors in the IR.}
  \label{figure:example_sbs_correction_coupling}
\end{figure}

This method has successfully been applied in the LHC for many years~\cite{PRAB:Aiba:First_Beating_Measurement_LHC,PRAB:Tomas:Record_Low_Beta_Beating_in_the_LHC,PRAB:Persson:LHC_Optics_Commissioning_OnePercent}.

\subsubsection*{Global Corrections}

Global corrections are based on a \concept{response matrix} approach~\cite{PHD:Vanbavinckhove,EPAC:Tomas:Procedures_Accuracy_Estimates_Beta_Beat_Correction_LHC}.
This matrix, constructed from the machine model and simulation codes, holds the information on the response of the model optics functions to changes made in model settings, usually magnet powering changes.
For instance, the response to a quadrupole \gls{knob} \gls{trim} of optics functions such as the phase advances, \(\beta\)-functions, normalized dispersion and tunes can be expressed as:

\begin{equation}
  \mathbf{R} \Delta \vec{K} = \left(\Delta \overrightarrow{\phi_x}, \Delta \overrightarrow{\phi_y}, \frac{\Delta \beta_x}{\beta_x}, \frac{\Delta \beta_y}{\beta_y}, \Delta \frac{\overrightarrow{D_x}}{\sqrt{\beta_x}}, \Delta Q_x, \Delta Q_y \right) \text{ .}
  \label{equation:response_matrix}
\end{equation}

Here \(\mathbf{R}\) is a \(M \times N\) matrix, where \(N\) is the number of adjusted quadrupole knobs and \(M\) is the number of observation points, usually the number of BPMs.
To calculate corrections the pseudo-inverse of the \(\mathbf{R}\) matrix, noted here \(\mathbf{R^{-1}}\), is calculated and multiplied with the measured deviations from the model.

\begin{equation}
  \Delta \vec{K} = \mathbf{R^{-1}} \left(w_1 \Delta \overrightarrow{\phi_x}, w_2 \Delta \overrightarrow{\phi_y}, w_3 \frac{\Delta \beta_x}{\beta_x}, w_4 \frac{\Delta \beta_y}{\beta_y}, w_5 \Delta \frac{\overrightarrow{D_x}}{\sqrt{\beta_x}}, w_6 \Delta Q_x, w_7 \Delta Q_y \right) \text{ .}
  \label{equation:global_corrections_from_pseudo_inverse_response_matrix}
\end{equation}

The \(w_1 \ldots w_7\) terms are weights which can be adjusted to either focus on the correcting a given quantity, ignore some parameters completely or to balance the corrections between all properties.
By choosing weights and plugging measured optics deviations into \cref{equation:global_corrections_from_pseudo_inverse_response_matrix}, one can determine the knob trims that could correct said observed deviations.

\glsresetall                                     % reset glossary entries counts for the next chapter
