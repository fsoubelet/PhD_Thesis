\chapter{Optics Measurements and Corrections at the LHC}
\label{chapter:lhc_omc} % For referencing the chapter elsewhere, use \cref{chapter:lhc_omc}

The Large Hadron Collider (LHC) is a \qty{26.659}{\kilo\metre} synchrotron collider located at the European Center for Nuclear Research (CERN), on the French-Swiss border.
It is part of CERN's Accelerator Complex, illustrated in \cref{figure:cern_accelerator_complex}, a chain of particle accelerators progressively bringing protons and heavy ions up to an energy of \qty{6.8}{\tera\electronvolt} per beam, as of \num{2023}.

\begin{figure}[!htb]
  \centering
  \includegraphics*[width=0.9\linewidth]{Figures/Optics_Measurements_Corrections_at_LHC/cern_accelerator_complex.png}
  \caption{The CERN Accelerator Complex in \num{2022}, not to scale~\cite{Website:CERN_Accelerator_Complex_Resource}. For typical LHC operation, a proton beam is produced in \(\mathrm{LINAC}\)~\num{4} and follows the chain: \(\mathrm{LINAC}\)\num{4} \(\rightarrow\) \(\mathrm{PSB}\) \(\rightarrow\) \(\mathrm{PS}\) \(\rightarrow\) \(\mathrm{SPS}\) \(\rightarrow\) \(\mathrm{LHC}\).}
  \label{figure:cern_accelerator_complex}
\end{figure}

Accelerated particles go through a chain of different particle accelerators before reaching their experimental destinations.
For protons colliding in the LHC, the first step is a linear accelerator, LINAC\num{4}, which accelerates them up to a kinetic energy of \qty{160}{\mega\electronvolt}.
Next, the protons are injected into the Proton Synchrotron Booster (PSB), where they are accelerated to an energy of \qty{1.4}{\giga\electronvolt}.
The next stage is the Proton Synchrotron, in which they will reach \qty{25}{\giga\electronvolt}; then the Super Proton Synchrotron (SPS) where they are accelerated to \qty{450}{\giga\electronvolt}, when they are finally injected into the LHC.

The LHC circulates two counter-rotating hadron beams, each in their ring, which are made to collide at four Interaction Points (IPs) to provide data for High Energy Physics (HEP) experiments.
The main data-taking experiments on the LHC are ATLAS~\cite{Website:ATLAS,Website:ATLAS_CDS}, LHCf~\cite{Website:LHCf,Website:LHCf_CDS}, ALICE~\cite{Website:ALICE,Website:ALICE_CDS}, CMS~\cite{Website:CMS,Website:CMS_CDS}, TOTEM~\cite{Website:TOTEM,Website:TOTEM_CDS}, LHCb~\cite{Website:LHCb,Website:LHCb_CDS} and MoEDAL~\cite{Website:MOEDAL,Website:MOEDAL_CDS}.
The LHC is currently the world's highest-energy hadron colliding machine, colliding beams at \qty{13.6}{\tera\electronvolt} center-of-mass energy as of Run~\num{3}, \num{2023}.

%----------------------------------------------------------------------------------------

\section{The LHC Lattice}
\label{section:lhc_lattice}

The LHC lattice consists of eight \intro{octants} each intersected by an \intro{Insertion Region} (IR).
Conventionally, the segment between two \IRs is called an \intro{arc} and the arc between IR1 and IR2 is named Arc12, and similarly for other arcs.
An octant is defined as going from mid-arc to mid-arc around a given \IR which is located at its center.
Each octant is named according to the \IR at its center: the octant with \(\mathrm{IR1}\) at its center is named Octant1, and similarly for other octants.
An illustration and a detailed description on naming conventions can be found in \cref{appendix:naming_conventions}.
% The \IRs host either an \intro{Interaction Point} (IP) where beams are made to collide (ATLAS at IP1, ALICE at IP2, CMS at IP5 and LHCb at IP8) or important instrumentation (momentum cleaning at IR3, Radio-Frequency cavities at IR4, beam dump system at IR6 and betatron cleaning at IR7).

Beam~1 rotates clockwise in its ring when viewing the LHC from above, and Beam~2 rotates counter-clockwise as viewed from above.
The beams occupy separate apertures, or beam pipes, except in the \IRs where they are eventually made to collide.
The layout of the LHC can be seen in \cref{figure:lhc_schematic_layout}, and full details can be found in the LHC Design Report~\cite{BOOK:Bruning:LHC_Design_Report_Main_Ring,BOOK:Bruning:LHC_Design_Report_Infrastructure,BOOK:Benedikt:LHC_Design_Report_Injector_Chain}.

\begin{figure}[!htb]
  \centering
  \includegraphics*[width=0.99\linewidth]{Figures/Optics_Measurements_Corrections_at_LHC/lhc_schematic.pdf}
  \caption{Schematic of the LHC layout, adapted from~\cite{PHD:Poyet}.}
  \label{figure:lhc_schematic_layout}
\end{figure}

\subsection{The LHC Arcs}
\label{subsection:lhc_arcs}

\begin{figure}[!hbt]
  \centering
  \includegraphics*[width=0.9\linewidth]{Figures/Optics_Measurements_Corrections_at_LHC/lhc_fodo_cell.pdf}
  \caption{\todo{Caption lhc fodo cell.}}
  \label{figure:lhc_fodo_cell}
\end{figure}

\begin{figure}[!hbt]
  \centering
  \includegraphics*[width=0.9\linewidth]{Figures/Optics_Measurements_Corrections_at_LHC/lhc_arc23.pdf}
  \caption{\todo{Caption lhc arc23.}}
  \label{figure:lhc_arc23}
\end{figure}

\subsection{The LHC Experimental Interaction Regions (EIR)}
\label{subsection:lhc_eirs}

To achieve a very small value of \betastar at the IPs, as is shown to be desirable in \cref{section:luminosity}, the periodic lattice of the accelerator is interrupted and insertion regions are introduced.

An insertion region in which beams are made to collide is called an \intro{Interaction Region}, or sometimes experimental interaction region.

\todo{Mention for figure that the matching section marked in the plot includes the dispersion suppressor.}

\begin{figure}[!hbt]
  \centering
  \includegraphics*[width=0.9\linewidth]{Figures/Optics_Measurements_Corrections_at_LHC/ir5_surroundings_optics_2.pdf}
  \caption{The horizontal and vertical \betafunctions in the LHC around IP5 at injection optics (top) and collision optics (bottom). Notice the drastically different scales on the vertical axes.}
  \label{figure:ir5_and_around}
\end{figure}

\subsection{Interaction Points}
\label{subsection:interaction_points}

\subsection{The LHC Triplet}
\label{subsection:lhc_triplet}

\subsection{Separation Dipoles}
\label{subsection:separation_dipoles}

\subsection{Matching Section}
\label{subsection:matching_section}

\subsection{Dispersion Suppressor}
\label{subsection:dispersion_suppressor}

\begin{figure}[!hbt]
  \centering
  \includegraphics*[width=0.9\linewidth]{Figures/Optics_Measurements_Corrections_at_LHC/lhc_dispersion_suppressor.pdf}
  \caption{\todo{Caption lhc dispersion suppressor.}}
  \label{figure:lhc_dispersion_suppressor}
\end{figure}

%----------------------------------------------------------------------------------------

\section{The Operational Cycle of the LHC}
\label{section:lhc_operational_cycle}

The LHC operational cycle~\cite{Report:LHCModes}, shown in \cref{figure:lhc_cycle}, starts with a pre-cycle of certain magnetic elements~\cite{Report:LHCMagnetsPreCycles}.
During pre-cycle no beams are present in the rings and the respective element currents are increased up to several \qty{}{\tera\electronvolt}, to ensure the reproducibility of the magnetic fields over successive fills.
After the pre-cycle, beams are injected from the Super Proton Synchrotron (SPS) at an energy of \qty{450}{\giga\electronvolt}.
At injection optics the \(\beta^{\ast}_{x,y}\) is \qty{11}{\metre}.
The number of bunches, their intensity and their filling pattern~\cite{Report:LHCStandardFillingSchemes} depends strongly on the experimental demands.
For optics measurements a single low intensity bunch of about \num{e10} particles is used for each beam.
After injection, the beam energy is increased up to collision energy (\qty{6.8}{\tera\electronvolt} in Run~3) while the \(\beta^{\ast}\) is squeezed.
This process, called combined ramp and squeeze has been used in the LHC since 2017~\cite{IPAC:Camillocci:CombinedRampAndSqueeze}.
Before then the squeezing process only started once the energy had reached collision value.
At top energy with squeezed optics, also called as flat-top, the optics is confirmed and eventually adjusted before collisions start with stable beams.
The fill ends when dumping the beams, after which the cycle ends by a ramp down of the magnets' currents.

\begin{figure}[h]
    \centering
    \includegraphics*[width=0.9\linewidth]{Figures/Optics_Measurements_Corrections_at_LHC/lhc_cycle.pdf}
    \caption{Simplified illustration of the LHC nominal cycle.}
    \label{figure:lhc_cycle}
\end{figure}

Starting in Run~\num{3}, some additional complexities were added to the cycle that are not shown in \cref{figure:lhc_cycle}.
In \num{2022} a \betastar-leveling was introduced, were collisions start at a \(\beta^{\ast}\) of \qty{60}{\centi\meter} and the \(\beta^{\ast}\) is progressively reduced to \qty{30}{\centi\meter} during stable beams.
This is done in order to limit pile-up for the experiments (at around \num{52} events per bunch crossing) and the impact on the triplets cryogenics capacity~\cite{CERN:Fartoukh:LHC_Config_Run3, CERN:Ferlin:Cryogenics}.
This \betastar-leveling is moved to start at \(\beta^{\ast} =\) \qty{1.2}{\meter} in \num{2023} and \num{2024}.
Starting in \num{2023} an anti-telesqueeze is performed in the ramp to help the forward physics experiments, and a crossing-angle rotation at LHCb (IP\num{8}) is done when reaching flat-top in order to maintain physics conditions at the IP regardless of the LHCb spectrometer polarity~\cite{CERN:Fartoukh:LHC_Config_Run3}.

%----------------------------------------------------------------------------------------

\section{Optics Measurements and Corrections}
\label{section:optics_measurements_and_corrections}

The LHC requires good control of the optics functions, and for instance a good control of the \betafunctions is essential for safe beam operations due to the destructive power of the LHC beams.
Moreover, a good control of the \(\beta^{\ast}\) is necessary to guarantee good luminosity production.
To validate the machine optics and to identify possible errors beam measurements are performed, from which corrections bringing the machine as close to the nominal scenario as possible are calculated and applied. 
This chapter gives an overview of the LHC machine and its operational cycle, as well as the essential beam instrumentation used for optics measurements.
Additionally, concepts of corrections methods are given.

\subsection{Beam Instrumentation for Optics Measurements}

\subsubsection{Beam Position Monitors}

% \subsubsection{Beam Current Monitors}

\subsubsection{BBQ System for Tune Measurements}

\subsubsection{Excitation Devices}


Measurements of beam optics are done by generating large transverse beam oscillations, typically much larger than the natural beam size.
A non-destructive method for beam excitation in hadron machines is achieved using an AC-dipole

\subsection{Optics Measurements}
\label{subsection:optics_measurements}

- excite
- harmonic analysis
- optics analysis

\subsubsection{Turn-by-turn Measurements}

\subsection{Measurement of Linear Coupling}

\subsection{Correction Principles}

\subsubsection{Global Corrections}

\subsubsection{Local Corrections}

Here mention the need for new stuff as things get hairy with the LHC upgrades.