\chapter{Optics Measurements and Corrections at the LHC}
\label{chapter:lhc_omc} % For referencing the chapter elsewhere, use \cref{chapter:lhc_omc}

The Large Hadron Collider (LHC) is a \qty{26.659}{\kilo\metre} synchrotron collider located at the European Center for Nuclear Research (CERN), on the French-Swiss border.
It is part of CERN's Accelerator Complex, illustrated in \cref{figure:cern_accelerator_complex}, a chain of particle accelerators progressively bringing protons and heavy ions up to an energy of \qty{6.8}{\tera\electronvolt} per beam, as of \num{2023}.

\begin{figure}[!htb]
  \centering
  \includegraphics*[width=0.9\linewidth]{Figures/Optics_Measurements_Corrections_at_LHC/cern_accelerator_complex.png}
  \caption{The CERN Accelerator Complex in \num{2022}, not to scale~\cite{Website:CERN_Accelerator_Complex_Resource}. For typical LHC operation, a proton beam is produced in \(\mathrm{LINAC}\)~\num{4} and follows the chain: \(\mathrm{LINAC}\)\num{4} \(\rightarrow\) \(\mathrm{PSB}\) \(\rightarrow\) \(\mathrm{PS}\) \(\rightarrow\) \(\mathrm{SPS}\) \(\rightarrow\) \(\mathrm{LHC}\).}
  \label{figure:cern_accelerator_complex}
\end{figure}

Accelerated particles go through a chain of different particle accelerators before reaching their experimental destinations.
For protons colliding in the LHC, the first step is a linear accelerator, LINAC\num{4}, which accelerates them up to a kinetic energy of \qty{160}{\mega\electronvolt}.
Next, the protons are injected into the Proton Synchrotron Booster (PSB), where they are accelerated to an energy of \qty{1.4}{\giga\electronvolt}.
The next stage is the Proton Synchrotron, in which they will reach \qty{25}{\giga\electronvolt}; then the Super Proton Synchrotron (SPS) where they are accelerated to \qty{450}{\giga\electronvolt}, when they are finally injected into the LHC.

The LHC circulates two counter-rotating hadron beams, each in their ring, which are made to collide at four Interaction Points (IPs) to provide data for High Energy Physics (HEP) experiments.
The main data-taking experiments on the LHC are ATLAS~\cite{ATLAS_Paper,Website:ATLAS,Website:ATLAS_CDS}, LHCf~\cite{LHCf_Paper,Website:LHCf,Website:LHCf_CDS}, ALICE~\cite{ALICE_Paper,Website:ALICE,Website:ALICE_CDS}, CMS~\cite{CMS_Paper,Website:CMS,Website:CMS_CDS}, TOTEM~\cite{TOTEM_Paper,Website:TOTEM,Website:TOTEM_CDS}, LHCb~\cite{LHCb_Paper,Website:LHCb,Website:LHCb_CDS} and MoEDAL~\cite{MoEDAL_Paper,Website:MOEDAL,Website:MOEDAL_CDS}.
The LHC is currently the world's highest-energy hadron colliding machine, colliding beams at \qty{13.6}{\tera\electronvolt} center-of-mass energy as of Run~\num{3}, \num{2023}.

%----------------------------------------------------------------------------------------

\section{The LHC Lattice}
\label{section:lhc_lattice}

The LHC lattice consists of eight \intro{octants} each intersected by an \intro{Insertion Region} (IR).
Conventionally, the segment between two \IRs is called an \intro{arc} and the arc between IR1 and IR2 is named Arc12, and similarly for other arcs.
An octant is defined as going from mid-arc to mid-arc around a given \IR which is located at its center.
Each octant is named according to the \IR at its center: the octant with \(\mathrm{IR1}\) at its center is named Octant1, and similarly for other octants.
An illustration and a detailed description on naming conventions can be found in \cref{appendix:naming_conventions}.
% The \IRs host either an \intro{Interaction Point} (IP) where beams are made to collide (ATLAS at IP1, ALICE at IP2, CMS at IP5 and LHCb at IP8) or important instrumentation (momentum cleaning at IR3, Radio-Frequency cavities at IR4, beam dump system at IR6 and betatron cleaning at IR7).

\begin{figure}[!h]
  \centering
  \includegraphics*[width=0.65\linewidth]{Figures/Optics_Measurements_Corrections_at_LHC/lhc_schematic.pdf}
  \caption{Schematic of the LHC layout, adapted from~\cite{PHD:Poyet}.}
  \label{figure:lhc_schematic_layout}
\end{figure}

Beam~\num{1} rotates clockwise in its ring when viewing the LHC from above, and Beam~\num{2} rotates counter-clockwise as viewed from above.
The beams occupy separate apertures, or beam pipes, except in the \IRs where they are eventually made to collide.
The layout of the LHC can be seen in simplified schematic form in~\cref{figure:lhc_schematic_layout}, and full details can be found in the LHC Design Report~\cite{BOOK:Bruning:LHC_Design_Report_Main_Ring,BOOK:Bruning:LHC_Design_Report_Infrastructure,BOOK:Benedikt:LHC_Design_Report_Injector_Chain}.

\subsection{The LHC Arcs}
\label{subsection:lhc_arcs}

Each arc in the LHC is made up of 23 cells and is approximately \qty{2.45}{\kilo\meter} long.
The layout of an LHC arc cell is given in~\cref{figure:lhc_schematic_arc_cell}, and a clearer schematic representation can be found in~\cite{MASTERS:Keintzel:Arc_Cell_Options_HELHC}.
The cell is based on a FBDB (FODO with Bends) layout alternating focusing and defocusing quadrupoles interspaced with dipoles.
These elements are all superconducting and are commonly labeled \intro{MQF}, \intro{MQD} and \intro{MB}, respectively.
% Details on equipment codes can be found at~\cite{CERN:Equipment_Codes}.

\begin{figure}[!hbt]
  \centering
  \includegraphics*[width=0.99\linewidth]{Figures/Optics_Measurements_Corrections_at_LHC/lhc_schematic_arc_cell.png}
  \caption{Schematic of an LHC arc cell~\cite{BOOK:Bruning:LHC_Design_Report_Main_Ring}.}
  \label{figure:lhc_schematic_arc_cell}
\end{figure}

Each cell contains two \intro{MQ} (one MQF, one MQD) with three MB in between, for a total of 6 MBs per cell.
The MBs are all powered in series and, for size constraint reasons~\cite{BOOK:Bruning:LHC_Design_Report_Main_Ring}, are of a dual bore design.
The MQs are themselves also powered in series but split in two families: one power circuit is dedicated to MQF magnets and another circuit for the MQD magnets, where each arc holds a circuit for each family.
As a consequence, these elements can only be trimmed in groups.
% A cross-section of the main LHC dipole assembly is given in~\cref{figure:lhc_main_dipole_cross_section}

% \begin{figure}
%   \centering
%   \includegraphics[width=0.85\linewidth]{Figures/Optics_Measurements_Corrections_at_LHC/lhc_main_dipole_cross_section.jpeg}
%   \caption{Cross-section of the LHC main dipole assembly~\cite{CERN:AC_Team:LHC_Dipole}.}
%   \label{figure:lhc_main_dipole_cross_section}
% \end{figure}

As a part of the main assemblies are superconducting \intro{spool piece magnets}, correctors used for the local compensation of magnetic errors in the main arc magnets~\cite{BOOK:Bruning:LHC_Design_Report_Main_Ring}.
These include sextupolar correctors, sextupolar spool pieces named \intro{MCS} and mounted on the ends of every main dipole, used to correct \(b_3\) errors of the MBs.
Similarly, octupole and decapole spool pieces are included and used for the compensation of \(b_4\) and \(b_5\) errors in the main arc magnets.
The octupole correctors are named \intro{MCO} while the decapole correctors are named \intro{MCD}, and both are nested together in an assembly named \intro{MCDO} which is mounted on the end of every second MB.
The spool piece magnets in the LHC are single aperture and powered in series similarly to the MQs, with one circuit assigned for each magnet family.

In addition to spool piece magnets, linear and nonlinear \intro{lattice correctors} are mounted on the main arc quadrupoles MQs.
These lattice correctors are powered in series per family, and independently for each beam.
Horizontal and vertical orbit correctors, respectively \intro{MCBH} and \intro{MCBV}, are installed at each focusing and defocusing MQ.
Normal trim quadrupoles, named \intro{MQT}, are primarily used for tune correction.
In each arc four MQTs are rotated by \qty{45}{\degree} to form skew quadrupoles, named \intro{MQS}, used for linear coupling correction.
Normal and skew sextupoles \intro{MS} and \intro{MSS}, used for natural chromaticity and chromatic coupling correction respectively, are mounted on the MQs.
Landau octupoles \intro{MO} provide damping of coherent oscillations, and are split into two families (focusing and defocusing) powered in series, such that there are two families per arc and per beam.

\Cref{figure:lhc_arc_cell_latwiss} shows a simplified layout of an LHC arc cell's elements as well as \(\beta\) and dispersion functions for \num{2022} optics at \(\beta^{\ast} =\) \qty{30}{\centi\meter}.
In the layout part of the plot element powerings are indicated, with MBs in \textcolor{latwiss_blue}{blue}, MQs in \textcolor{latwiss_red}{red}, MSs in \textcolor{latwiss_yellow}{yellow}, MOs in \textcolor{latwiss_green}{green} and beam position monitors (BPMs) as grey patches.
Note that not all elements are indicated there.
\Cref{figure:lhc_arc23_latwiss} shows a similar plot but across LHC arc\num{23} for the same optics.

\begin{figure}[!hbt]
  \centering
  \includegraphics*[width=0.99\linewidth]{Figures/Optics_Measurements_Corrections_at_LHC/lhc_arc_cell.pdf}
  \caption{Simplified layout and optics functions in an LHC arc cell for \(\beta^{\ast} =\) \qty{30}{\centi\meter} optics.}
  \label{figure:lhc_arc_cell_latwiss}
\end{figure}

\begin{figure}[!hbt]
  \centering
  \includegraphics*[width=0.99\linewidth]{Figures/Optics_Measurements_Corrections_at_LHC/lhc_arc23.pdf}
  \caption{Simplified layout and optics functions in LHC arc\num{23} for \(\beta^{\ast} =\) \qty{30}{\centi\meter} optics.}
  \label{figure:lhc_arc23_latwiss}
\end{figure}

The purpose of the arcs is that of beam transport to the more specific parts of the machine.

\subsection{The LHC Experimental Interaction Regions}
\label{subsection:lhc_eirs}

In the middle of each octant, in between arcs, the LHC hosts \intro{long straight sections} (see \cref{appendix:naming_conventions} for details) with specific purposes.
Each of these is centered around an Insertion Region where a dedicated layout is in place to fulfill the section's purpose.
The purpose of each straight section is briefly stated on \cref{figure:lhc_schematic_layout} and detailed in \cref{table:lhc_straight_sections}.

\begin{table}[!hbt]
  \centering
  \begin{tblr}{colspec={cc}}
      \hline
      \text{Straight Section} & \text{Description}                                 \\
      \hline
      IR\num{1}               & ATLAS and LHCf Experiments                           \\
      IR\num{2}               & ALICE Experiment and B\num{1} Injection              \\
      IR\num{3}               & Momentum Cleaning (Collimation)                      \\
      IR\num{4}               & RF Systems and LHC Instrumentation                   \\
      IR\num{5}               & CMS and TOTEM Experiments                            \\
      IR\num{6}               & Beam Dump System                                     \\
      IR\num{7}               & Betatron Cleaning (Collimation)                      \\
      IR\num{8}               & LHCb and MoEDAL Experiments, and B\num{2} Injection  \\
      \hline
  \end{tblr}
  \caption{Description and purpose of the straight sections in the LHC}.
  \label{table:lhc_straight_sections}
\end{table}

Of interest to this thesis are the \intro{experimental insertions}, located in IR\num{1}, IR\num{2}, IR\num{5}, and IR\num{8}, where the beams are made to collide.
An insertion region in which beams are made to collide is called an \intro{Interaction Region}, or sometimes Experimental Interaction Region.
In this document, when using the short form IR, it is meant to refer to an Interaction Region.

At the center of the IR, beams are made to collide at the \intro{Interaction Point} (IP).
In order to achieve high luminosity during collisions, and as shown in \cref{section:luminosity}, the \(\beta\)-functions at the IPs are squeezed to very small values.
\Cref{figure:ir5_and_around} shows the \(\beta\)-functions in the LHC around IP5 at both injection and collision optics, where the squeeze is apparent.

\begin{figure}[!hbt]
  \centering
  \includegraphics*[width=0.99\linewidth]{Figures/Optics_Measurements_Corrections_at_LHC/ir5_surroundings_optics_2.pdf}
  \caption{The horizontal and vertical \betafunctions in the LHC around IP5 at injection optics (top) and collision optics (bottom). Notice the drastically different scales on the vertical axes.}
  \label{figure:ir5_and_around}
\end{figure}

During normal operation the \(\beta^{\ast}\) at ATLAS and CMS is very squeezed: \(\beta^{\ast} =\) \qty{30}{\centi\metre} during collisions as of \num{2023}.
For the configuration, at ALICE and LHCb the \(\beta^{\ast}\) are only squeezed to higher values, \qty{10}{\meter} and \qty{2}{\meter} respectively in \num{2023}.
During collisions involving ions (Pb-Pb and p-Pb) the \(\beta^{\ast}\) is reduced at ALICE and LHCb.
\Cref{table:lhc_betastars_configurations} summarizes the \(\beta^{\ast}\) values for the different experiments and configurations.

\begin{table}[!htb]
  \centering
  $\begin{tblr}{colspec={cccc}}
      \hline
      \SetCell[r=2,c=1]{m,c} \text{IP} & \SetCell[c=3]{c} \mathbf{\beta^{\ast}}                                                    \\
      \cline{2-4}
                                       &  \text{Injection Optics}  &  \text{Proton Collisions} &  \text{Ion Collisions}   \\
      \hline
      \text{IP\num{1}}                 &  \qty{11}{\meter}         &  \qty{30}{\centi\meter}   &  \qty{50}{\centi\meter}  \\
      \text{IP\num{2}}                 &  \qty{11}{\meter}         &  \qty{10}{\meter}         &  \qty{50}{\centi\meter}  \\
      \text{IP\num{5}}                 &  \qty{11}{\meter}         &  \qty{30}{\centi\meter}   &  \qty{50}{\centi\meter}  \\
      \text{IP\num{8}}                 &  \qty{11}{\meter}         &  \qty{2}{\meter}          &  \qty{50}{\centi\meter}  \\
      \hline
  \end{tblr}$
  \caption{Value of the \(\beta^{\ast}_{x,y}\) at different IPs for different optics configurations as of Run~\num{3}.}
  \label{table:lhc_betastars_configurations}
\end{table}

In order to achieve a small \(\beta^{\ast}\) at the IPs, the beams are focused using a superconducting \intro{triplet} of quadrupoles just before the IP, on either side~\cite{CERN:Ostojic:Improved_Optical_System_LHC_Triplet}.
The triplet is optimized to be symmetric~\cite{CERN:DAmico:Analysis_Generic_Insertions}, with Q\num{1} and Q\num{3} being the same length at \qty{6.3}{\meter} and Q\num{2} split into two sub-magnets Q\num{2}a and Q\num{2}b of \qty{5.5}{\meter} each.
All three magnets are powered in series but can be adjusted individually using dedicated trim converters~\cite{PAC:Bordry:LHC_Inner_Triplet_Powering}.

This arrangement of three quadrupoles allows for a stong focusing of the \(\beta\)-functions in both transverse planes.
However, such an arrangement leads to high \(\beta\)-functions in the triplet quadrupoles themselves and neighbouring elements.
An illustration of the area close to IP\num{5} for collision optics with \(\beta^{\ast} =\) \qty{30}{\centi\metre} is given in \cref{figure:lhc_ir5_zoomed}.

\begin{figure}[!hbt]
  \centering
  \includegraphics*[width=0.99\linewidth]{Figures/Optics_Measurements_Corrections_at_LHC/lhc_ir5_zoomed.pdf}
  \caption{The simplified elements layout and \(\beta\)-functions around IP\num{5} at collision optics, without crossing angles.}
  \label{figure:lhc_ir5_zoomed}
\end{figure}

On the layout plot the three \textcolor{latwiss_red}{red patches} closest to the IP location correspond to Q\num{1}, Q\num{2} and Q\num{3} respectively, the triplet quadrupoles.
The \textcolor{latwiss_blue}{blue patches} correspond to D\num{1} (first batch) and D\num{2} (outer dipole), the \intro{separation / recombination dipoles} responsible for bringing the beams together / apart in the common region from / to their separate apertures in the arcs.
The separation dipole D\num{1} is made of six \qty{3.4}{\meter} long normal conducting magnets while D\num{2} is a superconducting twin aperture magnet \qty{9.45}{\meter} long.
Further quadrupoles are matching quadrupoles and will be discussed later.
The grey lines correspond to the location of Beam Position Monitors (BPMs), measurement instrumentation.

Due to the large \(\beta\)-functions in the triplet quadrupoles, as can be seen in \cref{figure:lhc_ir5_zoomed}, any magnetic error in the elements of the IR would have a strong impact on the beam dynamics.
To enable correction of these errors, linear and non-linear corrector magnets are installed along the IR, distributed symmetrically around the IP: every corrector magnet on one side of the IP has a counterpart on the other side.
Of interest to this thesis are the a\num{2} skew quadrupole correctors installed just before Q\num{3} on each side of the IP, the locations of which are highlighted in \cref{figure:lhc_ir5_zoomed} by green vertical lines.
A schematic of the corrector layout is shown in \cref{figure:lhc_ir_corrector_layout}.
All correctors are individually powered magnets.
\todo{ARE THEY?}

\begin{figure}[!hbt]
  \centering
  \includegraphics*[width=0.92\linewidth]{Figures/Optics_Measurements_Corrections_at_LHC/corrector_package.png}
  \caption{Layout of the triplet magnets and the linear and nonlinear correctors in the LHC experimental insertions~\cite{CERN:Bruning:Dynap_Studies}, showing common aperture magnets. The skew quadrupole correctors correspond to order a\num{2} and are located in the C\num{2} package.}
  \label{figure:lhc_ir_corrector_layout}
\end{figure}

In order to prevent parasitic crossings between the two beams' bunches around the IP during collisions, \intro{separation bumps} are implemented in a single transverse plane for each IP, in the form of closed orbit bumps.
Due to the presence of these bumps, in order to reach collisions a \intro{crossing angle} is introduced.
The optics in IR\num{1} and IR\num{5} are identical except for the crossing schemes: the crossing angle is in the vertical plane at IR\num{1} and in the horizontal plane at IR\num{5}.
Respectively, the separation bumps are in the horizontal plane at IR\num{1} and in the vertical plane at IR\num{5}.
\Cref{figure:lhc_crossing_schemes_ip15} shows the crossing schemes for both IR\num{1} and IR\num{5} at collision optics, with the location of the triplets highlighted in grey and the that of the separation dipoles in yellow.

\begin{figure}[!hbt]
  \centering
  \includegraphics*[width=0.99\linewidth]{Figures/Optics_Measurements_Corrections_at_LHC/lhc_crossing_schemes_ip15.pdf}
  \caption{Crossing schemes for IR\num{1} and IR\num{5} at collision optics.}
  \label{figure:lhc_crossing_schemes_ip15}
\end{figure}

Other IRs, not of interest to this thesis, have significantly different layouts which can be found in details in~\cite{BOOK:Bruning:LHC_Design_Report_Main_Ring,PHD:Vanbavinckhove}.

\subsection{Matching Sections and Dispersion Suppressors}
\label{subsection:matching_sections_dispersion_suppressors}

Assuring the transition between the arcs and the specific optics conditions of the IRs are \intro{matching sections} and \intro{dispersion suppressors}, as can be seen on \cref{figure:ir5_and_around}.
Note that the area designated as matching section on the figure also includes the dispersion suppressor.
Together, the two segments are responsible for matching the TWISS functions between the arcs and the IRs, and for reducing the dispersion to near-zero value, respectively.

The dispersion suppressor is made of two arc cells containing two instead of the regular three dipoles.
The quadrupoles in these cells, Q\num{7} to Q\num{10}, are powered individually.
The dispersion suppressor leading to IP\num{5} can be seen on \cref{figure:lhc_dispersion_suppressor}, where the beam travels from left to right.

\begin{figure}[!hbt]
  \centering
  \includegraphics*[width=0.99\linewidth]{Figures/Optics_Measurements_Corrections_at_LHC/lhc_dispersion_suppressor.pdf}
  \caption{Simplified layout and optics functions in the dispersion suppressor leading beam \num{1} to IP\num{5}, for \(\beta^{\ast} =\) \qty{30}{\centi\meter} optics.}
  \label{figure:lhc_dispersion_suppressor}
\end{figure}

The matching section is made of three individually powered superconducting quadrupoles, Q\num{4} to Q\num{6}, used to match the TWISS functions from their out of the arcs to that at the entrance of the triplets.
In order to help the matching to the arc, the trim quadrupoles QT\num{11} to QT\num{13}, adjacent to the FODO quadrupoles Q\num{11} to Q\num{13}, are also individually powered and used for the matching.
The full segment, from the start of the dispersion suppressor to just before separation dipole D\num{2}, is shown in \cref{figure:lhc_matching_section}.

\begin{figure}[!hbt]
  \centering
  \includegraphics*[width=0.99\linewidth]{Figures/Optics_Measurements_Corrections_at_LHC/lhc_matching_section.pdf}
  \caption{Simplified layout and optics functions in the matching section leading beam \num{1} to IP\num{5}, for \(\beta^{\ast} =\) \qty{30}{\centi\meter} optics.}
  \label{figure:lhc_matching_section}
\end{figure}

\subsection{The ATS Optics Scheme}
\label{subsection:lhc_ats_optics_scheme}

When pushing the \(\beta^{\ast}\) to smaller values, and therefore the \(\beta\)-functions in the triplets to higher ones, the chromatic effects produced by the triplet quadrupoles~\cref{equation:natural_chromaticity_approximation} increase drastically and need to be corrected.
As the beam energy reaches its maximum, the beam size gets smaller and an aperture margin that allows to increase the \(\beta\)-functions appears in the arcs.

The \intro{Achromatic Telescopic Squeeze} (ATS) optics scheme~\cite{CERN:Fartoukh:ATS_Report,PRAB:Fartoukh:Achromatic_Telescopic_Squeeze,IPAC:Pojer:LHC_ATS_Experience} consists in splitting the reduction of the \(\beta^{\ast}\) (the squeeze) in two stages.
In the first one, the \intro{pre-squeeze}, the \(\beta^{\ast}\) is reduced using the matching quadrupoles around the affected IP.
As using these matching quadrupoles has several limits (magnet strength, chromaticity correction, orbit control), a second stage is necessary.
In this second stage, the \intro{tele-squeeze}, the \(\beta^{\ast}\) is reduced by using the matching quadrupoles in the nearby IRs: IR\num{2} and IR\num{8} for the tele-squeeze of IR\num{1}, and IR\num{4} and IR\num{6} for the tele-squeeze of IR\num{5}.
Sectors \numlist{81;12;45;56} are therefore called ATS sectors.
This modulation in the second stage sends \(\beta\)-beating waves down the arcs, which make the \(\beta\)-functions peak at the location of sextupoles and octupoles in the those arcs, enhancing their efficiency.

The ratio between the \(\beta^{\ast}\) at the end of the pre-squeeze (\(\beta^{\ast}_{Pre}\)) and the \(\beta^{\ast}\) at the end of the tele-squeeze (\(\beta^{\ast}_{Tele}\)) is called \intro{tele-index} and is denoted \(r_{Tele}\).
It is defined as:

\begin{equation}
  r_{Tele} = \frac{\beta^{\ast}_{Pre}}{\beta^{\ast}_{Tele}}
  \label{equation:tele_index}
\end{equation}

\Cref{figure:lhc_ats_scheme} shows the \(\beta\)-functions at \qty{6.5}{\tera\electronvolt} around IP\num{5}, where one can see the \(\beta\)-beating wave in the neighbouring ATS sectors \num{45} and \num{56}.

\begin{figure}[!hbt]
  \centering
  \includegraphics*[width=0.99\linewidth]{Figures/Optics_Measurements_Corrections_at_LHC/lhc_ats_wave.pdf}
  \caption{The \(\beta\)-functions in sectors \num{45} and \num{56} at different points in the squeeze for the \num{2022} optics: at the end of the pre-squeeze (top) and at the end of the tele-squeeze (bottom).}
  \label{figure:lhc_ats_scheme}
\end{figure}

This ATS optics scheme has been used in the LHC starting Run~\num{2} and allowed reducing the collision optics \(\beta^{\ast}\) from its design value of \qty{55}{\centi\meter} to \qty{30}{\centi\meter}.
It is the operational baseline of Run~\num{3}.

%----------------------------------------------------------------------------------------

\section{The Operational Cycle of the LHC}
\label{section:lhc_operational_cycle}

The LHC operational cycle~\cite{Report:LHCModes}, illustrated in \cref{figure:lhc_cycle}, begins with a \intro{pre-cycle} of certain magnetic elements~\cite{Report:LHCMagnetsPreCycles}.
During pre-cycle no beams are present in the rings and the respective element currents are increased up to several \unit{\tera\electronvolt} beam energy configuration, to ensure the reproducibility of the magnetic fields over successive fills.
The exact nature of the pre-cycle depends on the magnetic elements, and a pre-cycle is not necessarily performed before each fill.

After the pre-cycle comes the \intro{injection} stage: beams are injected from the Super Proton Synchrotron (SPS) at an energy of \qty{450}{\giga\electronvolt}.
First a probe beam consisting of just a few bunches is injected to check the validity of several systems (injection interlock, orbit, tune, chromaticity and coupling control), then a physics beam meant for collisions is injected.
At injection optics the \(\beta^{\ast}_{x,y}\) at the main colliding IPs (IP\num{1} and IP\num{5}) is \qty{11}{\metre}.
The number of bunches, their intensity and their filling pattern~\cite{Report:LHCStandardFillingSchemes} depends strongly on the experimental demands.
For optics measurements for instance, between one and three low intensity, non-colliding bunches of about \num{e10} protons per bunch are injected for each beam.
For luminosity production a larger number of high intensity bunches is injected: in the order of \num{e3} bunches, with \(\ge\) \num{e11} protons per bunch.

After injection, the beam energy is increased up to collision energy (\qty{6.8}{\tera\electronvolt} in Run~3) while the beams are squeezed and the \(\beta^{\ast}\) reduced.
This process, called \intro{combined ramp and squeeze}, has been used in the LHC since 2017~\cite{IPAC:Camillocci:CombinedRampAndSqueeze}.
Before then the squeezing process only started once the energy had reached collision value.

After reaching top energy, a configuration known as \intro{flat-top}, another \intro{squeeze} is performed to bring the \(\beta^{\ast}\) to collision value.
This is when the ATS scheme mentioned in \cref{subsection:lhc_ats_optics_scheme} happens.

In a final step before luminosity production, called \intro{adjust}, the last few needed parameters are adjusted to bring the beams into collision: tunes, crossing angles, collapse of the separation bumps.
This configuration called \intro{stable beams} is kept throughout the luminosity production for the fill.
The fill ends when the beams are extracted from the machine, a.k.a \intro{beam dump}, after which the cycle ends by a \intro{ramp down} of the magnets' currents.

\todo{SEE WITH TOBIAS FOR THE TUNES IN THE CYCLE.
Note that the transverse tunes vary during the cycle.
At injection, the tunes are \((62.27, 60.315)\)
Inj: Don't know?
Ramp Squeeze: (62.275, 60.293) to (62.28, 60.31)
Squeeze part1 (130to60cm): (62.28, 60.31) no change
Squeeze part (60to30cm): (62.28, 60.31) no change
QCHANGE?: (62.28, 60.31) to (62.311, 60.318)
Physics: (62.311, 60.318) to (62.314, 60.319)
Physics[END] to (62.31, 60.32)}

\todo{Actually, from Delphine (\url{https://indico.cern.ch/event/1224987/contributions/5153318/attachments/2579893/4449512/TheLhcCycle-Chamonix2022_v2.pdf}).
At injection we have (60.293, 62.275) which goes to (60.318, 62.311) during squeeze / adjust, and stays so for collisions.}


\begin{figure}[!hbt]
    \centering
    \includegraphics*[width=0.9\linewidth]{Figures/Optics_Measurements_Corrections_at_LHC/lhc_cycle.pdf}
    \caption{Simplified illustration of the LHC nominal cycle.}
    \label{figure:lhc_cycle}
\end{figure}

Starting in Run~\num{3}, some additional complexities were added to the cycle that are not shown in \cref{figure:lhc_cycle}.
In \num{2022} a \(\beta^{\ast}\)-leveling was introduced, were collisions start at a \(\beta^{\ast}\) of \qty{60}{\centi\meter} and the \(\beta^{\ast}\) is progressively reduced to \qty{30}{\centi\meter} during stable beams.
This is done in order to limit pile-up for the experiments (at around \num{52} events per bunch crossing) and the impact on the triplets cryogenics capacity~\cite{CERN:Fartoukh:LHC_Config_Run3, CERN:Ferlin:Cryogenics}.
This \(\beta^{\ast}\)-leveling is moved to start at \(\beta^{\ast} =\) \qty{1.2}{\meter} in \num{2023} and \num{2024}.
Starting in \num{2023} an anti-telesqueeze is performed in the ramp to help the forward physics experiments, and a crossing-angle rotation at LHCb (IP\num{8}) is done when reaching flat-top in order to maintain physics conditions at the IP regardless of the LHCb spectrometer polarity~\cite{CERN:Fartoukh:LHC_Config_Run3}.

%----------------------------------------------------------------------------------------

\section{Optics Measurements and Corrections}
\label{section:optics_measurements_and_corrections}

The quality of the LHC optics has a significant impact on the machine's performance.
For instance, the luminosity achieved by the machine is directly determined by the \(\beta\)-functions at the IPs, as seen in \cref{section:luminosity}.
Furthermore, a good control of the \(\beta\)-functions is essential for safe beam operations due to the destructive power of the LHC beams, and the machine is subject to strict limits on the deviation from model values~\cite{CERN:Bruning:Field_Quality_Spec_LHC_Main_Dipoles}.
One can then define the \intro{beta-beating}, a good indicator of the quality of the linear optics, as the relative deviation of the machine's \(\beta\)-functions from that of the designed values.
It is defined as:

\begin{equation}
  \frac{\Delta \beta_z(s)}{\beta_z(s)} = \frac{\beta_z(s)_{\mathrm{measured}} - \beta_z(s)_{\mathrm{model}}}{\beta_z(s)_{\mathrm{model}}} \quad \text { where } z = x, y \text{ .}
  \label{equation:beta_beating_definition}
\end{equation}

In order to verify the machine's beam optics and find any potential faults, or deviations from the model values, beam measurements are necessary.
From these, comparisons to model values are made which allow for an assessment and understanding of the errors in the machine; and corrections can be calculated and applied to bring back the optics as close to the nominal scenario as possible.
As the linear optics functions impact the non-linear phenomenology of an accelerator, a well understood and corrected linear optics a pre-requisite to study of the non-linear dynamics.
Correction of the linear optics functions towards their nominal values also leads to an enhanced rms closed orbit around the ring since the orbit feedback algorithms in the LHC assume the nominal LHC model~\cite{PRAB:Tomas:Record_Low_Beta_Beating_in_the_LHC, PRAB:Persson:LHC_Optics_Commissioning_OnePercent}.

\todo{Maybe include a plot showing beta-beating in 2022 commissioning before and aftrer corrections.}

\subsection{Beam Instrumentation for Optics Measurements}
\label{subsection:beam_instrumentation_for_optics_measurements}

Circular machines such as the LHC include a variety of beam instrumentation devices which serve various purposes, from injection kickers and feedback systems used in regular operations to dedicated devices for optics measurements.

\subsubsection*{Beam Position Monitors and Tune Measurement}

\intro{Beam Position Monitors} (BPMs) are one of the most crucial devices for beam diagnostics.
They measure the transverse center of charge of circulating bunches, either in a given plane for single plane BPMs, or in both planes simultaneously for dual plane BPMs.
In the LHC, this centroid beam position can be measured on a turn-by-turn and bunch-by-bunch basis by around \num{500} dual plane BPMs across the machine.
So-called stripline BPMs are employed in the common apertures as they can discriminate between counter-rotating bunches, while button BPMs are used in the remaining portions of the machine~\cite{BOOK:Bruning:LHC_Design_Report_Main_Ring}. 
The location of BPMs in the lattice can be seen as vertical grey lines, in an insertion region such as IR\num{5} in \cref{figure:lhc_ir5_zoomed,figure:lhc_matching_section} and in the arcs in \cref{figure:lhc_arc_cell_latwiss}.

% Measurement of the tune can be done in different ways.
% One way consists at measuring transverse oscillations and looking at the spectrum of these oscillations.
In the LHC, the \intro{Base Band Tune} (BBQ) system~\cite{CERN:Boccardi:LHC_Transverse_Diagnostics_Systems,CERN:Boccardi:LHC_BBQ_Tune_Chromaticity_Systems} provides continuous, passive monitoring of the tune by performing spectral analysis of the orbit data at a specific location in IR\num{4}.
The BBQ is also capable of measuring an estimation of the linear coupling at the measurement location, which can be used as a rough first estimate for the coupling in the machine.
While it is possible to assess tune and coupling with the BBQ without external excitation of the beam, the LHC chirp can generate small transverse oscillations to improve the quality of these measurements.

\subsubsection*{Experimental Kickers}




EXCITATION: Talk kicker, AC Dipole.
Measurements of beam optics are done by generating large transverse beam oscillations, typically much larger than the natural beam size.
A non-destructive method for beam excitation in hadron machines is achieved using an AC-dipole

\begin{figure}[!htb]
  \centering
  \includegraphics*[width=0.99\linewidth]{Figures/Optics_Measurements_Corrections_at_LHC/kick_vs_acdipole.pdf}
  \caption{Comparison of turn-by-turn data obtained from a single free kick (top) and an AC-Dipole excitation (bottom).}
  \label{figure:kick_vs_acdipole_tbt}
\end{figure}

\subsection{Optics Measurements}
\label{subsection:optics_measurements}

- excite
- harmonic analysis
- optics analysis
Talk about (and formula) reconstruction of the phase and the beta-functions for instance (see Tobias' PhD).

\begin{figure}[!htb]
  \centering
  \includegraphics*[width=0.9\linewidth]{Figures/Optics_Measurements_Corrections_at_LHC/betas_nominal_vs_driven.pdf}
  \caption{Resulting \(\beta\)-functions in a FODO lattice in the case of free and driven oscillations.}
  \label{figure:acdipole_beta_functions_vs_nominal}
\end{figure}

\begin{figure}[!htb]
  \centering
  \includegraphics*[width=0.99\linewidth]{Figures/Optics_Measurements_Corrections_at_LHC/bbeatings_acdipole.pdf}
  \caption{AC Dipole induced \(\beta\)-beating in a general case, for various phase advances between an element where the observation is made and the kicker.}
  \label{figure:ac_dipole_induced_beta_beating}
\end{figure}

\cite{PRAB:Miyamoto:Parametrization_Driven_Betatron_Oscillation}
\cite{PRAB:Tomas:Adiabaticity_Ramping_Process_AC_Dipole}

\subsection{Measurement of Linear Coupling}

\cite{PRAB:Tomas:CERN_LHC_OMC}

\subsection{Correction Principles}

\subsubsection{Global Corrections}

\subsubsection{Local Corrections}

Here mention the need for new stuff as things get hairy with the LHC upgrades.