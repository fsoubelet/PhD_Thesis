% Chapter 4

\chapter{Machine Learning for Interaction Region Local Coupling} % Main chapter title

\label{Chapter4} % For referencing the chapter elsewhere, use \Cref{Chapter4}

%----------------------------------------------------------------------------------------

Some paragraph before the first section.

%----------------------------------------------------------------------------------------

\section{Relevant Theory of Machine Learning}

\subsection{Regression and Classification}

\subsection{Neural Networks}

\subsection{Decision Trees and Ensemble Methods}

\subsection{Generalization and Overfitting}

\subsection{Unsupervised Learning?}

%----------------------------------------------------------------------------------------

\section{Identification of Sources with Machine Learning}

Which magnets are tilted, eventually how tilted are they?

%----------------------------------------------------------------------------------------

\section{Prediction of Corrections for Local Coupling}

Should start here with a simple approach: linear behavior, simple errors only in the region and a Regression model.
Show results.
However, move on to add that we need realistic scenario for the training data and that includes nonlinear errors, so maybe another type of model would be better suited.
Explain additions to simulations for training data, changes to model approach.
Show results, do a comparison and highlight improvements.

%----------------------------------------------------------------------------------------

\section{Reinforcement Learning for Segment by Segment???}

Currently, the SbS needs manual input, should be a good ground for a RL model to find its way.
Forget about Jaime's automatic matching, never got it to work anyway.

%----------------------------------------------------------------------------------------

\section{Conclusions}

%----------------------------------------------------------------------------------------