\chapter{Experimental Measurement and Correction of Interaction Region Local Coupling in the LHC Run~\num{3}} % Main chapter title

\label{Chapter:Experimental_Results} % For referencing the chapter elsewhere, use \cref{Chapter:Experimental_Results}

%----------------------------------------------------------------------------------------

Some paragraph before the first section.

%----------------------------------------------------------------------------------------

To include somewhere in the chapter:

\begin{figure}[!thb]
    \centering
    \includegraphics*[width=\columnwidth]{Figures/Chapter6/ir5_rws_rematching.pdf}
    \caption{\betabeating observed in the machine from the implementation of the rigid waist shift in IR5 at \qty{6.8}{\tera\electronvolt} and \(\beta^{*}=\) \qty{30}{\centi\meter}, before and after the trim of the optics re-matching knob. Prior to the application of any knob \betabeating was kept around \qty{5}{\percent} throughout the machine by applied corrections, which is also achieved by the re-matching knob.}
    \label{fig:ir5_rws_rematching}
\end{figure}

\section{Beam Tests 2021 Results}

Can refer to appendix \Cref{AppendixB} for the fills used.

\begin{figure}[!htb]
    \centering
    \includegraphics*[width=\columnwidth]{Figures/Chapter6/beamtest_sbs_abs_f1001_ip5.pdf}
    \caption{Propagation of the measured \absfoneohone around IP5 and of the reconstructed values from the determined correction, measured at \qty{450}{\giga\electronvolt} and \(\beta^{*}=\) \qty{11}{\meter}.}
    \label{fig:beamtest_sbs_abs_f1001_ip5}
\end{figure}

\section{Measurement and Correction of Local Coupling in the IRs at $\beta^{*}_{IP} = 0.3m$}

Can refer to appendix \Cref{AppendixB} for the fills used.

\begin{figure}[!htb]
    \centering
    \includegraphics*[width=\columnwidth]{Figures/Chapter6/commissioning_sbs_real_f1001_ip1.pdf}
    \caption{Propagation of the measured \(\Re f_{1001}\) around IP1 and the reconstructed values from the determined correction, measured at \qty{6.8}{\tera\electronvolt} and \(\beta^{*}=\) \qty{30}{\centi\meter}.}
    \label{fig:commissioning_sbs_real_f1001_ip1}
\end{figure}

\begin{figure}[!htb]
    \centering
    \includegraphics*[width=\columnwidth]{Figures/Chapter6/commissioning_sbs_imag_f1001_ip1.pdf}
    \caption{Propagation of the measured \(\Im f_{1001}\) around IP1 and the reconstructed values from the determined correction, measured at \qty{6.8}{\tera\electronvolt} and \(\beta^{*}=\) \qty{30}{\centi\meter}.}
    \label{fig:commissioning_sbs_imag_f1001_ip1}
\end{figure}

\begin{table}[!htb]
    \centering
    \caption{Luminosity gains observed at the main experiments ATLAS and CMS from implementing the method's suggested corrections.}
    \begin{tblr}{colspec={ccc}}
        \hline
        \SetCell[r=2,c=1]{m,c} \textbf{Experiment} & \SetCell[c=2]{c} \textbf{Luminosity Gain \unit{\percent}}                       \\
        \cline{2,3}                          &    \(\beta^{\ast} = \) \qty{30}{cm}    &    \(\beta^{\ast} = \) \qty{42}{cm}    \\
        \hline
        \textbf{ATLAS (IP1)}                 &    \num{9.5}                           &     \num{5.2}                          \\
        \textbf{CMS (IP5)}                   &    \num{3.5}                           &     \num{1.5}                          \\
        \hline
    \end{tblr}
    \label{tab:rws_lumi_gains}
\end{table}

Mention that trims (as seen above) are done a bit around and this is indeed the "best".
The lesser gains at CMS are expected as the method suggests there was less of an error to correct.

%----------------------------------------------------------------------------------------

\section{2022 LHC MD Blocks Results (Containment Plans)}

%----------------------------------------------------------------------------------------

\section{Conclusions}

%----------------------------------------------------------------------------------------