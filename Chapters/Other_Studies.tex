\chapter{Optics Studies}
\label{chapter:others_and_software}

Two further studies performed during the course of this thesis are presented in this chapter.
The first one is a statistical analysis of the phase measurement error in the \gls{LHC}, that aimed to determine the influence of the measuring \glspl{BPM} on the phase error.
The second one is an investigation on the sextupolar contribution to amplitude detuning in the LHC.

% In parallel to other works presented in this document, a substantial effort was put into software developments to improve both analysis and simulation tools.
% The relevant contributions and achievements over the course of this work are succinctly presented in the last section of this chapter.

%----------------------------------------------------------------------------------------

\section{Phase Error Dependency on BPM Type and Location}
\label{section:phase_error_dependency_on_bpm_type_and_location}

In the \gls{LHC} and many other circular colliders \gls{BPM} measurement data constitutes the primary source for \gls{optics} properties computation.
As different BPM types~\cite{CERN:Equipment_Codes} have different measurement resolution, and higher \glspl{beta-function} lead to better signal-to-noise ratios on measurements, these measurements and the subsequent results were suspected to be substantially affected by both the type of BPM used for the measurements and the value of the \(\beta\)-functions at the measuring devices.
This study was performed to investigate the influence of these properties on the precision of the reconstructed phase values from turn-by-turn data.

In practice the presence of systematic errors can influence different sets of measurements to not be totally independent.
In the following study measurements are assumed to be independent and \concept{identically distributed normal random variables} (IIDs), meaning repeating the same BPM measurement in identical conditions will spawn a range of values forming a Gaussian distribution.
As no sets of measurements were available that made use of specific groups of BPMs, a statistical analysis on the phase errors from previous measurements was needed to differentiate between BPMs of interest. 
Using the underlying properties of the measurements-derived distributions, one can numerically infer the impact on the phase measurement error of specific subgroups of measuring devices.

\subsection*{Measurement Data and Underlying Distributions}

The methods and considerations below can be applied on BPM turn-by-turn measurement data from synchrotron machines.
The following analysis was done on data taken during the LHC \Gls{run}~\num{2}, in \num{2018} and at \(\beta^{\ast} =\)~\qty{30}{\centi\meter}.
Information about the measurement data used for this analysis can be found in \cref{appendix:measurement_fills}.
\Cref{figure:bpms_betas_histogram} shows BPMs in the LHC are spread across a wide range of \(\beta\)-functions.

\begin{figure}[!htb]
    \centering
    \includegraphics*[width=\textwidth]{Figures/Other_Studies/bpms_betas_histogram.pdf}
    \caption{Distribution of BPM \(\beta\)-functions across the machine for the Run~\num{2} measurements used. A small amount of BPMs located at much higher \(\beta\) are not shown on this plot.}
    \label{figure:bpms_betas_histogram}
\end{figure}

The sum of two IIDs - which our measurements are assumed to be - is also normal.
Using the characteristic function of a normal distribution

\begin{equation}
    \varphi(t) = \exp \left( it \mu - \frac{\sigma^2 t^2}{2} \right) \text{ ,}
    \label{equations:normal_distribution_characteristic_function}
\end{equation}
and given that the characteristic function of the sum of two independent random variables \(X\) and \(Y\) is the product of their respective characteristic functions, one gets

\begin{equation}
    \begin{aligned}
        \varphi_{X+Y}(t) = \varphi_{X}(t) \varphi_{Y}(t) &= \exp \left( it \mu_{X} - \frac{\sigma_{X}^2 t^2}{2} \right) \exp \left(it \mu_{Y} - \frac{\sigma_{Y}^2 t^2}{2} \right)  \text{ ,} \\
                                                         &= \exp \left( it \left( \mu_{X} + \mu_{Y} \right) - \frac{\left( \sigma_{X}^2 + \sigma_{Y}^2 \right) t^2}{2} \right)  \text{ ,}
    \end{aligned}
\end{equation}
which corresponds to a normal distribution with its mean being the sum of the two means, and its variance being the sum of the two variances.
Respectively, one can deduce the same for the subtraction of two IIDs by changing the signs.

In particle accelerators, phase advances are measured from BPM to BPM and are, in the simplest form, the result of a subtraction.
As a consequence the repeated phase advance measurements obtained from a given BPM pair form a normal distribution with average phase advance \(\bar{\varphi}\) and standard deviation \(\sigma_{\mu}\).
Traditionally, the error on phase advance is computed via the standard deviation of \(N\) measurements:

\begin{equation}
    \varepsilon^2 = \frac{1}{N} \sum_{i=1}^{N}\big(\varphi_{i} - \bar{\varphi} \big)^2 \text{ .}
    \label{equation:phase_error_calculation}
\end{equation}

This means the squares of phase advance (and phase advance error) values \(\varepsilon^2\) form a chi-square distribution, with a number of degrees of freedom  \(k = N - 1\) since the sample mean is subtracted.
Therefore, grouping sufficient BPM pairs of the same type or with similar \(\beta\)-functions, and computing the distribution of their \(\varepsilon^2\) one obtains a chi-square distribution.
\Cref{figure:square_errors_histograms} shows those chi-square distributions for different ranges of \(\beta\)-functions combinations between measuring BPMs.

\begin{figure}[!htb]
    \centering
    \includegraphics*[width=\textwidth]{Figures/Other_Studies/phase_errors_squared_distributions.pdf}
    \caption{Chi-square distribution of the squares of phase measurement errors for different BPM combinations, differentiated by the \(\beta\)-functions at the locations of the measuring BPMs. For this plot a BPM was considered "low" below \(\beta =\)~\qty{100}{\meter}, "high" above \(\beta =\)~\qty{200}{\meter} and "medium" in between.}
    \label{figure:square_errors_histograms}
\end{figure}

Similarly, the positive real square roots of values from this chi-square distribution form a chi distribution, which can be derived with a change of variable \(x=y^2\).
\Cref{figure:phase_errors_histograms} shows these distributions for the same \(\beta\)-function ranges seen in \cref{figure:square_errors_histograms}.

\begin{figure}[!htb]
    \centering
    \includegraphics*[width=\textwidth]{Figures/Other_Studies/phase_errors_distributions.pdf}
    \caption{Distribution of the phase measurement errors for different BPM combinations, differentiated by the \(\beta\)-functions at the locations of the measuring BPMs. For this plot a BPM was considered "low" below \(\beta =\)~\qty{100}{\meter}, "high" above \(\beta =\)~\qty{200}{\meter} and "medium" in between.}
    \label{figure:phase_errors_histograms}
\end{figure}

\subsection*{Computing the Standard Deviation on Phase Advance}

A chi-square distribution with \(k\) degrees of freedom  is the distribution of a sum of the squares of \(k\) independent standard normal random variables.
Let \(X_{i}\) represent the \(N\) normal random variables, then the associated standard deviation square is given by

\begin{equation}
    \sigma^{2} = \frac{1}{N} \sum_{i=1}^{N} \left( X_{i} - \bar{X} \right)^{2} \text{ ,}
    \label{equation:chi2_dist_build_from_normals}
\end{equation}
where \(\bar{X}\) is the sample mean of the normal distribution.
In our analysis codes the standard error differs slightly from \cref{equation:chi2_dist_build_from_normals} and is defined as:

\begin{equation}
    (SE)^{2} = \frac{1}{N-1} \sum_{i=1}^{N} \left( X_{i} - \bar{X} \right)^{2} \text{ .}
    \label{equation:omc3_standard_error}
\end{equation}

The ensemble of \(\sigma^{2}\) values from different sets of measurements form a chi-square distribution, such as those that can be seen in \cref{figure:square_errors_histograms}. 
The associated Probability Density Function (PDF) for \(k\) degrees of freedom, which is non-zero for positive values only, is given by:

\begin{equation}
    P(x \geq 0, k) = \frac{x^{k/2-1} e^{-x/2}} {2^{k/2} \Gamma(\frac{k}{2})} \text{ .}
    \label{equation:chi2_pdf_equation}
\end{equation}

The \concept{mode} is the value \(x^{*}\) that maximizes the PDF.
In this case, one can notice that for positive values where the PDF is defined, \(x^{*}\) will be the same for \(P(x)\) and \(\log P(x)\).
When adopting the following convention:

\begin{equation}
    \begin{aligned}
        A(x) &= \log P(x)  \text{ ,} \\
        C    &= - \frac{k}{2} \log(2) - \log \Gamma(k/2)  \text{ ,}
    \end{aligned}
    \label{equation:chi2_pdf_derivation_convention}
\end{equation}
where \(C\) is a constant of the given distribution, one can deduce the value of the mode \(x^{*}\) through a derivation:

\begin{equation}
	\begin{aligned}
        A(x)              &= C + \left( \frac{k}{2} - 1 \right) \log(x) - \frac{x}{2} \text{ ,} \\
        \frac{d A(x)}{dx} &= \left( \frac{k}{2} - 1 \right) \frac{1}{x} - \frac{1}{2} = \frac{k - 2 - x}{2x} \text{ .}
	\end{aligned}	
    \label{equation:chi2_pdf_mode_proof}
\end{equation}

Thus we find that the mode is \(x^{*} = k - 2\).
For \(k \leq 2\) the mode is 0 since the PDF in that case is strictly decreasing with \(x\).
\Cref{figure:chisquare_demo_fit} illustrates this property with a generated chi-square distribution and its PDF, highlighting the determined location of the mode from a distribution fit.
\newline

\begin{figure}[!htb]
    \centering
    \includegraphics*[width=0.95\textwidth]{Figures/Other_Studies/phase_chisquare_demo_fit.pdf}
    \caption{A randomly generated chi-square distribution (\textcolor{mplblue}{blue}) with \(k = 4\) degrees of freedom, and a numerically fit probability density function (\textcolor{mplorange}{orange}). The determined mode (\textcolor{mplred}{red}) is indeed located at \(k - 2 = 2\). Here 'df' (degrees of freedom), 'loc' (horizontal offset) and 'scale' (a scaling factor) are parameters determined during the fit. The horizontal axis is in units of \(\sigma^2\).}
    \label{figure:chisquare_demo_fit}
\end{figure}

The standard deviation of the phase advances \(\sigma_{\mu}\) is then given from the mode by:

\begin{equation}
    \sigma_{\mu} = \sqrt{\frac{(\epsilon^2)^*}{k - 2}} = \sqrt{\frac{(\epsilon^2)^*}{N - 3}} \text{ .}
    \label{equation:stdev_from_chi2_mode}
\end{equation}

For the chi distribution mentioned above and derived through \(x = y^2\), the associated probability density function of \cref{equation:chi2_pdf_equation} becomes:

\begin{equation}
    \begin{aligned}
        P_{\sigma}(y , k) &= P (y^2 , k) 2y \text{ ,} \\
        P_{\sigma}(y , k) &= \frac{y^{k-1} e^{-y^2/2}} {2^{k/2-1} \Gamma(\frac{k}{2})} \text{ ,}
    \end{aligned}
    \label{equation:chi_pdf}  
\end{equation}
and the mode of \(P_{\sigma}(y , k)\) is located at \(y^*=\sqrt{k-1}\).
The determination of the standard deviation of \cref{equation:stdev_from_chi2_mode} above becomes:

\begin{equation}
    \sigma_{\mu} = \frac{y^*}{\sqrt{k - 1}} = \frac{y^*}{\sqrt{N - 2}} \text{ .}
    \label{equation:stdev_from_chi_mode}
\end{equation}

However, for non-perfect distributions it is numerically difficult to accurately detect this mode: one can use the mode of a fitted probability density function (as done in \cref{figure:chisquare_demo_fit} on a perfect chi-square distribution) but suffer from the quality of the fit, or try to detect the highest bins from the distribution's histogram but suffer from bin resolution and outlier data points skewing the result.

It is possible to use other properties than the mode.
For instance, similarly shown as previously for the mode, one can compute back the desired standard deviation using the mean of the chi distribution through:

\begin{equation}
	\begin{aligned}
        \sigma_{\mu} &= \frac{\mu}{T} \text{ ,} \\
        T            &= \sqrt{2} \frac{\Gamma((k + 1) / 2)}{\Gamma(k / 2)} \text{ ,}
	\end{aligned}	
    \label{equation:stdev_from_chi_mean}
\end{equation}
where \(\sigma_{\mu}\) is the phase measurement's standard deviation, \(\mu\) is the chi distribution's mean, \(k\) is the chi distribution's degrees of freedom and \(\Gamma\) is the Gamma function.
\newline

To summarize so far, the desired standard deviation of the phase advance measurements is not available for specific subgroups of measuring BPMs.
It is however possible to compute it back from either the mode of the chi-square distribution of the phase advance errors squared, or from the mode or mean of the chi distribution of the phase advance errors, through the following steps:
\begin{enumerate}
    \item Perform optics analysis on a set of measurements and select part of the data corresponding to the BPMs of interest.
    \item Form either the chi-square or chi distribution of, respectively, their phase advance errors squared or their phase advance errors.
    \item Compute the relevant property from said distribution - mode or mean, depending on the distribution - and determine the standard deviation of the phase advance measurements from that quantity.
\end{enumerate}
This way, one can determine the standard deviation on the phase advance reconstruction for a given arbitrary subgroup of BPMs from the optics analysis of a set of measurements.

\subsection*{Application to Measurements}

Due to the presence of many outliers (see \cref{figure:bpms_betas_histogram}) in measurement data and the lack of a sufficient number of data points in some BPM type categories, in the following the calculation from either the chi-square (\cref{equation:stdev_from_chi2_mode}) or chi (\cref{equation:stdev_from_chi_mean}) distribution was used, depending on the specific BPM subgroup.

Optics analysis was run on turn-by-turn data, BPMs were categorized based on their types~\cite{CERN:Equipment_Codes} and the value of \(\beta\)-functions at their locations, and both the chi distributions and the chi-square distributions were generated as detailed above.
\Cref{figure:grid_distributions_standard_standard} shows the distributions of phase errors for different \(\beta\)-functions combinations among so-called \textit{standard} BPMs, the most common ones in the LHC.

\begin{figure}[!htb]
    \centering
    \includegraphics*[width=\textwidth]{Figures/Other_Studies/phase_grid_distributions_errors_standard_standard.pdf}
    \caption{Phase error distributions for different BPM \(\beta\) combinations, for \textit{standard} to \textit{standard} type BPMs.}
    \label{figure:grid_distributions_standard_standard}
\end{figure}

For each of these cases, the standard deviation on phase advance is computed using either \cref{equation:stdev_from_chi2_mode} or \cref{equation:stdev_from_chi_mean}.
Depending on parameters set for the optics analysis one might have to apply a correcting factor of \(\sqrt{N-1}\) - with \(N\) being the number of measurements used for the analysis - to compensate for this factor being already present in calculations performed by the analysis software.
Namely, choosing to ignore \(t\)-value correction and single file uncertainty makes this factor appear in the optics calculations and requires to leave it out in later on analysis.
Values inferred with this method for an LHC Run~\num{2} for \textit{standard} type BPMs can be seen in \cref{figure:phase_error_heatmap_standard_bpms}, and for \textit{warm} type BPMs in \cref{figure:phase_error_heatmap_warm_bpms}.
Results were computed for all BPM types but are not all shown.

\begin{figure}[!htb]
    \centering
    \includegraphics*[width=0.95\textwidth]{Figures/Other_Studies/phase_stdev_heatmap_mrad_standard_standard.pdf}
    \caption{Computed standard deviations on phase advances between \textit{standard} to \textit{standard} type BPMs for the LHC Run~\num{2} (\num{2018}, \(\beta^{\ast} =\)~\qty{30}{\centi\meter}) for different \(\beta\)-functions combinations of these BPMs.}
    \label{figure:phase_error_heatmap_standard_bpms}
\end{figure}

\begin{figure}[!htb]
    \centering
    \includegraphics*[width=0.95\textwidth]{Figures/Other_Studies/phase_stdev_heatmap_mrad_warm_warm.pdf}
    \caption{Computed standard deviations on phase advances between \textit{warm} to \textit{warm} type BPMs for the LHC Run~\num{2} (\num{2018}, \(\beta^{\ast} =\)~\qty{30}{\centi\meter}) for different \(\beta\)-functions combinations of these BPMs. Empty slots correspond to \(\beta\)-functions ranges where no BPMs of this type are present.}
    \label{figure:phase_error_heatmap_warm_bpms}
\end{figure}

Obtained results are in line with expectations: BPM types with lower resolution such as \textit{warm} or \textit{wide}-aperture BPMs consistently yield a higher standard deviation on phase advances than for instance standard BPMs.
Similarly, phase advances between measuring BPMs placed at higher \(\beta\)-functions offer better precision.

\subsection*{Conclusions}

Assuming the normal nature of BPM measurements, one can make use of the underlying properties of the chi-square and chi statistical distributions in order to compute the standard deviation of phase advances while also differentiating between categories of BPMs, without performing measurements involving only BPMs of interest.
The dependency of said standard deviation on both measuring BPMs' types and their \(\beta\)-functions was confirmed.
The ability to accurately compute key statistical values on phase advance measurements - but also potentially a number of optics values later on computed from phase - while differentiating on arbitrary BPM categories opens up potential applications in analysis algorithms, for instance refining the phase accuracy in harmonic analysis or optimizing BPM selection in the N-BPM method.

\section{Simulations of Sextupolar Amplitude Detuning in the LHC}
\label{section:sextupolar_amplitude_detuning}

As mentioned in \cref{subsection:amplitude_detuning}, a measurement of the machine non-linearities can be obtained by characterizing the \gls{ampdet}: the dependency of the tune as a function of the amplitude of particle oscillations.
Amplitude detuning traditionally comes from octupolar fields and is used for beam stabilization against collective effects~\cite{BOOK:Chao:Collective_instabilities,PRAB:Karpov:Thresholds_Loss_Landau_Damping}.
More generally, it comes from multipolar fields of even order \(m = 2n\), where \(n\) is a natural number strictly higher than one since quadrupolar fields provide a linear behavior.

Analytical formulae from \cite{REPORT:Bengtsson:Smear_Tune_Shift, BOOK:Lee:Accelerator_physics,BOOK:Chao:Handbook_Accelerator_Physics_Engineering} show that second order terms of sextupole strengths contribute to free amplitude detuning.
It has previously been suspected that the forced amplitude detuning induced by sextupolar fields would behave similarly than that already studied for octupolar fields in~\cite{PRAB:White:Direct_Amplitude_Detuning_AC_Dipole}.

This study aims to first demonstrate the possibility of observing this amplitude detuning with \gls{AC_dipole} excitation as previous works have shown that not all higher order effects can be measured with driven oscillations~\cite{PRAB:Persson:Suppression_Amplitude_Dependent_Closest_Tune_Approach}, and of confirming the similarity of its behavior to that of the octupolar one.
Simulations were performed to investigate the sextupolar amplitude detuning in the \gls{LHC} lattice, both through forced oscillations with AC dipole and free oscillations up to similarly high amplitudes.

\subsection*{Amplitude Detuning with an AC dipole}

The feasibility of direct amplitude detuning measurement with an AC dipole was demonstrated in~\cite{PRAB:White:Direct_Amplitude_Detuning_AC_Dipole}, as well as its benefits compared to the previously traditional technique of exciting the beam to large amplitudes with a single kick.
In the same paper, the authors go over the linear motion of particles and differentiate two cases.
The parametrization of the transverse coordinate \(z\) in the case of free and forced oscillations with an AC dipole is given by \cref{equation:free_transverse_motion} and \cref{equation:driven_transverse_motion}, respectively.

Analysis of the \gls{beta-function} of the driven particle, \(\beta_D\), can be found in~\cite{PHD:Miyamoto,PRAB:Miyamoto:Parametrization_Driven_Betatron_Oscillation,PRAB:White:Direct_Amplitude_Detuning_AC_Dipole}.
It is derived, still in~\cite{PRAB:White:Direct_Amplitude_Detuning_AC_Dipole}, that for multipoles of even order \(m = 2n\) the amplitude detuning in the case of horizontally driven oscillations is given by

\begin{equation}
    \begin{aligned}
        \Delta Q_x &= \frac{q B_{2n}}{2np} \frac{2^{-n}}{2 \pi} \frac{(2n)!}{(n-1)! (n-1)!} \beta_x \beta_{D,x}^{n-1} A^{n-1} \text{ ,} \\
        \Delta Q_y &= -\frac{q B_{2n}}{2np} \frac{2^{-n}}{2 \pi} \frac{(2n)!}{(n-1)! (n-1)!} \beta_y \beta_{D,x}^{n-1} A^{n-1} \text{ ,}
    \end{aligned}
    \label{eqn:transverse_amp_det}
\end{equation}
where \(q\) is the charge of the particle, \(p\) its momentum and \(B_{2n}\) the gradient of the multipole of order \(2n\) providing the detuning.
Importantly, it is analytically shown that, neglecting the typically small difference between \(\beta_z\) and \(\beta_{D,z}\), \textit{"for multipoles of order \(2n\), the direct term of the amplitude detuning measured with an AC dipole will be a factor \(n\) larger than for free oscillations while the cross terms in both cases are equal"}.

This means in the case of octupolar amplitude detuning that the direct term detuning \(Q_{x}(J_{x})\) under AC dipole oscillations is a factor \(n = 2\) stronger than under free oscillations, while the cross term detuning \(Q_{y}(J_{x})\) will be the same in both cases.
It was suspected that in the LHC sextupolar detuning can be measured with AC dipoles and that it behaves similarly with this factor \num{2} observed in the case of even order multipole, while being a multipole of order \num{3} itself.
Tracking simulations were performed to investigate this effect.

\subsection*{Amplitude Detuning via Particle Tracking}

On-momentum particles were tracked with the \gls{MADX} code through the LHC lattice of \num{2018}, with \(\beta^{\ast} =\)~\qty{30}{\centi\meter} optics.
In the setup, non-linearities were stripped down to the only contribution of sextupoles: no magnet errors, crossing angles or orbit bumps were included; and neither octupoles nor higher order magnets were powered.
This way the only non-negligible contribution to amplitude detuning is the sextupolar one.

Tracking was performed under in the case of both free and AC dipole forced oscillations (in the horizontal plane) with similar amplitudes.
Several scenarios corresponding to different tune separations for the driven motion were explored, namely \(\Delta Q_x =\)~\num{-1e-2} and \(\Delta Q_x =\)~\num{-5e-3}, resulting in driven fractional horizontal tunes of \(Q_{Dx} =\)~\num{0.30} and \(Q_{Dx} =\)~\num{0.305}, respectively
A small initial amplitude was applied in the vertical plane in order to enable measuring the vertical tune.

The AC dipole ramp-up and flattop times were chosen as used in optics measurements in the LHC, to preserve the adiabaticity of the process.
Turn-by-turn data was gathered at all BPMs during the tracking, and the resulting file was analyzed as done for LHC measurements.
Both the natural tunes and actions were derived from harmonic and then optics analysis of the data.

\Cref{figure:horizontal_detuning_comparison,figure:vertical_detuning_comparison} show the horizontal and vertical amplitude detuning results for both free and driven motion, respectively.
For each result a linear regression is performed against the data to determine the detuning coefficient.
In the case of simulations, where the uncertainty on the data points is negligible, this fit is enough to accurately determine the detuning coefficient.
Results are compiled in \cref{table:detuning_coefficients_results}.

\begin{figure}[!htb]
    \centering
    \includegraphics*[width=\textwidth]{Figures/Other_Studies/direct_detuning_driven_vs_free.pdf}
    \caption{Natural horizontal tune \(Q_x\) shift with horizontal free or forced action (\(J_x\) or \(A_x\)), in the case of free oscillations (\textcolor{mplorange}{orange}) and driven motion with \(\Delta Q_x = -0.01\) (\textcolor{mplblue}{blue}) and \(\Delta Q_x = -0.005\) (\textcolor{mplred}{red}).}
    \label{figure:horizontal_detuning_comparison}
\end{figure}

For the direct detuning (\cref{figure:horizontal_detuning_comparison}) where a factor \num{2} is expected - the fit to free oscillations data gives a detuning coefficient of \qty{13.14}{\micro\per\meter}, which makes the expected detuning coefficient for the forced oscillations around \qty{26.3}{\micro\per\meter}.
With a tune separation of \(\Delta Q_x =\)~\num{-1e-2} this coefficient is determined at \qty{24.04}{\micro\per\meter}, within a \qty{10}{\percent} margin of the expected value.
Halving the tune separation to \(\Delta Q_x =\)~\num{-5e-3} brings this coefficient closer to the factor \num{2}, now within a \qty{5}{\percent} margin.
This change is attributed to the modification of the \(\beta\)-function and phase advances by the AC dipole: the \(\beta_{D,z}\) term in \cref{equation:driven_transverse_motion}, which is assumed close enough to \(\beta_z\) in the factor \num{2} hypothesis.
As the relative difference between \(\beta_{D,z}\) and \(\beta_z\) reduces with the tune separation, this behavior is expected.

\begin{figure}[!htb]
    \centering
    \includegraphics*[width=\textwidth]{Figures/Other_Studies/cross_detuning_driven_vs_free.pdf}
    \caption{Natural vertical tune \(Q_y\) shift with horizontal free or forced action (\(J_x\) or \(A_x\)), in the case of free oscillations (\textcolor{mplorange}{orange}) and driven motion with \(\Delta Q_x = -0.01\) (\textcolor{mplblue}{blue}) and \(\Delta Q_x = -0.005\) (\textcolor{mplred}{red}).}
    \label{figure:vertical_detuning_comparison}
\end{figure}

For the cross detuning (\cref{figure:vertical_detuning_comparison}) where a factor \num{1} is expected, similar results are observed: the driven motion's detuning coefficient is determined at \qty{-15.99}{\micro\per\meter}, reasonably close to the \qty{-14.94}{\micro\per\meter} in the case of free oscillations.
Again, halving the tune separation to \(\Delta Q_x =\)~\num{-5e-3} brings us closer to the factor \num{1}.

\begin{table}[!htb]
    \centering
    \begin{tblr}{colspec={ccc}}
        \hline
        \SetCell[r=2,c=1]{m,c} \textbf{Scenario}   &  \SetCell[c=2]{c} Detuning Coefficient [\unit{\per\micro\meter}]   \\
        \cline{2,3}                                &  Direct Term    &  Cross Term                                      \\
        \hline
        Free Oscillations                           &  \num{13.14}   &  \num{-14.94}                                    \\
        Driven (\(\Delta Q_x = -0.01\))             &  \num{24.04}   &  \num{-15.99}                                    \\
        Driven (\(\Delta Q_x = -0.005\))            &  \num{25.01}   &  \num{-15.50}                                    \\
        \hline
    \end{tblr}
    \caption{Direct and cross term detuning coefficients for free and forced motion, determined the linear regression performing on tracking data.}
    \label{table:detuning_coefficients_results}
\end{table}

\subsection*{Conclusions}

Measurement of strictly the sextupolar amplitude detuning is not feasible in the LHC as other sources would entangle with the sextupolar contribution.
Particle tracking simulations have confirmed the ability to measure sextupolar amplitude detuning with an AC dipole and a sextupolar contribution to amplitude detuning in the LHC models.
Additionally, with small enough tune separation this contribution behaves similarly to the octupolar amplitude detuning one when comparing the case of forced motion and free oscillations.

\section{Summary}

A statistical analysis of the phase reconstruction precision in the \gls{LHC} was performed, while differentiating between \gls{BPM} categories.
The results confirmed the influence of both BPM type and the \glspl{beta-function} at the measuring device on the precision of the phase reconstruction, with up to a factor \num{3} between the best and worst performing BPMs.
This knowledge could be applied in optics measurements algorithms to, for instance, select a subset of BPMs to use for a given measurement based its requirements.

A second simulation-based study was performed to investigate the sextupolar contribution to the \gls{ampdet} in the LHC, which cannot not be directly observed in the machine due to being entangled with other contributions.
A clear contribution, though relatively small compared to that of other sources, was observed.
The suspected behavior of this contribution, similar to that of the octupolar amplitude detuning, was confirmed.

Substantial effort put into the maintenance and development of the software suite used by the \acrshort{OMC} team, both for analysis and simulation codes, was presented.

\glsresetall                                     % reset glossary entries counts for the next chapter
