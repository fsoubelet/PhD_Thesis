\chapter{Other Optics Studies and Software Developments}
\label{chapter:Others_and_Software}

\todo{Write a short intro here. Do not format the body before having this written.}

\section{Phase Error Dependency on BPM Type and Location in the LHC}

In the LHC and many other circular colliders BPM measurement data constitutes the primary source for optics properties computation.
These measurements and the subsequent results were suspected to be substantially affected by both the type of BPM used for the measurements and the value of the \(\beta\)-functions are the measuring device.
A study was performed to investigate the influence of these properties on the reconstructed phase value from turn-by-turn data.

In practice the presence of systematic errors can influence different sets of measurements to not be totally independent, but in the following study these are assumed to be independent and identically distributed normal random variables (IID), meaning repeating the same BPM measurement in identical conditions will spawn a range of values forming a Gaussian distribution.
Using the underlying properties of the measurement derived distributions, one can numerically infer the standard deviation on the phase measurement.

\subsection{Measurement Data and Underlying Distributions}

The methods and considerations below can be applied on BPM turn-by-turn measurement data from synchrotron machines.
The following analysis was done on data taken during the LHC Run~\num{2}, in \num{2018} and at \(\beta^{\ast} =\)~\qty{30}{\centi\meter}.

The suspicions mentioned above come from the fact that different BPM types have different resolution and higher \(\beta\)-functions lead to higher signal-to-noise ratios on measurements.
As can be seen in \cref{figure:bpms_betas_histogram}, measuring BPMs are spread across a wide range of \(\beta\)-functions for the Run~\num{2} configuration with \(\beta^{\ast} =\)~\qty{30}{\centi\meter}.
It is therefore of interest to have the ability to determine the standard deviation on the phase measurement while differentiating on BPM types and \(\beta\)-functions.

\begin{figure}[!htb]
    \centering
    \includegraphics*[width=\textwidth]{Figures/Other_Studies/bpms_betas_histogram.pdf}
    \caption{Distribution of BPM \(\beta\)-functions across the machine for the Run~\num{2} measurements used. A small amount of BPMs located at extremely high \(\beta\) are not shown on this plot.}
    \label{figure:bpms_betas_histogram}
\end{figure}

The sum of two IIDs - which our measurements are assumed to be - is also normal.
Using the characteristic function of a normal distribution:

\begin{equation}
    \varphi(t) = \exp \left( it \mu - \frac{\sigma^2 t^2}{2} \right) \text{ ,}
    \label{equations:normal_distribution_characteristic_function}
\end{equation}
and given that the characteristic function of the sum of two independent random variables \(X\) and \(Y\) is the product of their respective characteristic functions, one gets:

\begin{equation}
    \begin{aligned}
        \varphi_{X+Y}(t) = \varphi_{X}(t) \varphi_{Y}(t) &= \exp \left( it \mu_{X} - \frac{\sigma_{X}^2 t^2}{2} \right) \exp \left(it \mu_{Y} - \frac{\sigma_{Y}^2 t^2}{2} \right)  \text{ ,} \\
                                                         &= \exp \left( it \left( \mu_{X} + \mu_{Y} \right) - \frac{\left( \sigma_{X}^2 + \sigma_{Y}^2 \right) t^2}{2} \right)  \text{ ,}
    \end{aligned}
\end{equation}
which corresponds to a normal distribution with its mean being the sum of the two means, and its variance being the sum of the two variances.
Respectively, one can deduce the same for the subtraction of two IIDs by changing the signs.

In particle accelerators, phase advances are measured from BPM to BPM and are, in the simplest form, the result of a subtraction.
The repeated phase advance measurements obtained from a given BPM pair form a normal distribution with average phase advance \(\bar{\varphi}\) and standard deviation \(\sigma_{\mu}\).
Traditionally, the error on phase advance is computed via the standard deviation of \(N\) measurements:

\begin{equation}
    \varepsilon^2 = \frac{1}{N} \sum_{i=1}^{N}\big(\varphi_{i} - \bar{\varphi} \big)^2 \text{ .}
    \label{equation:phase_error_calculation}
\end{equation}

This means the squares of phase advance (and phase advance error) values (\(\varepsilon^2\)) form a chi-square distribution, with a number of degrees of freedom  \(k = N - 1\) since the sample mean is subtracted.
Therefore, grouping sufficient BPM pairs of the same type and with similar \(\beta\)-function and computing the distribution of their \(\varepsilon^2\) one obtains a Chi-square distribution.
\Cref{figure:square_errors_histograms} shows those chi-square distributions for different ranges of \(\beta\)-functions combinations between measuring BPMs.

\begin{figure}[!htb]
    \centering
    \includegraphics*[width=\textwidth]{Figures/Other_Studies/phase_errors_squared_distributions.pdf}
    \caption{Distribution of the squares of phase measurement errors for different BPM combinations, differentiated by the \(\beta\)-functions at which the BPMs are located. The different distributions form chi-square distributions. For this plot a few simple categories were established, and a BPM is considered "low" below \(\beta =\)~\qty{100}{\meter}, "high" above \(\beta =\)~\qty{200}{\meter} and "medium" in between.}
    \label{figure:square_errors_histograms}
\end{figure}

Similarly, the positive real square roots of values from this chi-square distribution form a chi distribution, which can be derived with a change of variable \(x=y^2\).
\Cref{figure:phase_errors_histograms} shows these distributions for the same \(\beta\)-function ranges seen in \cref{figure:square_errors_histograms}.

\begin{figure}[!htb]
    \centering
    \includegraphics*[width=\textwidth]{Figures/Other_Studies/phase_errors_distributions.pdf}
    \caption{Distribution of the phase measurement errors for different BPM combinations, differentiated by the \(\beta\)-functions at which the BPMs are located. For this plot a few simple categories were established, and a BPM is considered "low" below \(\beta =\)~\qty{100}{\meter}, "high" above \(\beta =\)~\qty{200}{\meter} and "medium" in between.}
    \label{figure:phase_errors_histograms}
\end{figure}

\subsection{Computing the Standard Deviation on Phase Advance Measurements}

\section{Simulations of Sextupolar Contribution to Amplitude Detuning in the LHC}

\todo{Write / include a small section on amplitude detuning in the theory chapter.}
\cite{PRAB:White:Direct_Amplitude_Detuning_AC_Dipole}

Blah blah.

\section{OMC Software Developments}

Blah blah.

\section{Summary}

Blah blah.
