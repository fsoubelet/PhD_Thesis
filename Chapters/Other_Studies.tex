\chapter{Optics Studies and Software Developments}
\label{chapter:Others_and_Software}

Two further studies performed during the course of this thesis are presented in this chapter.
The first one is a statistical analysis of the phase measurement error in the LHC, that aimed to determine the influence of the measuring BPMs on the phase error.
The second one is an investigation on the sextupolar contribution to amplitude detuning in the LHC.

In parallel to other works presented in this document, a substantial effort was put into software developments to improve both analysis and simulation tools.
The relevant contributions and achievements over the course of this work are succinctly presented in the last section of this chapter.

\section{Phase Error Dependency on BPM Type and Location}

In the LHC and many other circular colliders BPM measurement data constitutes the primary source for optics properties computation.
As different BPM types~\cite{CERN:Equipment_Codes} have different measurement resolution, and higher \(\beta\)-functions lead to better signal-to-noise ratios on measurements \todo{Ref?}, these measurements and the subsequent results were suspected to be substantially affected by both the type of BPM used for the measurements and the value of the \(\beta\)-functions at the measuring devices.
This study was performed to investigate the influence of these properties on the precision of the reconstructed phase values from turn-by-turn data.

In practice the presence of systematic errors can influence different sets of measurements to not be totally independent.
In the following study measurements are assumed to be independent and identically distributed normal random variables (IIDs), meaning repeating the same BPM measurement in identical conditions will spawn a range of values forming a Gaussian distribution.
As no sets of measurements were available that made use of specific groups of BPMs, a statistical analysis on the phase errors from previous measurements was needed to differentiate between BPMs of interest. 
Using the underlying properties of the measurements-derived distributions, one can numerically infer the impact on the phase measurement error of specific subgroups of measuring devices.

\subsection*{Measurement Data and Underlying Distributions}

The methods and considerations below can be applied on BPM turn-by-turn measurement data from synchrotron machines.
The following analysis was done on data taken during the LHC Run~\num{2}, in \num{2018} and at \(\beta^{\ast} =\)~\qty{30}{\centi\meter}.
Information about the measurement data used for this analysis can be found in \cref{appendix:measurement_fills}.
\Cref{figure:bpms_betas_histogram} shows BPMs in the LHC are spread across a wide range of \(\beta\)-functions.

\begin{figure}[!htb]
    \centering
    \includegraphics*[width=\textwidth]{Figures/Other_Studies/bpms_betas_histogram.pdf}
    \caption{Distribution of BPM \(\beta\)-functions across the machine for the Run~\num{2} measurements used. A small amount of BPMs located at much higher \(\beta\) are not shown on this plot.}
    \label{figure:bpms_betas_histogram}
\end{figure}

The sum of two IIDs - which our measurements are assumed to be - is also normal.
Using the characteristic function of a normal distribution

\begin{equation}
    \varphi(t) = \exp \left( it \mu - \frac{\sigma^2 t^2}{2} \right) \text{ ,}
    \label{equations:normal_distribution_characteristic_function}
\end{equation}
and given that the characteristic function of the sum of two independent random variables \(X\) and \(Y\) is the product of their respective characteristic functions, one gets:

\begin{equation}
    \begin{aligned}
        \varphi_{X+Y}(t) = \varphi_{X}(t) \varphi_{Y}(t) &= \exp \left( it \mu_{X} - \frac{\sigma_{X}^2 t^2}{2} \right) \exp \left(it \mu_{Y} - \frac{\sigma_{Y}^2 t^2}{2} \right)  \text{ ,} \\
                                                         &= \exp \left( it \left( \mu_{X} + \mu_{Y} \right) - \frac{\left( \sigma_{X}^2 + \sigma_{Y}^2 \right) t^2}{2} \right)  \text{ ,}
    \end{aligned}
\end{equation}
which corresponds to a normal distribution with its mean being the sum of the two means, and its variance being the sum of the two variances.
Respectively, one can deduce the same for the subtraction of two IIDs by changing the signs.

In particle accelerators, phase advances are measured from BPM to BPM and are, in the simplest form, the result of a subtraction.
As a consequence the repeated phase advance measurements obtained from a given BPM pair form a normal distribution with average phase advance \(\bar{\varphi}\) and standard deviation \(\sigma_{\mu}\).
Traditionally, the error on phase advance is computed via the standard deviation of \(N\) measurements:

\begin{equation}
    \varepsilon^2 = \frac{1}{N} \sum_{i=1}^{N}\big(\varphi_{i} - \bar{\varphi} \big)^2 \text{ .}
    \label{equation:phase_error_calculation}
\end{equation}

This means the squares of phase advance (and phase advance error) values \(\varepsilon^2\) form a chi-square distribution, with a number of degrees of freedom  \(k = N - 1\) since the sample mean is subtracted.
Therefore, grouping sufficient BPM pairs of the same type and with similar \(\beta\)-function, and computing the distribution of their \(\varepsilon^2\) one obtains a chi-square distribution.
\Cref{figure:square_errors_histograms} shows those chi-square distributions for different ranges of \(\beta\)-functions combinations between measuring BPMs.

\begin{figure}[!htb]
    \centering
    \includegraphics*[width=\textwidth]{Figures/Other_Studies/phase_errors_squared_distributions.pdf}
    \caption{Distribution of the squares of phase measurement errors for different BPM combinations, differentiated by the \(\beta\)-functions at the locations of the measuring BPMs. The different distributions form chi-square distributions. For this plot a few simple categories were established, and a BPM is considered "low" below \(\beta =\)~\qty{100}{\meter}, "high" above \(\beta =\)~\qty{200}{\meter} and "medium" in between.}
    \label{figure:square_errors_histograms}
\end{figure}

Similarly, the positive real square roots of values from this chi-square distribution form a chi distribution, which can be derived with a change of variable \(x=y^2\).
\Cref{figure:phase_errors_histograms} shows these distributions for the same \(\beta\)-function ranges seen in \cref{figure:square_errors_histograms}.

\begin{figure}[!htb]
    \centering
    \includegraphics*[width=\textwidth]{Figures/Other_Studies/phase_errors_distributions.pdf}
    \caption{Distribution of the phase measurement errors for different BPM combinations, differentiated by the \(\beta\)-functions at the locations of the measuring BPMs. For this plot a few simple categories were established, and a BPM is considered "low" below \(\beta =\)~\qty{100}{\meter}, "high" above \(\beta =\)~\qty{200}{\meter} and "medium" in between.}
    \label{figure:phase_errors_histograms}
\end{figure}

\subsection*{Computing the Standard Deviation on Phase Advance Measurements}

A chi-square distribution with \(k\) degrees of freedom  is the distribution of sum of the squares of \(k\) independent standard normal random variables.
Let \(X_{i}\) represent the \(N\) normal random variables, then the associated standard deviation square is given by:

\begin{equation}
    \sigma^{2} = \frac{1}{N} \sum_{i=1}^{N} \left( X_{i} - \bar{X} \right)^{2} \text{ ,}
    \label{equation:chi2_dist_build_from_normals}
\end{equation}
where \(\bar{X}\) is the sample mean of the normal distribution given by

\begin{equation}
    \bar{X} = \frac{1}{N} \sum_{i=1}^{N} X_{i} \text{ .}
    \label{equation:normal_mean}
\end{equation}

The standard error used in analysis codes is defined as

\begin{equation}
    (SE)^{2} = \frac{1}{N-1} \sum_{i=1}^{N} \left( X_{i} - \bar{X} \right)^{2} \text{ .}
    \label{equation:omc3_standard_error}
\end{equation}

The ensemble of \(\sigma^{2}\) values from different sets of measurements form a chi-square distribution. 
The associated Probability Density Function (PDF) for \(k\) degrees of freedom, which is non-zero for positive values only, is given by:

\begin{equation}
    P(x \geq 0, k) = \frac{x^{k/2-1} e^{-x/2}} {2^{k/2} \Gamma(\frac{k}{2})} \text{ .}
    \label{equation:chi2_pdf_equation}
\end{equation}

The \intro{mode} is the value \(x^{*}\) that maximizes the PDF.
In this case, one can notice that for positive values where the PDF is defined, \(x^{*}\) will be the same for \(P(x)\) and \(\log P(x)\).
When adopting the following convention:

\begin{equation}
    \centering
    A(x) = \log P(x) \mbox{,   and  } C =
    \tikz[baseline]{
        \node[draw=red,rounded corners,anchor=base] (m5)
        {\(\displaystyle - \frac{k}{2} \log(2) - \log \Gamma(k/2)\)};
        \node[below of=m5] (12) {constant};
        \draw[-,red] (12) -- (m5)
    } \text{ ,}
    \label{equation:chi2_pdf_derivation_convention}
\end{equation}
one can deduce the value \(x^{*}\) through a simple derivation:

\begin{equation}
	\begin{aligned}
    A(x)              &= C + \left( \frac{k}{2} - 1 \right) \log(x) - \frac{x}{2} \text{ ,} \\
    \frac{d A(x)}{dx} &= \left( \frac{k}{2} - 1 \right) \frac{1}{x} - \frac{1}{2} = \frac{k - 2 - x}{2x} \text{ .}
	\end{aligned}	
    \label{equation:chi2_pdf_mode_proof}
\end{equation}

Thus, we get that the mode is \(x^{*} = k - 2\).
If \(k \leq 2\) then the mode is 0, since the PDF in that case is strictly decreasing with \(x\).
\Cref{figure:chisquare_demo_fit} illustrates this property with a generated chi-square distribution and its PDF, highlighting the determined location of the mode from a distribution fit.

\begin{figure}[!htb]
    \centering
    \includegraphics*[width=\textwidth]{Figures/Other_Studies/phase_chisquare_demo_fit.pdf}
    \caption{A randomly generated chi-square distribution (\textcolor{mplblue}{blue}) with \(k = 4\) degrees of freedom, and a numerically fit probability density function (\textcolor{mplorange}{orange}). The determined mode (\textcolor{mplred}{red}) is indeed located at \(k - 2 = 2\). Here 'df' (degrees of freedom), 'loc' (horizontal offset) and 'scale' (a scaling factor) are parameters determined during the fit. The horizontal axis is in units of sigmas square.}
    \label{figure:chisquare_demo_fit}
\end{figure}

The standard deviation of the phase measurements \(\sigma_{\mu}\) is then given from the mode by:

\begin{equation}
    \sigma_{\mu} = \sqrt{\frac{(\epsilon^2)^*}{k - 2}} = \sqrt{\frac{(\epsilon^2)^*}{N - 3}} \text{ .}
    \label{equation:stdev_from_chi2_mode}
\end{equation}

For the chi distribution mentioned above and derived through \(x = y^2\), the associated probability density function of \cref{equation:chi2_pdf_equation} becomes:

\begin{equation}
    \begin{aligned}
        P_{\sigma}(y , k) &= P (y^2 , k) 2y \text{ ,} \\
        P_{\sigma}(y , k) &= \frac{y^{k-1} e^{-y^2/2}} {2^{k/2-1} \Gamma(\frac{k}{2})} \text{ .}
    \end{aligned}
    \label{equation:chi_pdf}  
\end{equation}

For this chi distribution, the mode of \(P_{\sigma}(y , k)\) is at \(y^*=\sqrt{k-1}\) and \cref{equation:stdev_from_chi2_mode} above becomes:

\begin{equation}
    \sigma_{\mu} = \frac{y^*}{\sqrt{k - 1}} = \frac{y^*}{\sqrt{N - 2}} \text{ .}
    \label{equation:stdev_from_mode}
\end{equation}

However, for non-perfect distributions it is numerically difficult to accurately detect this mode: one can use the mode of a fitted probability density function (as done in \cref{figure:chisquare_demo_fit} on a perfect chi-square distribution) but suffer from the quality of the fit, or try to detect the highest bins from the distribution's histogram but suffer from bin resolution and outlier data points skewing the result.

It is possible to use other properties than the mode.
For instance, similarly shown as previously for the mode, one can compute back the desired standard deviation using the mean of the chi distribution through:

\begin{equation}
	\begin{aligned}
        \sigma_{\mu} &= \frac{\mu}{T} \text{ ,} \\
        T            &= \sqrt{2} \frac{\Gamma((k + 1) / 2)}{\Gamma(k / 2)} \text{ ,}
	\end{aligned}	
    \label{equation:stdev_from_chi_mean}
\end{equation}
where \(\sigma_{\mu}\) is the phase measurement's standard deviation, \(\mu\) is the chi distribution's mean, \(k\) is the chi distribution's degrees of freedom and \(\Gamma\) is the Gamma function.

\subsection*{Application to Measurements}

Due to the presence of many outliers (see \cref{figure:bpms_betas_histogram}) and the lack of a sufficient number of data points in some BPM type categories, the calculation from the mean of the chi distribution seen in \cref{equation:stdev_from_chi_mean} was used in the results presented below.

Optics analysis was run on turn-by-turn data, and BPMs were categorized based on their types~\cite{CERN:Equipment_Codes} and the value of \(\beta\)-functions at their locations.
Both the chi distributions and the chi-square distributions are generated as detailed above, respectively from the phase advance standard errors and from the square of these errors.
\Cref{figure:grid_distributions_standard_standard} shows the distributions of phase errors for different \(\beta\)-functions combinations among so-called \textit{standard} BPMs, the most common ones in the LHC.

\begin{figure}[!htb]
    \centering
    \includegraphics*[width=\textwidth]{Figures/Other_Studies/phase_grid_distributions_errors_standard_standard.pdf}
    \caption{Phase error distributions for different BPM \(\beta\) combinations, for \textit{standard} to \textit{standard} type BPMs.}
    \label{figure:grid_distributions_standard_standard}
\end{figure}

For each of these cases, the mean of the distribution is computed and the calculation from \cref{equation:stdev_from_chi_mean} is used.
Depending on parameters set for the optics analysis one might have to apply a correcting factor of \(\sqrt{N-1}\) - with \(N\) being the number of measurements used for the analysis - to compensate for this factor being already present in calculations performed by the analysis software.
Namely, choosing to ignore \(t\)-value correction and single file uncertainty makes this factor appear in the optics calculations and requires to leave it out in later on analysis.
Values inferred with this method for an LHC Run~\num{2} for \textit{standard} type BPMs can be seen in \cref{figure:phase_error_heatmap_standard_bpms}.
Results for all BPM types were also computed but are not shown.% can be seen in Fig.~\ref{fig:stripline_error_heatmap} and Fig.~\ref{fig:warm_error_heatmap}.

Overall the results are in line with expectations: BPM types with lower resolution such as warm BPMs or wide-aperture BPMs consistently yield a higher standard deviation on phase errors than for instance standard BPMs.
Similarly, BPMs placed at higher \(\beta\)-functions offer better results.

\begin{figure}[!htb]
    \centering
    \includegraphics*[width=\textwidth]{Figures/Other_Studies/phase_stdev_heatmap_mrad_standard_standard.pdf}
    \caption{Computed standard deviation on phase advance measurements between \textit{standard} to \textit{standard} type BPMs for the LHC Run~\num{2} (\num{2018}, \(\beta^{\ast} =\)~\qty{30}{\centi\meter}) for different \(\beta\)-functions combinations of these BPMs.}
    \label{figure:phase_error_heatmap_standard_bpms}
\end{figure}

\subsection*{Conclusions}

Using the normal nature of BPM measurements, one can make use of the underlying properties of the chi-square and chi statistical distributions in order to compute the standard deviation of phase measurement while also differentiating between categories of BPMs.
The suspected dependency of said standard deviation on both measuring BPMs' types and their \(\beta\)-functions has been confirmed.
The ability to accurately compute key statistical values on phase advance measurements - but also potentially a number of optics values later on computed from phase - while differentiating on arbitrary BPM categories opens up potential applications in analysis algorithms, for instance refining the phase accuracy in harmonic analysis or optimizing BPM selection in the N-BPM method.

\section{Simulations of Sextupolar Contribution to Amplitude Detuning in the LHC}

\todo{Write / include a small section on amplitude detuning in the theory chapter.}

As mentioned in \cref{chapter:Theory}, a measurement of the machine nonlinearities can be obtained by characterizing the amplitude detuning: the dependency of the tune as a function of the amplitude of particle oscillations.
Amplitude detuning traditionally comes from octupolar fields and is used for beam stabilization against collective effects.
More generally, it comes from multipolar fields of even order \(m = 2n\), where \(n\) is a natural number strictly higher than one, since quadrupolar fields provide a linear behavior.

Analytical formulae from \cite{REPORT:Bengtsson:Smear_Tune_Shift, BOOK:Lee:Accelerator_physics,BOOK:Chao:Handbook_Accelerator_Physics_Engineering} show that second order terms of sextupole strengths contribute to free amplitude detuning.
It has previously been suspected that the forced amplitude detuning induced by sextupolar fields would behave similarly than that already studied for octupolar fields in~\cite{PRAB:White:Direct_Amplitude_Detuning_AC_Dipole}.
This study aims to first demonstrate the possibility of measuring this amplitude detuning with AC dipole excitation as previous works have shown that not all higher order effects can be measured with driven oscillations~\cite{PRAB:Persson:Suppression_Amplitude_Dependent_Closest_Tune_Approach}.
Simulations were performed to investigate the sextupolar amplitude detuning in the LHC lattice, both through forced oscillations with AC dipole and free oscillations up to similarly high amplitudes.

\subsection*{Amplitude Detuning with an AC Dipole}

The feasibility of direct amplitude detuning measurement with an AC dipole was demonstrated in~\cite{PRAB:White:Direct_Amplitude_Detuning_AC_Dipole}, as well as its benefits compared to the previously traditional technique of exciting the beam to large amplitudes with a single kick.
In the same paper, the authors go over the linear motion of particles and differentiate two cases.
The parametrization of the transverse coordinate \(z\) in the case of free and forced oscillations with an AC dipole is given by \cref{equation:free_transverse_motion} and \cref{equation:driven_transverse_motion}, respectively.

Analysis of the \(\beta\)-function of the driven particle, \(\beta_D\), can be found in~\cite{PHD:Miyamoto,PRAB:Miyamoto:Parametrization_Driven_Betatron_Oscillation,PRAB:White:Direct_Amplitude_Detuning_AC_Dipole}.
It is derived, still in~\cite{PRAB:White:Direct_Amplitude_Detuning_AC_Dipole}, that for multipoles of even order \(m = 2n\) the amplitude detuning in the case of horizontally driven oscillations is given by:

\begin{equation}
    \begin{aligned}
        \Delta Q_x &= \frac{q B_{2n}}{2np} \frac{2^{-n}}{2 \pi} \frac{(2n)!}{(n-1)! (n-1)!} \beta_x \beta_{D,x}^{n-1} A^{n-1} \text{ ,} \\
        \Delta Q_y &= -\frac{q B_{2n}}{2np} \frac{2^{-n}}{2 \pi} \frac{(2n)!}{(n-1)! (n-1)!} \beta_y \beta_{D,x}^{n-1} A^{n-1} \text{ ,}
    \end{aligned}
    \label{eqn:transverse_amp_det}
\end{equation}
where \(q\) is the charge of the particle, \(p\) its momentum and \(B_{2n}\) the gradient of the multipole of order \(2n\) providing the detuning.
Importantly, it is analytically shown that, neglecting the typically small difference between \(\beta_z\) and \(\beta_{D,z}\), \textit{"for multipoles of order \(2n\), the direct term of the amplitude detuning measured with an AC dipole will be a factor \(n\) larger than for free oscillations while the cross terms in both cases are equal"}.

This means in the case of octupolar amplitude detuning that the direct term detuning \(Q_{x}(J_{x})\) under AC dipole oscillations is a factor \(n = 2\) stronger than under free oscillations, while the cross term detuning \(Q_{y}(J_{x})\) will be the same in both cases.
It was suspected that in the LHC sextupolar detuning can be measured with ac dipoles and that it behaves similarly with this factor \num{2} observed in the case of even order multipole, while being a multipole of order \num{3} itself.
Tracking simulations were performed to investigate this effect.

\subsection*{Amplitude Detuning via Particle Tracking}

On-momentum particles were tracked with the MAD-X code through the LHC lattice of \num{2018}, with \qty{30}{\centi\meter} optics.
Nonlinearities were stripped down to the only contribution of sextupoles: no magnet errors, crossing angles or orbit bumps were included; and neither octupoles nor higher order magnets were powered.
This way the only non-negligible contribution to amplitude detuning is the sextupolar one.

Tracking was performed under in the case of both free and forced oscillations (in the horizontal plane) with similar amplitudes.
Several scenarios corresponding to different tune separations for the driven motion were explored.
Turn-by-turn data was gathered at all BPMs during the tracking, and the resulting file was analyzed as done for LHC measurements.

\Cref{figure:horizontal_detuning_comparison,figure:vertical_detuning_comparison} show the horizontal and vertical amplitude detuning results, respectively, for both free and driven motion.
For each result a linear regression is performed against the data to determine the detuning coefficient.

\begin{figure}[!htb]
    \centering
    \includegraphics*[width=\textwidth]{Figures/Other_Studies/direct_detuning_driven_vs_free.pdf}
    \caption{Natural horizontal tune \(Q_x\) shift with horizontal free or forced action (\(J_x\) or \(A_x\)), in the case of free oscillations (\textcolor{mplorange}{orange}) and driven motion with \(\Delta Q_x = -0.01\) (\textcolor{mplblue}{blue}) and \(\Delta Q_x = -0.005\) (\textcolor{mplred}{red}).}
    \label{figure:horizontal_detuning_comparison}
\end{figure}

\begin{figure}[!htb]
    \centering
    \includegraphics*[width=\textwidth]{Figures/Other_Studies/cross_detuning_driven_vs_free.pdf}
    \caption{Natural vertical tune \(Q_y\) shift with horizontal free or forced action (\(J_x\) or \(A_x\)), in the case of free oscillations (\textcolor{mplorange}{orange}) and driven motion with \(\Delta Q_x = -0.01\) (\textcolor{mplblue}{blue}) and \(\Delta Q_x = -0.005\) (\textcolor{mplred}{red}).}
    \label{figure:vertical_detuning_comparison}
\end{figure}

Results are compiled in \cref{table:detuning_coefficients_results}.

\begin{table}[!htb]
    \centering
    \begin{tblr}{colspec={ccc}}
        \hline
        \SetCell[r=2,c=1]{m,c} \textbf{Scenario}   &  \SetCell[c=2]{c} Detuning Coefficient [\unit{\per\micro\meter}]   \\
        \cline{2,3}                                &  Direct Term    &  Cross Term                                      \\
        \hline
        Free Oscillations                           &  \num{13.14}   &  \num{-14.94}                                    \\
        Driven (\(\Delta Q_x = -0.01\))             &  \num{24.04}   &  \num{-15.99}                                    \\
        Driven (\(\Delta Q_x = -0.005\))            &  \num{25.01}   &  \num{-15.50}                                    \\
        \hline
    \end{tblr}
    \caption{Direct and cross term detuning coefficients for free and forced motion, determined by performing a linear regression on data.}
    \label{table:detuning_coefficients_results}
\end{table}

\todo{Sentence about the results in the table.}

\subsection*{Conclusions}

While machine imperfection impede the measurement of strictly the sextupolar amplitude detuning in the LHC, particle tracking simulations confirm the presence of a suspected sextupolar contribution.
\todo{Rephrase this.
Firstly, this study demonstrates our ability to measure sextupolar amplitude detuning with an AC dipole.
Additionally, not only does this detuning appear present, it seems to behave reasonably close to expectations, behaving similarly to octupolar amplitude detuning when comparing the case of forced motion and free oscillations.}

\section{Software Developments}

\newcommand{\ilparagraph}[1]{\paragraph{\bfseries #1\textcolor{cern}{.}}}

Blah blah.

\ilparagraph{JobSubmitter}
Developed with J.~Dilly\orcidlink{0000-0001-7864-5448} and M.~Hofer\orcidlink{0000-0001-6173-0232}, JobSubmitter~\cite{CODE:OMC:pylhc_submitter} is a Python package for easily submitting parametrized studies to the HPC queuing system HTCondor~\cite{CODE:Douglas:condor-practice}.
It has quickly been adopted by colleagues due to its simplicity and efficiency.

\ilparagraph{Omc3}
The omc3~\cite{CODE:OMC:omc3} Python package is our main optics analysis and correction software, developed by most members of the OMC team.
Many contributions were made in the form of bug fixes, maintenance, translation of old codes to the new version, documentation, testing and implementation of CI/CD pipelines.

\ilparagraph{OMC Documentation}
Together with J.~Dilly\orcidlink{0000-0001-7864-5448}, we created a website for the OMC team~\cite{Website:OMC_Documentation} to serve as a wiki, resource collection and entrypoint for team members.
It compiles information about the physics behind the OMC activities, experimental procedures, documentation and guides on the team's software, and resources for newcomers.

\ilparagraph{PyhDToolkit}
Initially designed for personal use, PyhDToolkit~\cite{CODE:Soubelet:pyhdtoolkit} is a Python package for efficient simulations and visualizations with the cpymad~\cite{CODE:HIBTC:cpymad} and MAD-X~\cite{CODE:MADX_guide} codes.
It is the software used for all results in this thesis.
Its core functionality was extracted into the lightweight cpymadtools~\cite{CODE:Soubelet:cpymadtools} package, which has been adopted by colleagues for its clean and efficient API.

\ilparagraph{PyLHC}
Developped with J.~Dilly\orcidlink{0000-0001-7864-5448} and M.~Hofer\orcidlink{0000-0001-6173-0232}, PyLHC~\cite{CODE:OMC:pylhc} is a Python package holding various complementary scripts and modules to our other softwares.
Contributions include knob extraction scripts, data conversion functionality between database and simulation codes formats, and quick analysis of specific measurements.

\ilparagraph{Pyrws}
Created for the LHC Run~\num{3} commissioning of \num{2022}, pyrws~\cite{CODE:Soubelet:pyrws} is a Python package to design Rigid Waist Shift configurations from LHC optics.
It was used to prepare the experimental setup leading to the main work in this thesis, exposed in \cref{chapter:IR_Local_Coupling}.

\ilparagraph{Tfs-Pandas}
The tfs-pandas~\cite{CODE:OMC:tfs_pandas} Python package is a workhorse of our codes used to handle the Table Format System (TFS) files output by both our simulations codes and analysis software.
Significant effort was put into a complete rewrite of the package, speeding up the operations by up to \num{100} times as well as the addition of quality-of-life features and tests for robustness.
This package is very downloaded as its use is widespread.

\ilparagraph{Turn-by-Turn}
The turn-by-turn~\cite{CODE:OMC:turn_by_turn} Python package handles measurement data from different format corresponding to various machines.
Contribution was made in the form of extraction of this functionality from old codes and its rewrite into a clean API.

\section{Summary}

A statistical analysis of the phase reconstruction precision in the LHC was performed, while differentiating between BPM categories.
The results confirmed the influence of both BPM type and the \(\beta\)-functions at the measuring device on the precision of the phase reconstruction, with up to a factor \num{3} between the best and worst performing BPMs.
This knowledge could be applied in optics measurements algorithms to, for instance, select a subset of BPMs to use for a given measurement based its requirements.

A second simulation-based study was performed to investigate the sextupolar contribution to the amplitude detuning in the LHC, which cannot not be directly observed in the machine due to being entangled with other contributions.
A clear contribution, though relatively small compared to that of other sources, was observed.
The suspected behavior of this contribution, similar to that of the octupolar amplitude detuning, was confirmed.

Substantial effort put into the maintenance and development of the software suite used by the OMC team, both for analysis and simulation codes, was presented.
