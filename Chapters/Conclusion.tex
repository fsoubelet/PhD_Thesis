\chapter{Conclusion}
\label{chapter:conclusion}

% Suggestions from my friend ChatGPT:
% \begin{itemize}
%     \item Summary of Findings: Provide a concise summary of the key findings from your research. This should include both positive and negative results.
%     \item Contribution to the Field: Discuss how your research has contributed to the existing knowledge in your field. Highlight any novel or significant contributions your research has made.
%     \item Implications and Applications: Explain the implications and applications of your research findings. Consider how your findings could be used to solve real-world problems or inform future research.
%     \item Limitations and Future Directions: Acknowledge any limitations of your research and suggest areas for future research that could build on your findings. What is missing? What could be done in the future? How can we build up and improve even more?
%     \item Conclusion and Significance: Sum up the overall significance of your research and the contributions it has made to the field. End on a strong note that emphasizes the importance of your research and its potential impact. Outlook and broader implications for the LHC and other machines?
% \end{itemize}

% Actual items I am thinking of in order:
% \begin{itemize}
%     \item In \cref{chapter:ir_local_coupling} blah blah blah. \gls{RWS}. This is where I go over the main result, highlight that I have achieved the goal. Spend some time highlighting the advantage of this new approach (in terms of operational considerations, see bullet list below) and that it is valid for other colliders.
%     \item In \cref{chapter:ml_for_ir_coupling} talk about the machine learning approach, go over what was achieved in terms of results. Spend some time on what was not achieved, what could be done in the future. What are realistic targets for such an approach (model finding corrections, or building up on the approach of chap 3 model that could suggest optics changes to refine its predictions?).
%     \item In \cref{chapter:other_studies} talk about the other studies in the broader concept of \gls{OMC} activities, how we could use them to tie into the approaches mentioned previously (BPMs thing could help reach the better accuracy needed for the models developed in chap 4).
%     \item Final statement: the work presented in this thesis has contributed to the \gls{OMC} activities at the \gls{LHC} and has provided a new approach to the local \gls{IR} coupling correction. This approach has been shown to be advantageous in terms of operational considerations and has been successfully used in the \gls{LHC} during the \num{2022} commissioning. Tie into the fact that it is relevant to the other colliders. Mention both existing ones (SuperKEKB) and future ones (HLLHC, FCC, see end of PRAB paper).
% \end{itemize}

% Could be good points to raise in the conclusion / discussion of the PhD thesis and for future groundwork.
% \begin{enumerate}
%     \item Maybe something else than trimming the colin knob could be another good scenario (trim one side and change f1001 at IP differently for both beams, correct global coupling etc).
%     \item BUT the colin means we don't change the coupling bump in the region because of phase advances (Left and Right compensate) and we can do it at high intensity!
%     \item Probably this is not the optimal, which is one of the reasons we can't just optimize by luminosity itself with naive scans!
%     \item For machine protection you change AS LITTLE AS POSSIBLE with high intensity -> you want the settings you have from commissioning to not be changed if possible during operations.
%     \item Lose 2h of physics production beam time 10s of thousands in money (the RF at the moment is 10s of millions)
% \end{enumerate}

In their quest for new physics, modern particle colliders such as the \gls{LHC} are always striving to increase their \gls{luminosity} - both instantaneous and integrated - in order to reduce statistical uncertainties and increase the potential of new discoveries.
As a result the machine is subject to constant upgrades and improvements in terms of hardware, software and operational configuration.
In parallel to these upgrades, an ever tighter control of the \gls{beam} dynamics is necessary to ensure safe operation as well as the best possible performance out of the existing machine.
One performance limitation in the \gls{LHC} is the proper handling of local \gls{betatron_coupling} in the \glspl{IR} which, if left uncorrected, has the potential to degrade or even interrupt beam operations, as well as reduce the collision numbers as was observed during an incident in \num{2018}.
The upcoming \gls{HL-LHC} upgrade is bound to drastically enhance the impact of any uncorrected local coupling on the machine's performance, warranting a dedicated effort on this topic.

The objective of this work has been, in the context of contributing to the \gls{OMC} team's activities at large, to provide a new approach that can measure and correct the local \gls{IR} coupling and to experimentally determine corrections to be used in LHC operations.
The ideal solution, which was achieved through simulations, new experimental setups, and measurements; would not impact the existing stage of local corrections that are crucial for safe operation.
\break

The first two chapters of this document provided the reader with the theoretical and technical information necessary to understand the context of this work.
In \cref{chapter:ir_local_coupling} the 

Paragraph about the experimental results.

Paragraph about the operation with deadge MQSX.





\glsresetall                                     % reset glossary entries counts for the next chapter
