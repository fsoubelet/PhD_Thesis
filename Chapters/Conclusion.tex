\chapter{Conclusion}
\label{chapter:conclusion}

In the quest for new physics, modern particle colliders such as the \gls{LHC} are always striving to increase their \gls{luminosity} - both instantaneous and integrated - in order to reduce statistical uncertainties and increase the potential of new discoveries.
As a result the machine is subject to constant upgrades and improvements in terms of hardware, software and operational configuration.
In parallel to these upgrades, an ever tighter control of the \gls{beam} dynamics is necessary to ensure safe operation as well as the best possible performance out of the existing machine.

One such performance limitation in the \gls{LHC} is the proper handling of local \gls{betatron_coupling} in the \glspl{IR} which, if left uncorrected, has the potential to degrade or even interrupt beam operations, as well as reduce the collision numbers as was observed during an incident in \num{2018}.
The upcoming \gls{HL-LHC} upgrade is bound to drastically enhance the impact of any uncorrected local coupling on the machine's performance, warranting a dedicated effort on this topic.
The objective of this work has been to provide a novel approach that can measure and correct the local \gls{IR} coupling and to experimentally determine corrections to improve \gls{LHC} operations.
% The ideal solution, which was achieved through simulations, new experimental setups, and measurements; would not impact the existing stage of local corrections that are crucial for safe operation.
\break

The main body of the presented work addresses the target of measuring and correcting the local \gls{betatron_coupling} in the LHC \glspl{IR}.
The shortcomings of existing local \gls{IR} measurements methods for local coupling (segment-by-segment technique, k-modulation and alternative observables such as combined \glspl{RDT}) justified the creation of new tools.
The newly developed tools, the colinearity \gls{knob} and \gls{RWS}, allow relating relevant quantities of interest to strong observables such as the \glspl{RDT} and the \glssymbol{Cminus} in order to drive a correction.
The effect of each was demonstrated through extensive simulations, and an experimental setup to use them for the determination of corrections was laid out.

Measurements with the new experimental setup in the \gls{LHC} \num{2022} commissioning, at \qty{6.8}{\tera\electronvolt} and \(\beta^{\ast} =\)~\qty{30}{\centi\meter}, were presented together with the determined corrections obtained by comparing said measurements to simulations.
These corrections were trimmed in the machine and their efficiency assessed through instantaneous \gls{luminosity} measurements.
Great improvements in collision numbers were observed at the main \glspl{IP}: an additional \qty{9.7}{\percent} and \qty{5.2}{\percent} instantaneous luminosity at the \acrshort{ATLAS} and \acrshort{CMS} \glspl{experiment}, respectively.
Comparison of these results to expectations from simulations revealed a disagreement, with the \acrshort{ATLAS} numbers lower than expected.
This discrepancy was attributed to the early calibration setup of the detector, considering luminosity predictions and measurements showed great agreement at the same detector in identical measurements performed later in the year with higher intensity beams.
Great agreement was observed at the \acrshort{CMS} detector.

This new method allowed measuring, quantifying and correcting, for the first time, the previously unmeasurable local \gls{betatron_coupling} in the \gls{LHC} \glspl{IR}.
It has the advantage of being applicable early on in commissioning with low intensity beams and enables distinguishing between the two beams, which could require different types of adjustments: magnet realignment, orbit corrections, etc.
Furthermore, it has been shown to be relevant and applicable to both existing and future colliders.

Due to radiation exposure of the corrector magnets expected to surpass their allowed limits, the impact of losing such a crucial element was studied.
It was demonstrated through simulations that one could compensate for the loss of a single corrector magnet and guarantee both safe machine operation and squeezing of the beams down to collision configuration.
This compensation, however, would happen at the cost of luminosity: up to a \qty{60}{\percent} loss of instantaneous luminosity in the most affected case.
These expectations were confirmed through beam measurements in which the loss of correctors were simulated.
Using these numbers and taking into account the \glssymbol{betastar}-leveling as well as other parameters in the \num{2023} configuration, up to a \qty{25}{\percent} loss of integrated luminosity over a complete physics fill is expected.
Further studies are required for scenarios involving the simultaneous loss of both correctors in a given \gls{IR} and to investigate potential solutions.
\break

After the successful application of machine learning techniques to several other tasks in particle accelerators, including optics corrections, the possibility of using them for a different approach to tackle local coupling in the \gls{LHC} \glspl{IR} was explored.
Through extensive simulation-supported data sets, specific models were trained to predict the location of coupling sources in the form of quadrupole tilts from measured \glspl{RDT}.
Models were obtained that achieved great prediction performance on clean data sets but did not generalize as well to realistically noisy data sets.
Nonetheless, these models show promising results but improvements in the precision of the reconstructed \glspl{RDT} would be necessary before confidently using the models in the \gls{LHC}.
Such improvements could be achieved with better beam instrumentation, more accurate optics reconstruction methods or other dedicated machine learning models.
\break

\Cref{chapter:others_and_software} details additional work done in the context of the \gls{OMC} team's activities.
Through statistical analysis, the influence of both \gls{BPM} type and \gls{beta-function} at the measuring device on the precision of the phase reconstruction in the \gls{LHC} \glspl{BPM} was determined.
The results allowed to identify the best conditions for phase reconstruction, which could be applied in optics measurements algorithms to, for instance, select a subset of \glspl{BPM} to use for a given measurement based on its requirements.
The impact of the sextupolar contribution to the \gls{ampdet} in the \gls{LHC} was observed through simulations, and it was shown that the contribution behaved similarly to the octupolar one, exhibiting a factor \num{2} between the free and forced oscillation cases for the direct detuning term, while exhibiting a similar cross detuning term.
The substantial contribution to the \gls{OMC} team's software as well as the development of new tools was also presented, which contributed to the improvement of the optics simulations and analysis ecosystem as a whole in \acrshort{BE}-\acrshort{ABP}.
\break
\todo{The phase advance, shows the potential of paring BPMs for better phase and potentially beta measurement.}

The overall goal of this thesis has been to develop a new approach to the measurement and correction of local \gls{betatron_coupling} in the \gls{LHC} \glspl{IR}.
Different methods were developed and tested including, mainly, a new experimental setup which was used to measure and correct local coupling in the \gls{LHC} \num{2022} commissioning, improving the instantaneous luminosity at the \acrshort{ATLAS} and \acrshort{CMS} \glspl{experiment}.
This new method is relevant for typical conditions of an Interaction Region, and therefore relevant for other existing colliders such as SuperKEKB, but also future machines such as the \gls{HL-LHC} and the \gls{FCC} variants.
The work presented in this document contributes to the improved performance and understanding of the \gls{LHC}, but also potentially other present and future particle colliders.


\glsresetall                                     % reset glossary entries counts for the next chapter
