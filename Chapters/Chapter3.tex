% Chapter 3

\chapter{Interaction Region Local Coupling Correction in the LHC} % Main chapter title

\label{Chapter3} % For referencing the chapter elsewhere, use \ref{Chapter3}

%----------------------------------------------------------------------------------------

Some paragraph before the first section.

%----------------------------------------------------------------------------------------

\section{Linear Coupling in the Interaction Regions}

\subsection{Overview of IR Difficulties (phase advances suck, DFFT of x -jpx, no instruments)}

\subsection{Twiss with Coupling and Ripken parameters}

\subsection{Equivalency of Ripken and Tracking when looking at beam size}

\subsection{Plan for Correction (or later?)}

%----------------------------------------------------------------------------------------

\section{The Hunt for an Observable}

\subsection{Combined RDTs}

Some theory here (see franchi's paper, see michael's paper), it can use DFFT of x/y only.
Some studies that it's difficult to use directly (2021.8), maybe sbs?

\subsection{SbS with combined RDTs and that it works better than with rdts?}

\subsection{Forced RDTs}

Why am I looking into this again?
Potentially if I have time we can see if using non-compensated stuff gives better corrections.
Very optionnal at the moment.

\subsection{Conclusioon that we might need to look at outside observables}

%----------------------------------------------------------------------------------------

\section{Proof of Principle: Measurement and Correction of Local Coupling in the LHC Interaction Regions}

\subsection{Relating to outside observables}

\subsection{Beam-Based Study of IRs Local Coupling}

\subsection{Simulations of IRs Local Coupling}

%----------------------------------------------------------------------------------------

\section{Impact of Local Linear Coupling Correction on Beam Lifetime/Quality?}

\subsection{Impact on Tune Footprint (hopefully minimal)?}

\subsection{Impact on Dynamic Aperture (hopefully none)?}

\subsection{Impact on Luminosity (hopefully yayyy)?}

%----------------------------------------------------------------------------------------

\section{Operational Correction Procedure(s)}

\subsection{Full Procedure Steps}

\subsection{Developped Software}

%----------------------------------------------------------------------------------------

\section{Containment Plan in Case of MQSX Failures}

\subsection{Lifetime Considerations of MQSX Elements}

See F. Cerutti slides (slides 22, 23, and 12 to 16) at 2021 Evian Workshop
(https://indico.cern.ch/event/1077835/contributions/4533356/attachments/2352134/4012821/Evian.pdf)

Explain that there is a real risk that some of our MQSX mqgnets die, especially the ones in IR1 (ATLAS).
In this case, we will need a containment plan, as not only are they used for the but the local corrections they are a part of are a baseline for us to compute higher order terms corrections.
This means in simple terms that MQSX dying will impact the LHC's operations, and potentielly shut the machine down.

\subsection{Containment Concept: Tilt of Triplet Elements}

Talk about what we want to do (tilt Q3 or Q2) to induce a skew component + simulation results.
Found settings of the Q3 or Q2 that would negate the MQSX one determined in beam test / commissioning.
The idea is if we 

Also show we have a very minimal beta-beating from this.
Show we have had a look at different optics (30cm and 1m betastar) and it works for both.

\subsection{Operational Constraints}

The LHC systems are not meant for this, but for vertical alignment of these magnets!
System relies on bellows: 2 pieds IP side and 1 pied other side for Q2 for instance.
This means that inducing rotation is not only not the design purpose, but also imperfect (on move les 2 pieds pour essayer de mimer une rotation mais c'est pas parfait).
Would be very good to have a plot of the assemblies here to show what I mean. Ask MP people? See in the LHC design report?

It is considered by MP people to be quite dangerous to do this unless forced to (read an MQSX dies), especially in cold mass, as if we damage the belows then we're in for 1 year of shutdown to repair it.
Say that for these reasons we decided not to test this concept in the machine, unfortunately.

%----------------------------------------------------------------------------------------

\section{Conclusions}

%----------------------------------------------------------------------------------------