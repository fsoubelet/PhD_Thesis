\chapter{Theory of Single-Particle Beam Dynamics in the Large Hadron Collider} % Main chapter title

\label{Chapter:Theory} % For referencing the chapter elsewhere, use \cref{Chapter:Theory}

%----------------------------------------------------------------------------------------

The design, operation, performance, and safety of a particle accelerator depend on the study of beam dynamics, a field of accelerator science.
This section provides an overview of the beam dynamics theories relevant to the material in this thesis, and more specifically beam optics.
\todo{The chapter starts out with a description of linear dynamics, then moves on to aspects of non-linear dynamics, and ends with a discussion of luminosity}.

%----------------------------------------------------------------------------------------

\section{Linear Beam Dynamics}

The linear dynamics of an accelerator are, mainly, the endeavor to bend and focus particle beams to confine them within the machine's aperture.
To force the beam's particles into a closed trajectory, they are subjected to magnetic fields that deflect their trajectories.
The force exerted onto the beam is the Lorentz force \(F_{L}\), given by the equation:

\begin{equation}
    \vec{F_L}= \dfrac{d\vec{p}}{dt} = q (\vec{E} + \vec{v} \times \vec{B}) ,
    \label{equation:lorentz_force}
\end{equation}
where \(p\) is the particle momentum, \(q\) the particle charge, \(\vec{E}\) the electric field, \(\vec{v}\) the particle velocity and \(\vec{B}\) the magnetic field.
In most accelerators, including the LHC, the particles' speed is close to the celerity of light \(c\) and the force from the magnetic field is significantly stronger than that produced by the electric field for realistic values of \(\vec{E}\) and \(\vec{B}\).
As a result, in high energy particle accelerators magnetic fields are typically used to guide particles.
The guiding magnetic field can be expanded into a series of multipolar fields, for instance here in the horizontal plane:
% ---------- Equation ----------
\begin{align}
    B_{y} = 
    \tikz[baseline]{
        \node[draw=red,rounded corners,anchor=base] (m1)
        {$\displaystyle B_{y0}$};
        \node[below of=m1] (l1) {dipole};
        \draw[-,red] (l1) -- (m1);
    }
    +
    \tikz[baseline]{
        \node[draw=red,rounded corners,anchor=base] (m2)
        {$\displaystyle \frac{dB_{y}}{dx} x$};
        \node[below of=m2] (l2) {quadrupole};
        \draw[-,red] (l2) -- (m2);
    }
    +
    \tikz[baseline]{
        \node[draw=red,rounded corners,anchor=base] (m3)
        {$\displaystyle \frac{1}{2!} \frac{d^2B_{y}}{dx^2} x^2$};
        \node[below of=m3] (l2) {sextupole};
        \draw[-,red] (l2) -- (m3);
    }
    +
    \tikz[baseline]{
        \node[draw=red,rounded corners,anchor=base] (m4)
        {$\displaystyle \frac{1}{3!} \frac{d^3B_{y}}{dx^3} x^3$};
        \node[below of=m4] (l2) {octupole};
        \draw[-,red] (l2) -- (m4);
    }
    + \ldots    
\end{align}

% Within the linear approximation, non-linear magnetic fields are ignored and the beam dynamics is described by linear differential equations.
Bending forces are supplied by dipole magnets with a magnetic field perpendicular to the beam trajectory, while focusing is typically performed with the use of quadrupole magnets.
Higher orders belong to the nonlinear dynamics and will be discussed later on.
\Cref{figure:frenet_serret_system} illustrates the Frenet-Serret coordinate system traditionally used in linear beam dynamics.
\bigbreak

\begin{figure}[!htb]
    \begin{center}
    \includegraphics[width = 0.8\linewidth]{Figures/Chapter2/Frenet_Serret_Coordinate_System.png}
    \caption{The Frenet-Serret coordinate system used in accelerator physics. Here \(\hat{x}\), \(\hat{y}\), and \(\hat{s}\) form the right-handed orthogonal basis, while \(\rho\) is the local bending radius.}
    \label{figure:frenet_serret_system}
    \end{center}
\end{figure}

The coordinate system travels longitudinally with the particle, along a reference trajectory defined by an ideal, or \emph{synchronous}, particle.
The longitudinal curvilinear coordinate is \(s\), and denotes the position of the particle along the ideal orbit with respect to an arbitrary initial point at \(s = 0\).
One can define a local radius of curvature, \(\rho(s)\), which depends on the local magnetic field \(\vec{B}\) and varies along ring.
The transverse phase space is defined by \((x, x^{\prime}, y, y^{\prime})\), where \(x\) and \(y\) are a particle's coordinates in the transverse plane relative to the reference trajectory.
The \(x^{\prime}\) and \(y^{\prime}\) coordinates are \emph{divergent angles}, with the prime indicating differentiation with respect to \(s\).
\bigbreak

In the linear regime, magnetic dipoles define ideal orbit for a particle of \emph{reference momentum} \(p_0\).
This ideal orbit goes through the magnetic center of all elements in the machine to close back on itself after a revolution, and is called a \emph{closed orbit}.
In practice the real closed orbit will deviate from the ideal designed orbit due to various effects such as dipolar field errors.
Particles within the beam are distributed in amplitude and oscillate around the closed orbit, which corresponds to the path of a particle with zero amplitude within the beam, because of focusing forces. 
\bigbreak

Focusing forces are typically provided by magnetic quadrupoles.
A quadrupolar field acting on a charged particle displaced from the closed orbit will provide a restoring (focusing) force proportional to the displacement in one transverse plane, while simultaneously providing a divergent (defocusing) force in the other. 
As a convention, a quadrupole focusing in the horizontal transverse plane and defocusing in the vertical is referred to as a \emph{focusing quadrupole}. 
Respectively, a quadrupole defocusing in the horizontal plane but focusing in the vertical is referred to as a \emph{defocusing quadrupole}.
A net focusing effect in both planes can be obtained with a setup of quadrupoles of alternating polarity in equal distance, a widely used configuration named the $\mathrm{FODO}$ cell, a layout alternating quadrupoles in equal distance.
\Cref{figure:dipole_quadrupole_fields} illustrates magnetic fields in an idealized dipole and quadrupole.
\bigbreak

\begin{figure}[htp]
    \centering
    \subfloat[.8\linewidth][Ideal dipole.]{
        \includegraphics[width=6.5cm]{Figures/Chapter2/ideal_dipole_cos_theta.png}
        \label{fig:ideal_dipole}
    }
    \hspace{0.5cm}
    \subfloat[.8\linewidth][Ideal quadrupole.]{
        \includegraphics[width=6.5cm]{Figures/Chapter2/ideal_quadrupole_cos_2theta.png}
        \label{fig:ideal_quadrupole}
    }
    \caption{Magnetic fields and forces in an idealized dipole and quadrupole, with a \(\cos(\theta)\) and \(\cos(2\theta)\) current distribution in the circular coil, respectively. Current in the dipole and quadrupole coils are indicated in color. Taken from \cite{CERN:Russenschuck:CAS_Design_Magnets}.}
    \label{figure:dipole_quadrupole_fields}
\end{figure}

In circular machine focusing from quads is periodic in $s$ with a period of max the circumference of the machine.
Motion in the transverse plane is described by Hill’s equation, \cref{eq:hill_equation}, where $k(s)$ is a periodic coefficient describing the restoring force due to the distribution of focusing fields around the ring.
\bigbreak

\begin{equation}
    u^{\prime \prime} \pm k(s) u = 0; \quad u = x, y; \quad u^{\prime} = \frac{\mathrm{d}u}{\mathrm{d}s}
    \label{eq:hill_equation}
\end{equation}
\bigbreak

Solutions to Hill’s equation take the form of \cref{eq:hill_solution},
\bigbreak

\begin{equation}
    \begin{aligned}
    x &= \sqrt{\beta_{x}(s) \epsilon_{x}} \cos \left(\phi_{x}(s) + \phi_{x_0}\right) \\
    y &= \sqrt{\beta_{y}(s) \epsilon_{y}} \cos \left(\phi_{y}(s) + \phi_{y_0}\right)
    \end{aligned}
    \label{eq:hill_solution}
\end{equation}

where $\epsilon$ is the emittance of a particle and is a constant of the motion at a given energy.
$\beta(s)$ is called the \emph{beta-function} and describes the variation of the oscillation envelope around the ring. 
In particle colliders such as the LHC, it is usual to denote the \betafunctions at the Interaction Points (where the beams are made to collide) as \betastar.
\bigbreak

Particles oscillate around the ring, \emph{betatron oscillations}, and the number of oscillations around the ring is the \emph{tune}, $Q_{x,y}$.
The tune is defined in \cref{eq:tune_definition}, where $\Delta \phi_{x, y}$ is the total betatron phase advance undergone by a particle during one revolution around the accelerator ring.
\bigbreak

\begin{equation}
    Q_{x, y} = \frac{1}{2 \pi} \Delta \phi_{x, y} = \frac{1}{2 \pi} \oint \frac{\mathrm{d}s}{\beta_{x, y}(s)}
    \label{eq:tune_definition}
\end{equation}

\Cref{figure:particle_trajectories} shows a tracking simulation of a particle undergoing such betatron oscillations in the LHC Arc12.
Dipole errors were added which have distorted the closed orbit away from the ideal path.
\bigbreak

\begin{figure}[!htb]
    \begin{center}
    \includegraphics[width = 0.7\linewidth]{Figures/placeholder.png}
    \caption{Tracking simulation in LHC Arc12 of a particle undergoing betatron oscillations.}
    \label{figure:particle_trajectories}
    \end{center}
\end{figure}

Define the \emph{gamma-function} $\gamma(s)$ which describes the envelope of oscillations in $x\prime$ and $y\prime$.
Both are related by the \emph{alpha-function}:
\bigbreak

\begin{equation}
    \alpha_{x, y} = -\frac{1}{2} \frac{\mathrm{d}}{\mathrm{d}s} \beta_{x, y}(s) = \sqrt{\gamma_{x, y}(s) \beta_{x, y}(s) - 1}
    \label{eq:alpha_function}
\end{equation}
\bigbreak

In linear regime, trajectories are ellipses in phase space $(x, x\prime, y, y\prime)$.
$\alpha(s)$, $\beta(s)$, $\gamma(s)$ and $\epsilon$ define the equation of the ellipse, \cref{eq:ellipse_equation}.
\Cref{figure:phase_space_ellipse} shows a schematic illustration of the phase space ellipse.
\bigbreak

\begin{equation}
    \gamma_{u}(s) u^{2} + 2 \alpha_{u}(s) u u^{\prime} + \beta_{u}(s) z^{\prime 2} = \epsilon \quad \text { where } u = x, y
    \label{eq:ellipse_equation}
\end{equation}
\bigbreak

\begin{figure}[!htb]
    \begin{center}
    \includegraphics[width = 0.85\linewidth]{Figures/placeholder.png}
    \caption{Phase space ellipse in the transverse $z$, $z\prime$ plane, where $z$ represents either $x$ or $y$.}
    \label{figure:phase_space_ellipse}
    \end{center}
\end{figure}

Say emittance ($\epsilon$) defines phase space ellipse area.
Liouville's theorem: phase space volume (ellipse area) is constant in a closed system.
In the LHC, protons, little synchrotron radiation [ref 20 ewen] we can consider emittance a constant (but at multi-$\mathrm{TeV}$ energy radiation emission may become non-negligible).
Acceleration -> no more Liouville and \emph{physical emittance} ($\epsilon$) will reduce with increasing energy.
One can construct \emph{normalized emittance} ($\epsilon_{\gamma}$) which is invariant with beam energy.
See \cref{eq:normalized_emittance}, where $\beta_{rel}$ and $\gamma_{rel}$ are the relativistic beta and gamma functions:
\bigbreak

\begin{equation}
    \epsilon_{\gamma} = (\beta_{\mathrm{rel}} \gamma_{\mathrm{rel}}) \epsilon
    \label{eq:normalized_emittance}
\end{equation}
\bigbreak

For a specific particle we use \emph{single particle emittance}.
Different particles may have different ones.
They will go through (undergo?) betatron oscillations of different amplitudes.
We can define \emph{beam emittance}: typically defined as the emittance corresponding $1\sigma$ amplitude in assumed Gaussian particle distribution.
\bigbreak

The phase space trajectory of a particle depends on $\alpha(s)$, $\beta(s)$, and $\gamma(s)$.
The transformation to \emph{Courant-Snyder coordinates} [ref 21 ewen] (sometimes called \emph{normalized Courant-Snyder coordinates}) removes this dependency, see \cref{eq:courant_snyder_coordinates},
\bigbreak

\begin{equation}
    \left(\begin{array}{c}
    \hat{u} \\
    \hat{u}^{\prime}
    \end{array}\right) = \left(\begin{array}{cc}
    \frac{1}{\sqrt{\beta_{u}(s)}} & \frac{\alpha_{u}(s)}{\sqrt{\beta_{u}(s)}} \\
    0 & \sqrt{\beta_{u}(s)}
    \end{array}\right)\left(\begin{array}{c}
    u \\
    u^{\prime}
    \end{array}\right) \quad \text{ where } u = x, y
    \label{eq:courant_snyder_coordinates}
\end{equation}

where the Courant-Snyder coordinates are denoted by \^{}.
In this new system particles follow circular trajectories in phase space.
\bigbreak

Say particles within the beam have a distribution in momentum, around designed momentum $p_0$.
For a particle that doesn't have $p_0$ we define the \textit{relative momentum deviation} $\delta$, \cref{eq:momentum_deviation}:
\bigbreak

\begin{equation}
    \delta = \frac{p - p_0}{p_0}
    \label{eq:momentum_deviation}
\end{equation}
\bigbreak

Define \textit{magnetic rigidity}: relates magnetic flux $\mathbf{B}$ (perpendicular to motion) with local radius of curvature and particle momentum $\mathbf{P}$.
See \cref{eq:beam_rigidity}:
\bigbreak

\begin{equation}
    \lvert \mathbf{B} \rho \rvert = \frac{\lvert \mathbf{P} \lvert}{e}
    \label{eq:beam_rigidity}
\end{equation}
\bigbreak

Say difference to $p_0$ introduce \emph{chromatic} errors into the beam dynamics.
The most important one is \emph{dispersion}.
Different momenta(um?) -> different beam rigidity -> different local radius of curvature in dipoles -> different orbit.

Deviation to reference orbit is defined by the \emph{Dispersion function}, $D(s)$ [ref 23 ewen].
In a region of non-zero dispersion the contribution to the orbit of a particle is described by \cref{eq:dispersion_contribution_to_orbit}:
\bigbreak

\begin{equation}
    \begin{aligned}
    \Delta x_{\mathrm{dispersion}} &= D_{x}(s) \delta \\
    \Delta y_{\mathrm{dispersion}} &= D_{y}(s) \delta
    \end{aligned}
    \label{eq:dispersion_contribution_to_orbit}
\end{equation}
\bigbreak

The orbit of any given particle is defined by \cref{eq:_particle_orbit}:
\bigbreak

\begin{equation}
    u = u_{\mathrm{betatronic}} + u_{\mathrm{dispersion}} + \left.u_{\mathrm{closed \ orbit}} \right\rvert_{\delta = 0}
    \label{eq:_particle_orbit}
\end{equation}

%----------------------------------------------------------------------------------------

\section{Non-Linear Magnetic Multipoles}

%----------------------------------------------------------------------------------------

\section{Formalism of Non-Linear Beam Dynamics}

%----------------------------------------------------------------------------------------

\section{Phenomenology of Non-Linear Beam Dynamics}

\subsection{Chromaticity}

\subsection{Detuning with Amplitude}

\subsection{Decoherence}

%----------------------------------------------------------------------------------------

\section{Normal Form Formalism and Resonance Driving Terms}


The derivation in this section follows the approach given in \cite{Tomas_thesis, Franchi_thesis}.
While the non-linear dynamics can not be described with matrices, it can be described by the transfer map formalism.
In the frame where the one turn map is represented by a pure rotation - in normal forms coordinates - it can be written as \cite{Tomas_thesis}:\\

\begin{equation}
    \mathcal{M} = 
    e^{:\eqnmarkbox[blue]{node1}{\tilde{h_{1}}}:}
    e^{:\eqnmarkbox[red]{node2}{\tilde{h_{2}}}:}
    \ldots
    e^{:\eqnmarkbox[green]{noden}{\tilde{h_{n}}}:}
    R
    \label{eq:norm_form_one_turn_map}
\end{equation}
\annotate[yshift=0.5em]{above, left}{node1}{First element}
\annotate[yshift=-0.75em]{below}{node2}{Second element}
\annotate[yshift=0.5em]{above, right}{noden}{Nth element}\\\\
% ------------------ %
where \(e^{:\tilde{h_{1}}:}\) is an exponential Lie operator describing a nonlinear element, and \(\mathbf{R}\) is the rotation matrix describing the linear motion.
This simplifies through the Campbell-Baker-Hausdorff theorem to:

\begin{equation}
    \mathcal{M} = e^{:h:} R
\end{equation}

In case the \(\tilde{h_{n}}\) are small, then \(h\) may be approximated by:

\begin{equation}
    h = \sum_{n=1}^{N} \tilde{h}_{n} + \sum_{n, m<n}^{N} \left[\tilde{h}_{m}, \tilde{h}_{n} \right] + \ldots
    \label{eq:h_expansion}
\end{equation}

Using only the first order in \(\tilde{h_{n}}\), \(h\) may be expanded according to \cref{eq:h_approximation_first_order} using the action-angle variables TODO.

\begin{equation}
    h = \sum_{j k l m} h_{j k l m} \left(2 J_{x}\right)^{\frac{j+k}{2}} \left(2 J_{y}\right)^{\frac{l+m}{2}} e^{i \left[(j-k)\left(\phi_{x}-\phi_{x_{0}}\right) + (l-m)\left(\phi_{y}-\phi_{y_{0}}\right) \right]}
    \label{eq:h_approximation_first_order}
\end{equation}

Here \(h_{j k l m}\) are Hamiltonian coefficients representing the contributions from multipoles of order \(n = j + k + l + m\).
A multipole of order \(n\) generates terms in the Hamiltonian \(\propto x^{j+k} y^{l+m}\), where again \(n = j + k + l + m\).

In the case of a skew quadrupole, such an element gives rise to terms in the Hamiltonian \(\propto xy\), meaning that it contributes to the Hamiltonian terms \(h_{1010}\), \(h_{1001}\), \(h_{0110}\) and \(h_{0101}\).
The idea behind normal form coordinates is to perform a transformation from a system with amplitude and phase dependence to a simpler form.
The simplest form is an amplitude dependent rotation, i.e. a rotation in phase space where the angle depends on the amplitude of the particle.
An excellent approach to normal forms formalism can be found in \cite{Carlier_thesis}.

The coordinate change is represented by a similarity transformation of the one turn map:

\begin{equation}
    e^{-: F:} e^{: h:} R e^{: F:}
\end{equation}
% ------------------ %
where \(F\) is the generating function for the transformation.
The formal solution to finding the generating function F is given in \cite{Forest_normal_forms} and the explicit expression is obtained in \cite{Tomas_thesis} as

\begin{equation}
    F = \sum_{j k l m} f_{j k l m} \left(2 I_{x}\right)^{\frac{j+k}{2}}\left(2 I_{y}\right)^{\frac{l+m}{2}} e^{i\left[(j-k)\left(\psi_{x}-\psi_{x_{0}}\right)+(l-m)\left(\psi_{y}-\psi_{y_{0}}\right)\right]}
    \label{eq:F_generating}
\end{equation}
% ------------------ %
where \(f_{jklm}\) are the resonance driving terms corresponding to the Hamiltonian terms \(h_{jklm}\) respectively.
They can be expressed according to \cref{eq:f_rdts} \cite{Tomas_thesis, Franchi_thesis}, where \(Q_x\) and \(Q_y\) are the unperturbed tunes.

\begin{equation}
    f_{jklm} = \frac{h_{jklm}}{1 - e^{i 2 \pi \left[(j-k) Q_{x} + (l-m) Q_{y} \right]}}
    \label{eq:f_rdts}
\end{equation}
% ------------------ %
\Cref{eq:f_rdts} diverges when \(j, k, l, m, Q_x\) and \(Q_y\) satisfy the condition:

\begin{equation}
    (j-k) Q_{x} + (l-m) Q_{y} = p \quad \text{ where } p \in \mathcal{Z}
    \label{eq:resonance_condition}
\end{equation}
% ------------------ %
Hence, the \(f_{jklm}\) terms are the driving terms of the resonances \([(j-k),(l-m)]\).
Every Hamiltonian term is associated with a resonance, which explains the term Resonance Driving Terms.
The normalized Courant-Snyder coordinates are related to the action-angle variable as

\begin{equation}
    \begin{aligned}
    z &= \sqrt{2 J_{z}} \cos (\phi_{z} - \phi_{z_{0}}) \\
    p_{z} &= -\sqrt{2 J_{z}} \sin (\phi_{z} - \phi_{z_{0}}) \quad \text { where } z=x, y
    \end{aligned}
    \label{eq:courant_snyder_to_action_angle}
\end{equation}
% ------------------ %
It is convenient to introduce the resonant basis h defined as

\begin{equation}
    \begin{aligned}
    h_{z}^{\pm} &= z \pm i p_{z} = \sqrt{2 J_{z}} e^{\mp i \left(\phi_{z}-\phi_{z_{0}}\right)} \quad \text { where } z=x, y \\
    \mathbf{h} &= \left( h_{x}^{+}, h_{x}^{-}, h_{y}^{+}, h_{y}^{-} \right)
    \end{aligned}
    \label{eq:resonant_basis_h}
\end{equation}

The transformation to a new set of Normal Form coordinates \(\left(\zeta_{x}^{+}, \zeta_{x}^{-}, \zeta_{y}^{+}, \zeta_{y}^{-}\right)\) is given by the operator \(e^{: -F :}\).
This is expressed as

\begin{equation}
    \zeta_{z}^{\pm} = \sqrt{2 I_{z}} e^{\pm i \left(\phi_{z}+\phi_{z 0} \right)} = e^{:-F:} h_{z}^{\pm}
    \label{eq:action_angle_to_normal_form}
\end{equation}
% ------------------ %
where \(I_{z}\) is the invariant of motion in the new frame.
The one-turn map in normal form coordinates is by construction an amplitude dependent rotation, and hence the motion in these coordinates as a function of the turn number N is given by

\begin{equation}
    \zeta_{z}^{-}(N) = \sqrt{2 I_{z}} e^{2 \pi \nu_{x} N + \phi z_{0}}
    \label{eq:normal_form_by_turn}
\end{equation}

The inverse transformation from the new action-angle variables to the linearly normalized variable is to first order written as

\begin{equation}
    h_{z}^{-} = e^{: F:} \zeta_{z}^{-} \simeq \zeta_{z}^{-} + \left[F, \zeta_{z}^{-}\right]
    \label{eq:inverse_normal_form_transform}
\end{equation}

and using \cref{eq:normal_form_by_turn} and \cref{eq:inverse_normal_form_transform} the normalized coordinates can be expressed in the form

\begin{equation}
    \begin{aligned}
    h_{x}^{-}(N) &= \sqrt{2 I_{x}} e^{i\left(2 \pi \nu_{x} N - \psi_{x_{0}}\right)} - \\
    & 2 i \sum_{jklm} j f_{jklm} \left(2 I_{x}\right)^{\frac{j+k-1}{2}} \left(2 I_{y}\right)^{\frac{l+m}{2}} e^{i \left[(1-j+k) \left(2 \pi \nu_{x} N-\psi_{x_{0}}\right) + (m-l) \left(2 \pi \nu_{y} N-\psi_{y_{0}}\right) \right]} \\
    h_{y}^{-}(N) &= \sqrt{2 I_{y}} e^{i\left(2 \pi \nu_{y} N - \psi_{y_{0}}\right)} - \\
    & 2 i \sum_{jklm} l f_{jklm} \left(2 I_{x}\right)^{\frac{j+k}{2}} \left(2 I_{y}\right)^{\frac{l+m-1}{2}} e^{i \left[(k-j) \left(2 \pi \nu_{x} N-\psi_{x_{0}}\right) + (1-l+m) \left(2 \pi \nu_{y} N-\psi_{y_{0}}\right) \right]}
    \end{aligned}
    \label{eq:normal_form_coordinates}
\end{equation}

https://cds.cern.ch/record/1533084/files/CERN-THESIS-2013-022.pdf


\subsection{Betatron Coupling RDTs}

Talk about the closest tune approach \(\Delta Q_{\mathrm{min}}\) quantity.
Without betatron coupling in the machine, horizontal and vertical tunes can be matched to the same fractional value.
In the presence of betatron coupling, the \emph{coupled fractional tunes} \(Q_1\) and \(Q_2\) cannot be matched to the same value and a minimal separation can be observed.
This separation is called the \emph{closest tune approach} and is given as the \(\Delta Q_{\mathrm{min}}\) quantity.
\Cref{figure:closest_tune_approach} shows an illustration of the phenomenon, where both the unperturbed and coupled fractional tunes are plotted against the uncoupled tune split.
The closest tune approach is the minimum distance between the two red curves, highlighted by an arrow.

\begin{figure}[!htb]
    \begin{center}
    \includegraphics[width = 0.9\linewidth]{Figures/Chapter2/tune_perturbation.pdf}
    \caption{Illustration of coupled and uncoupled fractional tunes versus the uncoupled tune split.}
    \label{figure:closest_tune_approach}
    \end{center}
\end{figure}

%----------------------------------------------------------------------------------------

\section{Luminosity}

%----------------------------------------------------------------------------------------