% Chapter 2

\chapter{Optics Measurements and Corrections at the LHC} % Main chapter title

\label{Chapter2} % For referencing the chapter elsewhere, use \Cref{Chapter2}

%----------------------------------------------------------------------------------------

The LHC requires good control of the optics functions, and for instance a good control of the \(\beta\)-functions is essential for safe beam operations due to the destructive power of the LHC beams.
Moreover, a good control of the \(\beta^{\ast}\) is necessary to guarantee good luminosity production.
To validate the machine optics and to identify possible errors beam measurements are performed, from which corrections bringing the machine as close to the nominal scenario as possible are calculated and applied. 
This chapter gives an overview of the LHC machine and its operational cycle, as well as the essential beam instrumentation used for optics measurements.
Additionally, concepts of corrections methods are given.

%----------------------------------------------------------------------------------------

%\section{The CERN Accelerator Complex}

%----------------------------------------------------------------------------------------

\section{The LHC Lattice}

%----------------------------------------------------------------------------------------

\subsection{The LHC Arcs}

\subsection{The LHC Insertion Regions}

\subsection{Error Estimates for the LHC Lattice}

%----------------------------------------------------------------------------------------

\section{The LHC Experimental Interaction Regions (EIR)}

\subsection{Interaction Point}

\subsection{The LHC Triplet}

\subsection{Separation Dipoles}

\subsection{Matching Section}

\subsection{Dispersion Suppressor}

%----------------------------------------------------------------------------------------

\section{The Operational Cycle of the LHC}

%----------------------------------------------------------------------------------------

\section{Measurements and Corrections}


\subsection{Beam Instrumentation for Optics Measurements}

\subsubsection{Beam Position Monitors}

% \subsubsection{Beam Current Monitors}

\subsubsection{BBQ System for Tune Measurements}

\subsubsection{Excitation Devices}


Measurements of beam optics are done by generating large transverse beam oscillations, typically much larger than the natural beam size.
A non-destructive method for beam excitation in hadron machines is achieved using an AC-dipole

\subsection{Optics Measurements}

- excite
- harmonic analysis
- optics analysis

\subsubsection{Turn-by-turn Measurements}

\subsection{Correction Principles}

\subsubsection{Global Corrections}

\subsubsection{Local Corrections}

Here mention the need for new stuff as things get hairy with the LHC upgrades.