\chapter{Thin Kick Hamiltonian Derivation}
\label{appendix:hamiltonian_derivation}

This appendix contains the full derivation leading to the result of \cref{equation:h_thin_kick_expansion}, showing the hamiltonian thin kick expansion.

\section{Multinomial Expansion}
\label{section:multinomial_expansion}

Let us start by reminding the multinomial expansion rule.
For any positive integer \(m\) and non-negative integer \(m\), the multinomial expansion describes the expansion of a sum of \(m\) numbers raised to the power \(n\):

\begin{equation}
    (x_1 + \ldots + x_m)^n = \sum_{k_1 + \ldots + k_m = n} \frac{n!}{k_1! \ldots k_m!} x_1^{k_1} \ldots x_m^{k_m} \text{ .}
    \label{equation:multinomial_expansion}
\end{equation}

Another form of writing the multinomial expansion is with the Kronecker delta:

\begin{equation}
    \delta_{i,j} = 
        \begin{cases} 
            0 & \text{if } i \neq j \text{ ,} \\
            1 & \text{if } i = j \text{ .}
        \end{cases}
    \label{equation:kronecker_delta}
\end{equation}

In this case, \cref{equation:multinomial_expansion} can be written:

\begin{equation}
    (x_1 + \ldots + x_m)^n = \sum_{k_1 + \ldots + k_m \leq n} \delta_{j + k + l + m, n} \frac{n!}{k_1! \ldots k_m!} x_1^{k_1} \ldots x_m^{k_m}
    \label{equation:multinomial_expansion_kronecker}
\end{equation}

Note that in this form, the indices of the summation \(k_1 + k_2 + \ldots + k_m\) are \emph{less than or equal} to \(n\).
Let us illustrate with the following example:

\begin{equation}
    \begin{aligned}
    (a + b + c + d)^n &= \sum_{j + k + l + m = n}  \frac{n!}{j! k! l! m!} a^j b^k c^l d^m  \\
                      &= \sum_{j + k + l + m \leq n} \delta_{j + k + l + m, n} \frac{n!}{j! k! l! m!} a^j b^k c^l d^m  \\
                      &= \sum^n_{j=0} \sum^n_{k=0} \sum^n_{l=0} \sum^n_{m=0} \delta_{j + k + l + m, n} \frac{n!}{j! k! l! m!} a^j b^k c^l d^m
    \end{aligned}
\end{equation}

\section{Hamiltonian Derivation}

Starting from \cref{equation:hill_solution}, the transverse coordinates \(x\) and \(y\) can be expressed from action and angle variables:

\begin{equation}
    \begin{aligned}
        x(s) &= \sqrt{2 J_x \beta_x} \cos \left( \phi_x + \phi_{x,0} \right) \text{ ,} \\
        y(s) &= \sqrt{2 J_y \beta_y} \cos \left( \phi_y + \phi_{y,0} \right) \text{ .}
    \end{aligned}
    \label{equation:hill_solution_xy_from_action_angle_variables}
\end{equation}

Here the dependence on the longitudinal coordinate \(s\) has been removed for clarity compared to \cref{equation:hill_solution}.
Using Euler's formula, these can be rewritten as:

\begin{equation}
    \begin{aligned}
        x &= \sqrt{2 J_x \beta_x} \frac{e^{i \left( \phi_x + \phi_{x,0} \right)} + e^{-i \left( \phi_x + \phi_{x,0} \right)}}{2} \\
        y &= \sqrt{2 J_y \beta_y} \frac{e^{i \left( \phi_y + \phi_{y,0} \right)} + e^{-i \left( \phi_y + \phi_{y,0} \right)}}{2} \\
    \end{aligned}
    \label{equation:hill_solution_xy_from_action_angle_variables_euler}
\end{equation}

As seen in \cref{section:non_linear_magnetic_multipoles}, the multipole expansion of the transverse planes Hamiltonian for a multipole of order \(n\) goes as:

\begin{equation}
    \mathcal{H} = \operatorname{Re} \left[ (K_n + i J_n) \frac{(x + i y)^n}{n!} \right]
\end{equation}

This is a slight change over \cref{equation:hamiltonian_multipole_order_n} as here the terms \(K_n\) and \(J_n\), representing the \emph{integrated magnet strengths}, are used over \(B_n\) and \(A_n\).
When plugging in the forms of \cref{equation:hill_solution_xy_from_action_angle_variables_euler}, one gets:

\begin{equation}
    \begin{aligned}
        \mathcal{H} = \operatorname{Re} \biggl[
            \frac{1}{2^n \cdot n!}  (K_n + iJ_n)
            \biggl(
                & \sqrt{2 J_x \beta_x} e^{i \left( \phi_x + \phi_{x,0} \right)}\\
                & + \sqrt{2 J_x \beta_x} e^{-i \left(\phi_x + \phi_{x,0} \right)}\\
                & + i \sqrt{2 J_y \beta_y} e^{i \left(\phi_y+ \phi_{y,0} \right)} \\
                & + i \sqrt{2 J_y \beta_y} e^{-i \left(\phi_y + \phi_{y,0} \right)} 
            \biggr)^n 
            \biggr]
    \end{aligned}
    \label{equation:hamiltonian_multipole_order_n_xy}
\end{equation}

Now, we can do the multinomial expansion of last term via \cref{equation:multinomial_expansion}.
Let us define \(a\), \(b\), \(c\), and \(d\) such that \(\left( a + b + c + d \right)^n\) equals this last term.
This yields:

\begin{equation}
    \begin{aligned}
        a^j &= \left( \sqrt{2 J_x \beta_x} e^{i \left( \phi_x + \phi_{x,0} \right)} \right)^j = 2^{\frac{j}{2}} J_x^{\frac{j}{2}} \beta_x^{\frac{j}{2}} e^{i j \left( \phi_x + \phi_{x,0} \right)} \text{ ,} \\
        b^k &= \left( \sqrt{2 J_x \beta_x} e^{-i \left( \phi_x + \phi_{x,0} \right)} \right)^k = (2 J_x)^{\frac{k}{2}} \beta_x^{\frac{k}{2}} e^{-i k \left( \phi_x + \phi_{x,0} \right)}           \text{ ,} \\
        c^l &= \left( i \sqrt{2 J_y \beta_y} e^{i \left( \phi_y + \phi_{y,0} \right)} \right)^l = i^l  (2 J_y)^{\frac{l}{2}} \beta_y^{\frac{l}{2}} e^{i l \left( \phi_y + \phi_{y,0} \right)}      \text{ ,} \\
        d^m &= \left( i \sqrt{2 J_y \beta_y} e^{i \left( \phi_y + \phi_{y,0} \right)} \right)^m = i^m (2 J_y)^{\frac{m}{2}} \beta_y^{\frac{m}{2}} e^{-i m \left( \phi_y + \phi_{y,0} \right)}      \text{ .} \\
    \end{aligned}
    \label{equation:variables_definition_for_multinomial_expansion}
\end{equation}

Combining these terms two by two gives:

\begin{equation}
    \begin{aligned}
        a^j b^k &= \phantom{i^{l+m}} (2J_x)^{\frac{j+k}{2}} \beta_x^{\frac{j+k}{2}} e^{i \left(j-k\right) \left(\phi_x + \phi_{x,0}\right)} \text{ ,} \\
        c^l d^m &= i^{l+m}           (2J_y)^{\frac{l+m}{2}} \beta_y^{\frac{l+m}{2}} e^{i \left(l-m\right) \left(\phi_y + \phi_{y,0}\right)}
    \end{aligned}
    \label{equation:ab_cd_combining}
\end{equation}

Combining these again, we get:

\begin{equation}
    \begin{aligned}
        a^j b^k c^l d^m &= i^{l+m} 
                   (2J_x)^{\frac{j+k}{2}} (2J_y)^{\frac{l+m}{2}} 
                   \beta_x^{\frac{j+k}{2}} \beta_y^{\frac{l+m}{2}} 
                   e^{i\left[ (j-k)(\phi_x + \phi_{x,0}) + (l-m)(\phi_y + \phi_{y,0}) \right]}
    \end{aligned}
\end{equation}

After isolating the terms independent of the position \(s\):

\begin{equation}
    \begin{aligned}
        a^j b^k c^l d^m = & i^{l+m} \beta_x^{\frac{j+k}{2}} \beta_y^{\frac{l+m}{2}} e^{i\left[ \left(j-k\right) \phi_x + \left(l-m\right)\phi_y \right]} \\
                          & \left(2 J_x\right)^{\frac{j+k}{2}} \left(2 J_y\right)^{\frac{l+m}{2}} e^{ i\left[ \left(j-k\right) \phi_{x,0} + \left(l-m\right) \phi_{y,0} \right]}
    \end{aligned}
\end{equation}

Going back to the \(\mathcal{H}\) form of \cref{equation:hamiltonian_multipole_order_n_xy}, using the \(a\), \(b\), \(c\), and \(d\) terms defined above and the multinomial expansion rules of \cref{equation:multinomial_expansion,equation:multinomial_expansion_kronecker}, we get for a given element:

\begin{equation}
    \begin{aligned}
        \mathcal{H} &=  \operatorname{Re} \left[ \frac{1}{2^n \cdot n!} (K_n + i J_n) (a + b + c + d)^n \right] \\
                    &=  \operatorname{Re} \left[ \frac{1}{2^n \cdot n!} (K_n + i J_n) \sum_{j + k + l + m = n} \frac{n!}{j! k! l! m!} a^j b^k c^l d^m \right] \\
                    &=  \operatorname{Re} \left[ (K_n + i J_n)  \sum_{j + k + l + m = n} \frac{1}{2^{j+k+l+m} \cdot j! k! l! m!} a^j b^k c^l d^m \right] \\
                    &=  \operatorname{Re} 
                    \biggl[ 
                        (K_n + i J_n) \sum_{j + k + l + m = n} \frac{1}{2^{j+k+l+m} \cdot j! k! l! m!} \\
                        & \begin{aligned} \phantom{ \operatorname{Re} \biggl[ (K_n + i J_n)  \sum_{j + k + l + m = n} \quad}
                        & i^{l+m} \beta_{x}^{\frac{j+k}{2}} \beta_{y}^{\frac{l+m}{2}} e^{i \left[ (j-k) \phi_x + (l-m) \phi_y \right]} \\
                        & (2 J_x)^{\frac{j+k}{2}} (2J_y)^{\frac{l+m}{2}} e^{i \left[ (j-k) \phi_{x,0} + (l-m) \phi_{y,0} \right]}
                    \biggr]
        \end{aligned}
    \end{aligned}
    \label{equation:hamiltonian_fully_multinomial_expanded}
\end{equation}

Noting that \(i\) is the imaginary unit, the real part of the sum is then directly influenced by the parity of (\(l + m\)).
The equation can be rewritten taking this into account, effectively selecting between the two terms:

\begin{equation}
    \Omega(i) =
        \begin{cases} 
        1 & \mbox{if } i \mbox{ is even}, \\
        0 & \mbox{if } i \mbox{ is odd}.
        \end{cases}
\end{equation}

And \cref{equation:hamiltonian_fully_multinomial_expanded} becomes:

\begin{equation}
    \begin{aligned}
        \mathcal{H} &= \sum_{j + k + l + m= n} \biggl[ \frac{K_n \Omega(l+m) + i J_n \Omega(l+m+1)}{2^{j+k+l+m} \cdot j! k! l! m!} i^{l+m} \beta_x^{\frac{j+k}{2}} \beta_y^{\frac{l+m}{2}} \\
                    & \begin{aligned}
                        \phantom{\biggl[ \sum_{j + k + l + m = n} \quad}
                    & (2 J_x)^{\frac{j+k}{2}} (2 J_y)^{\frac{l+m}{2}} e^{i \left[ (j-k) (\phi_x + \phi_{x,0}) + (l-m) (\phi_y + \phi_{y,0}) \right]}
                                \biggr]
                        \end{aligned}
    \end{aligned}
    \label{equation:hamiltonian_multinomial_expanded}
\end{equation}

One can then define \(h_{jklm}\) as:

\begin{equation}
    h_{jklm} = \frac{K_n \Omega(l+m) + i J_n \Omega(l+m+1)} {2^{j+k+l+m} \cdot j! k! l! m!} i^{l+m} \beta_x^{\frac{j+k}{2}} \beta_y^{\frac{l+m}{2}}
\end{equation}

Which reduces the above to \cref{equation:final_hamiltonian_derivation}, the form given in \cref{subsection:non_linear_transfer_maps}:

\begin{equation}
    \mathcal{H} = \sum_{jklm} h_{jklm} (2 J_x)^{\frac{j+k}{2}} (2J_y)^{\frac{l+m}{2}} e^{i\left[ (j-k)(\phi_x + \phi_{x,0}) + (l-m)(\phi_y + \phi_{y,0}) \right]}
    \label{equation:final_hamiltonian_derivation}
\end{equation}
where \(j+k+l+m=n\).

\todo{I THINK I HAVE A SIGN ERROR IN THE PHASES HELP}