\chapter{Appendix Title Here}
\label{appendix:hamiltonian_derivation}

This derivation is based on
\href{https://journals.aps.org/prab/pdf/10.1103/PhysRevSTAB.17.074001}{Andrea
Franchi's Resonance Driving Terms} paper. The derivation has been
completely redone here, only the notation is followed to stay
consistent.

As a reminder, the multipole expansion is given below, where we'll only
consider one type of multipole, so a fixed \(n\):
\begin{equation}H = \Re \left[(K_n + iJ_n)\frac{(x+iy)^n}{n!} \right]\end{equation}

The variables \(x\) and \(y\) can be expressed as phase action
variables:

\hl{this is $x_{\pm} = \hat{z} \pm i \hat{p}_z$ right?}

\begin{equation}
\begin{aligned}
  x(s) &= \sqrt{2 J_x \beta_x} \cos(\phi_x + \phi_{x0})\\
  y(s) &= \sqrt{2 J_y \beta_y} \cos(\phi_y + \phi_{y0})
\end{aligned}
\end{equation}

Where \(s\) indicates the observed point and \(x_0\) the location of the
magnet.

Using Euler's formula we can rewrite those variables: \begin{equation}
\begin{aligned}
  x &= \sqrt{2 J_x \beta_x} \frac{e^{i(\phi_x + \phi_{x0})} + e^{-i(\phi_x + \phi_{x0})}}{2}\\
  y &= \sqrt{2 J_y \beta_y} \frac{e^{i(\phi_y + \phi_{y0})} + e^{-i(\phi_y + \phi_{y0})}}{2}\\
\end{aligned}
\end{equation}

Which then gives us:

\begin{equation}
\begin{aligned}
H = \Re \biggl[\frac{1}{2^n \cdot n!}  (K_n + iJ_n)\biggl(&\sqrt{2 J_x \beta_x} e^{i(\phi_x + \phi_{x0})}\\
                                                      &+ \sqrt{2 J_x \beta_x} e^{-i(\phi_x + \phi_{x0})}\\
                                                      &+ i\sqrt{2 J_y \beta_y} e^{i(\phi_y+ \phi_{y0})} \\
                                                      &+ i\sqrt{2 J_y \beta_y} e^{-i(\phi_y + \phi_{y0})} 
                                                \biggr)^n\biggr]
\end{aligned}
\end{equation}

Now, we can do the multinomial expansion of last term via
\cref{eq:multinomial_expansion}:

\begin{equation}(a + b + c + d)^n = \sum_{j + k + l + m = n} \frac{n!}{j!k!l!m!} a^j b^k c^l d^m\end{equation}

\begin{equation}\begin{aligned}
a^j &= \left(\sqrt{2 J_x \beta_x} e^{i(\phi_x + \phi_{x0})} \right)^j\\
    &= 2^{\frac{j}{2}} J_x^{\frac{j}{2}} \beta_x^{\frac{j}{2}} e^{ij(\phi_x + \phi_{x0})},\\
b^k &= \left(\sqrt{2 J_x \beta_x} e^{-i(\phi_x + \phi_{x0})}\right)^k \\
    &= (2J_x)^{\frac{k}{2}} \beta_x^{\frac{k}{2}} e^{-ik(\phi_x + \phi_{x0})}, \\
c^l &= \left(i \sqrt{2 J_y \beta_y} e^{i(\phi_y + \phi_{y0}}\right)^l \\
    &= i^l  (2J_y)^{\frac{l}{2}} \beta_y^{\frac{l}{2}} e^{il(\phi_y + \phi_{y0})}, \\
d^m &= \left( \sqrt{2 J_y \beta_y} e^{i(\phi_y + \phi_{y0})}\right)^m \\
    &= i^m (2J_y)^{\frac{m}{2}} \beta_y^{\frac{m}{2}} e^{-im(\phi_y + \phi_{y0})} \\
\end{aligned}\end{equation}

\begin{equation}\begin{aligned}
a^j b^k &= \phantom{i^{l+m}} (2J_x)^{\frac{j+k}{2}} \beta_x^{\frac{j+k}{2}} e^{i(j-k)(\phi_x + \phi_{x0})}, \\
c^l d^m &= i^{l+m} (2J_y)^{\frac{l+m}{2}} \beta_y^{\frac{l+m}{2}} e^{i(l-m)(\phi_y + \phi_{y0})}
\end{aligned}\end{equation}

\begin{equation}\begin{aligned}
a^j b^k c^l d^m &= i^{l+m} 
                   (2J_x)^{\frac{j+k}{2}} (2J_y)^{\frac{l+m}{2}} 
                   \beta_x^{\frac{j+k}{2}} \beta_y^{\frac{l+m}{2}} 
                   e^{i\left[ (j-k)(\phi_x + \phi_{x0}) + (l-m)(\phi_y + \phi_{y0} \right]}
\end{aligned}\end{equation}

We can now isolate the terms that are independant of the position \(s\):
\begin{equation}\begin{aligned}
a^j b^k c^l d^m &= 
                   i^{l+m} 
                   \beta_x^{\frac{j+k}{2}} \beta_y^{\frac{l+m}{2}} 
                   e^{i\left[ (j-k)\phi_x + (l-m)\phi_y \right]}
                   (2J_x)^{\frac{j+k}{2}} (2J_y)^{\frac{l+m}{2}} 
                   e^{i\left[ (j-k)\phi_{x0} + (l-m)\phi_{y0} \right]}
\end{aligned}\end{equation}

Which gets us to \(H_w\), the Hamiltonian for an element \(w\), noting
that \(i\) is the imaginary unit (\(i^2 = -1\)):

\begin{equation}\begin{aligned}
  H_w &=  \Re\left[\frac{1}{2^n \cdot n!} (K_{nw} + iJ_{nw}) (a_w + b_w + c_w + d_w)^n \right] \\
    &=  \Re\left[ 
         \frac{1}{2^n \cdot n!} 
         (K_{nw} + iJ_{nw}) 
         \sum_{j + k + l + m = n} 
         \frac{n!}{j!k!l!m!} a^j_w b^k_w c^l_w d^m_w \right] \\
    &=  \Re\left[ 
         (K_{nw} + iJ_{nw}) 
         \sum_{j + k + l + m = n} 
         \frac{1}{2^{j+k+l+m} \cdot j!k!l!m!} a^j_w b^k_w c^l_w d^m_w \right] \\
    &=  \Re\biggl[ 
         (K_{nw} + iJ_{nw})
         \sum_{j + k + l + m = n}
         \frac{1}{2^{j+k+l+m} \cdot j!k!l!m!} \\
    &\begin{aligned}\phantom{ \Re\biggl[ (K_{nw} + iJ_{nw})  \sum_{j + k + l + m = n} \quad}
           &i^{l+m} 
           \beta_{xw}^{\frac{j+k}{2}} \beta_{yw}^{\frac{l+m}{2}} 
           e^{i\left[ (j-k)\phi_{x} + (l-m)\phi_{y} \right]}\\
           &(2J_x)^{\frac{j+k}{2}} (2J_y)^{\frac{l+m}{2}} 
           e^{i\left[ (j-k)\phi_{x0} + (l-m)\phi_{y0} \right]}
           \biggr]
    \end{aligned}
\end{aligned}\label{eq:hamiltonian_multinomial_expanded}\end{equation}

We can see that the term \(i^{l+m}\) in the sum will have a decisive
impact on the result, being multiplied either by \(K_{nw}\) or
\(iJ_{nw}\). The real part of the sum is then directly influenced by the
parity of (\(l+m\)).\\
The equation can be rewritten taking this into account, effectively
selecting between the two terms:

\begin{equation}
\Omega(i) =
\begin{cases} 
  1, & \mbox{if } i \mbox{ is even}, \\
  0, & \mbox{if } i \mbox{ is odd}.
\end{cases}
\end{equation}

\begin{equation}\begin{aligned}
 H_w  &=  
         \sum_{j + k + l + m = n}
        \biggl[ 
         \frac{ K_{nw}\Omega(l+m) + iJ_{nw}\Omega(l+m+1) }{2^{j+k+l+m} \cdot j!k!l!m!}
         i^{l+m} 
         \beta_{xw}^{\frac{j+k}{2}} \beta_{yw}^{\frac{l+m}{2}} \\
   &\begin{aligned}\phantom{\biggl[\sum_{j + k + l + m = n} \quad}
           &(2J_x)^{\frac{j+k}{2}} (2J_y)^{\frac{l+m}{2}} 
           e^{i\left[ (j-k)(\phi_{x} + \phi_{x0}) + (l-m)(\phi_{y} + \phi_{y0}) \right]}
           \biggr]
    \end{aligned}
\end{aligned}\label{eq:hamiltonian_multinomial_expanded}\end{equation}

We can then define \(h_{w,jklm}\) as:

\begin{equation}
h_{w,jklm} = \frac{K_{w,n} \Omega(l+m) + iJ_{w,n} \Omega(l+m+1)}{2^{j+k+l+m} \cdot j!k!l!m!} i^{l+m} \beta_{w,x}^{\frac{j+k}{2}}\beta_{w,y}^{\frac{l+m}{2}}
\end{equation}

Leading to:

\begin{equation}
H_w = \sum_{j+k+l+m=n} h_{w,jklm}
           (2J_x)^{\frac{j+k}{2}} (2J_y)^{\frac{l+m}{2}} 
           e^{i\left[ (j-k)(\phi_{x} + \phi_{x0}) + (l-m)(\phi_{y} + \phi_{y0}) \right]}
\end{equation}

The beam at the observed position \(b\) can be described by changing the
position variable.

\begin{equation}\begin{aligned}
  x_b &= x \cdot e^{i\Delta \phi_{x}^b} \\
  y_b &= y \cdot e^{i\Delta \phi_{y}^b}
\end{aligned}\end{equation}

Where \(\Delta \phi_{x,y}^b\) is the phase advance between the locations
b and w. This simply adds one term to our previous hamiltonian:

\begin{equation}\begin{aligned}
H_{bw} = \sum_{j+k+l+m=n}& h_{w,jklm}
           e^{i\left[ (j-k)(\Delta \phi_{w,x}^b) + (l-m)(\Delta \phi_{w,y}^b) \right]}\\
           &(2J_x)^{\frac{j+k}{2}} (2J_y)^{\frac{l+m}{2}} 
           e^{i\left[ (j-k)(\phi_{x} + \phi_{x0}) + (l-m)(\phi_{y} + \phi_{y0}) \right]}
\end{aligned}\end{equation}