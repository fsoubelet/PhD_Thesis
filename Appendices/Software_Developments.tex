\chapter{Software Developments}
\label{appendix:code_developments}

In parallel to other works presented in this document, a substantial effort was put into software developments to improve both analysis and simulation tools.
Below is shown the main software development work done over the course of this PhD.
For each software the main contributions are listed and references are provided.

\ilparagraph{JobSubmitter}{\cite{CODE:OMC:pylhc_submitter}}
Developed with J.~Dilly\orcidlink{0000-0001-7864-5448} and M.~Hofer\orcidlink{0000-0001-6173-0232}, JobSubmitter is a Python package for easily submitting parametrized studies to the HPC queuing system HTCondor~\cite{CODE:Douglas:condor-practice}.
It has quickly been adopted by colleagues due to its simplicity and efficiency.

\ilparagraph{Omc3}{\cite{CODE:OMC:omc3}}
The omc3 Python package is our main optics analysis and correction software, developed by most members of the \acrshort{OMC} team.
Many contributions were made in the form of bug fixes, maintenance, translation of old codes to the new version, documentation, testing and implementation of CI/CD pipelines.

\ilparagraph{OMC Documentation}{\cite{Website:OMC_Documentation}}
Together with J.~Dilly\orcidlink{0000-0001-7864-5448}, we created a website for the \acrshort{OMC} team to serve as a wiki, resource collection and entrypoint for team members.
It compiles information about the physics behind the OMC activities, experimental procedures, documentation and guides on the team's software, and resources for newcomers.

\ilparagraph{PyhDToolkit}{\cite{CODE:Soubelet:pyhdtoolkit}}
Initially designed for personal use, PyhDToolkit is a Python package for efficient simulations and visualizations with the cpymad~\cite{CODE:HIBTC:cpymad} and \gls{MADX}~\cite{CODE:MADX_guide} codes.
It is the software used for all results in this thesis.
Its core functionality was extracted into the lightweight cpymadtools~\cite{CODE:Soubelet:cpymadtools} package, which has been adopted by colleagues for its clean and efficient API.

\ilparagraph{PyLHC}{\cite{CODE:OMC:pylhc}}
Developed with J.~Dilly\orcidlink{0000-0001-7864-5448} and M.~Hofer\orcidlink{0000-0001-6173-0232}, PyLHC is a Python package holding various complementary scripts and modules to our other softwares.
Contributions include knob extraction scripts, data conversion functionality between database and simulation codes formats, and quick analysis of specific measurements.

\ilparagraph{Pyrws}{\cite{CODE:Soubelet:pyrws}}
Created for the \gls{LHC} \Gls{run}~\num{3} commissioning of \num{2022}, pyrws is a Python package to design \acrlong{RWS} configurations from LHC \gls{optics}.
It was used to prepare the experimental setup leading to the main work in this thesis, exposed in \cref{chapter:ir_local_coupling} and reported in \cref{appendix:experimental_knobs}.

\ilparagraph{Tfs-Pandas}{\cite{CODE:OMC:tfs_pandas}}
The tfs-pandas Python package is a workhorse of our codes used to handle the Table Format System (TFS) files output by both our simulations codes and analysis software.
Significant effort was put into a complete rewrite of the package, speeding up the operations by up to \num{100} times as well as the addition of quality-of-life features and tests for robustness.
This package is very downloaded as its use is widespread.

\ilparagraph{Turn-by-Turn}{\cite{CODE:OMC:turn_by_turn}}
The turn-by-turn Python package handles measurement data from different format corresponding to various machines.
Contribution was made in the form of extraction of this functionality from old codes and its rewrite into a clean API.

\todo{If this is included, adapt the chapter thumbs.}

\glsresetall                                     % reset glossary entries counts for the next chapter
