\chapter{Inconclusive Rigid Waist Shift Measurements}
\label{appendix:inconclusive_measurements}

\todo{Change the location of figures here.}

Results shown in \cref{section:rws_experimental_results} are from measurements in which the RWS knob was trimmed in as defined in \cref{table:rigid_waist_shift_knob}.
As mentioned in \cref{appendix:experimental_knobs} and as can be seen in \cref{figure:rigid_waist_shift_knob_effect_on_waist} the knob can be trimmed with an arbitrary factor, shifting the waist of the beams left or right of the IP. 
Measurements have also been conducted in both IRs with the RWS knobs trimmed in with a factor of \num{-1}, which we will call a \textit{negative RWS}.
For these, done under time constraints at the end of experimental shifts, some mistakes were made in the experimental setups and the data was deemed inconclusive after analysis.
Nonetheless, these results are included below.

\begin{figure}[!htb]
    \centering
    \includegraphics*[width=\textwidth]{Figures/Appendices/rws_measurement_ir1_b1_neg.pdf}
    \caption{Measurement scan with a negative RWS done at IR\num{1} for beam \num{1} (\textcolor{mplblue}{blue}) and simulations for the same setup (\textcolor{mplr}{red}). \todo{figure out what went wrong in the simulation here}}
    \label{figure:ir1_b1_neg_measurement}
\end{figure}

\Cref{figure:ir1_b1_neg_measurement} shows the results of a measurement scan done at IR\num{1} for beam \num{1} with a negative RWS.
When preparing for this scan an issue was encountered when trimming in the optics rematching knob.
As the knob was wrongly designed beforehand, it bypassed the powering limit of the Q\num{6} magnets and thus could only be trimmed in the machine with a factor \num{0.7} as the LHC protection system would not allow lower currents.
As a consequence a strong \(\beta\)-beating remained in the machine and cast doubts on the measurement data. 
Additionally, when reproducing this setup in simulations the obtained \(\abs{C^{-}}\) curve exhibits a suspicious behavior far from what would be expected (see \cref{figure:knob_to_cminus_with_waist}) and comparison with measurement data suggests a \textit{very} large error to be corrected: \num{15} units of the colinearity knob.
This data was discarded when determining corrections.

\Cref{figure:ir1_b2_neg_measurement} shows the results of a measurement scan done at IR\num{1} for beam \num{2} with a negative RWS.
Similarly to the above the optics rematching knobs was wrongly designed and exceeded the powering limit of the Q\num{6} magnets.
It could only be trimmed in the machine with a factor \num{0.5}, resulting once again in a strong \(\beta\)-beating in the machine during measurements.
The comparison with simulations also suggested a \textit{very} large error to be corrected: \num{16} units of the colinearity knob.
This data was discarded when determining corrections.

\begin{figure}[!htb]
    \centering
    \includegraphics*[width=\textwidth]{Figures/Appendices/rws_measurement_ir1_b2_neg.pdf}
    \caption{Measurement scan with a negative RWS done at IR\num{1} for beam \num{2} (\textcolor{mplblue}{blue}) and simulations for the same setup (\textcolor{mplr}{red}).}
    \label{figure:ir1_b2_neg_measurement}
\end{figure}

\Cref{figure:ir5_b1_neg_measurement} shows the results of a measurement scan done at IR\num{5} for beam \num{1} with a negative RWS.
During this scan, an issue was made when trimming the optics rematching knob: the rematching knob from the previous scan was not trimmed out of the machine, and as a result a mix of the both was present during the scan of the colinearity knob.
As a consequence, again, strong errors were left in the machine and while the measurement data does not suggest an unbelievable correction as in \cref{figure:ir1_b2_neg_measurement}, the failed experimental setup and suspicious shape of the measurements' \(\abs{C^{-}}\) curve led to this data being discarded when determining corrections.

\begin{figure}[!htb]
    \centering
    \includegraphics*[width=\textwidth]{Figures/Appendices/rws_measurement_ir5_b1_neg.pdf}
    \caption{Measurement scan with a negative RWS done at IR\num{5} for beam \num{1} (\textcolor{mplblue}{blue}) and simulations for the same setup (\textcolor{mplr}{red}).}
    \label{figure:ir5_b1_neg_measurement}
\end{figure}

\Cref{figure:ir5_b2_neg_measurement} shows the results of a measurement scan done at IR\num{5} for beam \num{1} with a negative RWS.
For this measurement scan, no mistake was made in the experimental setup and the data suggests a correction similar to what was found with the scans presented in \cref{subsection:rws_pos_measurements_corrections} (see \cref{table:rws_corrections_summary}).

\begin{figure}[!htb]
    \centering
    \includegraphics*[width=\textwidth]{Figures/Appendices/rws_measurement_ir5_b2_neg.pdf}
    \caption{Measurement scan with a negative RWS done at IR\num{5} for beam \num{1} (\textcolor{mplblue}{blue}) and simulations for the same setup (\textcolor{mplr}{red}).}
    \label{figure:ir5_b2_neg_measurement}
\end{figure}

Overall, a combination of human mistakes both during the preparation of the experimental setup and during the measurement scans themselves led to a large amount of data being discarded.
