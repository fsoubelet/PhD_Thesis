\chapter{Experimental Knobs Designed for the LHC}
\label{appendix:experimental_knobs}

One can find in this Appendix the full information of the different experimental knobs that were designed for the LHC.
The knobs are reported below as they have been implemented in the \emph{LHC Software Architecture} (LSA) framework.
In each case, the values correspond to a trim factor of the knob of \num{1}, and scale linearly with the trim factor.

These knobs have been designed for all beam processes of the LHC Run~3 optics as they were at the time of the 2022 commissioning, for a beam energy of \qty{6800}{\giga\electronvolt}.

%----------------------------------------------------------------------------------------

\section{Definitions of the Colinearity Knobs}

\Cref{table:lsa_ip1_colinearity_knob,table:lsa_ip5_colinearity_knob} show the settings used in LSA to define the colinearity knobs at \(\mathrm{IR1}\) and \(\mathrm{IR5}\).
These knobs control the skew quadrupole correctors left and right of the \(\mathrm{IP}\), the \(\mathrm{MQSX}\) magnets.

\begin{table}[!hbt]
    \centering
    \begin{tblr}{colspec={cc}}
        \hline
        \textbf{Component} & \textbf{Value} \\
        \hline
        \(\mathrm{RQSX3.L1/K1S}\)  &  \num{1E-4}  \\
        \(\mathrm{RQSX3.R1/K1S}\)  &  \num{-1E-4}  \\
        \hline
    \end{tblr}
    \caption{Definition of the colinearity knob for IR1 as implemented in LSA.}
    \label{table:lsa_ip1_colinearity_knob}
\end{table}

\begin{table}[!hbt]
    \centering
    \begin{tblr}{colspec={cc}}
        \hline
        \textbf{Component} & \textbf{Value} \\
        \hline
        \(\mathrm{RQSX3.L5/K1S}\)  &  \num{1E-4}  \\
        \(\mathrm{RQSX3.R5/K1S}\)  &  \num{-1E-4}  \\
        \hline
    \end{tblr}
    \caption{Definition of the colinearity knob for IR5 as implemented in LSA.}
    \label{table:lsa_ip5_colinearity_knob}
\end{table}

%----------------------------------------------------------------------------------------

\section{Definitions of the Rigid Waist Shift Knobs}

\Cref{table:lsa_ip1_rigid_waist_shift,table:lsa_ip5_rigid_waist_shift} show the settings used in LSA to define the rigid waist shift knobs at \(\mathrm{IR1}\) and \(\mathrm{IR5}\).
These knobs control the triplet magnets left and right of the \IP in order to move all four betatron waists simultaneously.

\begin{table}[!hbt]
    \centering
    \begin{tblr}{colspec={cc}}
        \hline
        \textbf{Component} & \textbf{Value} \\
        \hline
        \(\mathrm{MQXA1.L1/K1}\)  &  \num{-4.38891E-5}  \\
        \(\mathrm{MQXA1.R1/K1}\)  &  \num{-4.38891E-5}  \\
        \(\mathrm{MQXA3.L1/K1}\)  &  \num{-4.38891E-5}  \\
        \(\mathrm{MQXA3.R1/K1}\)  &  \num{-4.38891E-5}  \\
        \(\mathrm{MQXB2.L1/K1}\)  &  \num{4.38891E-5}  \\
        \(\mathrm{MQXB2.R1/K1}\)  &  \num{4.38891E-5}  \\
        \hline
    \end{tblr}
    \caption{Definition of the rigid waist shift knob for IR1 as implemented in LSA.}
    \label{table:lsa_ip1_rigid_waist_shift}
\end{table}

\begin{table}[!hbt]
    \centering
    \begin{tblr}{colspec={cc}}
        \hline
        \textbf{Component} & \textbf{Value} \\
        \hline
        \(\mathrm{MQXA1.L5/K1}\)  &  \num{-4.38891E-5}  \\
        \(\mathrm{MQXA1.R5/K1}\)  &  \num{-4.38891E-5}  \\
        \(\mathrm{MQXA3.L5/K1}\)  &  \num{-4.38891E-5}  \\
        \(\mathrm{MQXA3.R5/K1}\)  &  \num{-4.38891E-5}  \\
        \(\mathrm{MQXB2.L5/K1}\)  &  \num{4.38891E-5}  \\
        \(\mathrm{MQXB2.R5/K1}\)  &  \num{4.38891E-5}  \\
        \hline
    \end{tblr}
    \caption{Definition of the rigid waist shift knob for IR5 as implemented in LSA.}
    \label{table:lsa_ip5_rigid_waist_shift}
\end{table}

%----------------------------------------------------------------------------------------

\section{Definitions of the Optics Rematching Knobs}

\Cref{table:lsa_ip1_pos_rematching_knob,table:lsa_ip1_neg_rematching_knob} show the settings used in LSA to define the optics rematching knobs needed after applying the rigid waist shift knob, at \(\mathrm{IR1}\).
\Cref{table:lsa_ip1_pos_rematching_knob} gives the settings that rematch the optics when the \(\mathrm{IR1}\) rigid waist shift knob is applied with a factor of \num{1}, while \cref{table:lsa_ip1_neg_rematching_knob} gives the settings that rematch the optics when the \(\mathrm{IR1}\) rigid waist shift knob is applied with a factor of \num{-1}.
These knobs control the independent magnets \(\mathrm{Q4}\) to \(\mathrm{Q10}\) left and right of the \IP for both beams.

\begin{table}[!hbt]
    \centering
    \begin{tblr}{colspec={ccc}}
        \hline
        \textbf{Component} & \textbf{Beam 1 Value} & \textbf{Beam 2 Value} \\
        \hline
        \(\mathrm{RQ4.L1B[12]/K1}\)   &  \num{7.351348E-5}   &  \num{3.132269E-7}   \\
        \(\mathrm{RQ4.R1B[12]/K1}\)   &  \num{4.704082E-5}   &  \num{5.434962E-5}   \\
        \(\mathrm{RQ5.L1B[12]/K1}\)   &  \num{-2.142214E-4}  &  \num{1.283481E-4}   \\
        \(\mathrm{RQ5.R1B[12]/K1}\)   &  \num{-2.086652E-4}  &  \num{-1.782987E-5}  \\
        \(\mathrm{RQ6.L1B[12]/K1}\)   &  \num{1.252269E-4}   &  \num{-6.654750E-5}  \\
        \(\mathrm{RQ6.R1B[12]/K1}\)   &  \num{2.181278E-4}   &  \num{4.843161E-5}   \\
        \(\mathrm{RQ7.L1B[12]/K1}\)   &  \num{-1.654347E-5}  &  \num{-6.727209E-6}  \\
        \(\mathrm{RQ7.R1B[12]/K1}\)   &  \num{-2.380601E-5}  &  \num{7.671464E-5}   \\
        \(\mathrm{RQ8.L1B[12]/K1}\)   &  \num{4.701971E-5}   &  \num{-7.422834E-6}  \\
        \(\mathrm{RQ8.R1B[12]/K1}\)   &  \num{-5.166308E-5}  &  \num{2.816191E-5}   \\
        \(\mathrm{RQ9.L1B[12]/K1}\)   &  \num{1.183309E-4}   &  \num{-1.835947E-4}  \\
        \(\mathrm{RQ9.R1B[12]/K1}\)   &  \num{1.395191E-4}   &  \num{-6.844480E-5}  \\
        \(\mathrm{RQ10.L1B[12]/K1}\)  &  \num{-2.470503E-5}  &  \num{-1.415361E-4}  \\
        \(\mathrm{RQ10.R1B[12]/K1}\)  &  \num{-1.960854E-5}  &  \num{-1.257251E-4}  \\
        \hline
    \end{tblr}
    \caption{Definition of the optics rematching knob for \(\mathrm{IR1}\) as implemented in LSA. These settings rematch the optics for an applied rigid waist shift knob trimmed with a factor \num{1}.}
    \label{table:lsa_ip1_pos_rematching_knob}
\end{table}

\begin{table}[!hbt]
    \centering
    \begin{tblr}{colspec={ccc}}
        \hline
        \textbf{Component} & \textbf{Beam 1 Value} & \textbf{Beam 2 Value} \\
        \hline
        \(\mathrm{RQ4.L1B[12]/K1}\)   &  \num{2.890068E-5}   &  \num{5.819791E-5}   \\
        \(\mathrm{RQ4.R1B[12]/K1}\)   &  \num{-3.102580E-5}  &  \num{1.740345E-5}   \\
        \(\mathrm{RQ5.L1B[12]/K1}\)   &  \num{1.794669E-5}   &  \num{-2.591577E-4}  \\
        \(\mathrm{RQ5.R1B[12]/K1}\)   &  \num{2.076919E-4}   &  \num{-5.673007E-5}  \\
        \(\mathrm{RQ6.L1B[12]/K1}\)   &  \num{4.143969E-5}   &  \num{3.008508E-4}   \\
        \(\mathrm{RQ6.R1B[12]/K1}\)   &  \num{-1.848040E-4}  &  \num{-6.006251E-5}  \\
        \(\mathrm{RQ7.L1B[12]/K1}\)   &  \num{5.634562E-5}   &  \num{-3.647266E-5}  \\
        \(\mathrm{RQ7.R1B[12]/K1}\)   &  \num{-8.441509E-6}  &  \num{5.911061E-7}   \\
        \(\mathrm{RQ8.L1B[12]/K1}\)   &  \num{3.582289E-5}   &  \num{-1.783459E-4}  \\
        \(\mathrm{RQ8.R1B[12]/K1}\)   &  \num{1.446909E-4}   &  \num{1.123520E-4}   \\
        \(\mathrm{RQ9.L1B[12]/K1}\)   &  \num{-1.056186E-4}  &  \num{6.796452E-5}   \\
        \(\mathrm{RQ9.R1B[12]/K1}\)   &  \num{-1.686284E-4}  &  \num{1.143128E-4}   \\
        \(\mathrm{RQ10.L1B[12]/K1}\)  &  \num{-8.416329E-5}  &  \num{-1.871513E-5}  \\
        \(\mathrm{RQ10.R1B[12]/K1}\)  &  \num{-4.208946E-5}  &  \num{4.750863E-5}   \\
        \hline
    \end{tblr}
    \caption{Definition of the optics rematching knob for \(\mathrm{IR1}\) as implemented in LSA. These settings rematch the optics for an applied rigid waist shift knob trimmed with a factor \num{-1}.}
    \label{table:lsa_ip1_neg_rematching_knob}
\end{table}


\Cref{table:lsa_ip5_pos_rematching_knob,table:lsa_ip5_neg_rematching_knob} show the settings used in LSA to define the optics rematching knobs needed after applying the rigid waist shift knob, at \(\mathrm{IR5}\).
\Cref{table:lsa_ip5_pos_rematching_knob} gives the settings that rematch the optics when the \(\mathrm{IR5}\) rigid waist shift knob is applied with a factor of \num{1}, while \cref{table:lsa_ip1_neg_rematching_knob} gives the settings that rematch the optics when the \(\mathrm{IR5}\) rigid waist shift knob is applied with a factor of \num{-1}.
These knobs control the independent magnets \(\mathrm{Q4}\) to \(\mathrm{Q10}\) left and right of the \IP for both beams.

\begin{table}[!hbt]
    \centering
    \begin{tblr}{colspec={ccc}}
        \hline
        \textbf{Component} & \textbf{Beam 1 Value} & \textbf{Beam 2 Value} \\
        \hline
        \(\mathrm{RQ4.L5B[12]/K1}\)   &  \num{4.732644E-5}   &  \num{7.708167E-7}   \\
        \(\mathrm{RQ4.R5B[12]/K1}\)   &  \num{2.995622E-5}   &  \num{5.229277E-5}   \\
        \(\mathrm{RQ5.L5B[12]/K1}\)   &  \num{-1.540577E-4}  &  \num{1.320986E-4}   \\
        \(\mathrm{RQ5.R5B[12]/K1}\)   &  \num{-1.541586E-4}  &  \num{-2.507269E-5}  \\
        \(\mathrm{RQ6.L5B[12]/K1}\)   &  \num{8.088711E-5}   &  \num{-7.675093E-5}  \\
        \(\mathrm{RQ6.R5B[12]/K1}\)   &  \num{9.079173E-5}   &  \num{8.686051E-5}   \\
        \(\mathrm{RQ7.L5B[12]/K1}\)   &  \num{-5.050462E-5}  &  \num{6.267658E-6}   \\
        \(\mathrm{RQ7.R5B[12]/K1}\)   &  \num{-2.046442E-5}  &  \num{7.702426E-5}   \\
        \(\mathrm{RQ8.L5B[12]/K1}\)   &  \num{8.284445E-5}   &  \num{1.469226E-5}   \\
        \(\mathrm{RQ8.R5B[12]/K1}\)   &  \num{-1.498689E-5}  &  \num{6.506405E-5}   \\
        \(\mathrm{RQ9.L5B[12]/K1}\)   &  \num{1.330684E-4}   &  \num{-1.746977E-4}  \\
        \(\mathrm{RQ9.R5B[12]/K1}\)   &  \num{1.770079E-4}   &  \num{-7.499273E-5}  \\
        \(\mathrm{RQ10.L5B[12]/K1}\)  &  \num{3.745423E-6}   &  \num{-1.405645E-4}  \\
        \(\mathrm{RQ10.R5B[12]/K1}\)  &  \num{3.1220939E-5}  &  \num{-1.199213E-4}  \\
        \hline
    \end{tblr}
    \caption{Definition of the optics rematching knob for \(\mathrm{IR5}\) as implemented in LSA. These settings rematch the optics for an applied rigid waist shift knob trimmed with a factor \num{1}.}
    \label{table:lsa_ip5_pos_rematching_knob}
\end{table}

\begin{table}[!hbt]
    \centering
    \begin{tblr}{colspec={ccc}}
        \hline
        \textbf{Component} & \textbf{Beam 1 Value} & \textbf{Beam 2 Value} \\
        \hline
        \(\mathrm{RQ4.L5B[12]/K1}\)   &  \num{3.261927E-5}   &  \num{6.393201E-5}   \\
        \(\mathrm{RQ4.R5B[12]/K1}\)   &  \num{-3.059847E-5}  &  \num{2.057514E-5}   \\
        \(\mathrm{RQ5.L5B[12]/K1}\)   &  \num{1.578156E-5}   &  \num{-2.813483E-4}  \\
        \(\mathrm{RQ5.R5B[12]/K1}\)   &  \num{2.063856E-4}   &  \num{-5.432443E-5}  \\
        \(\mathrm{RQ6.L5B[12]/K1}\)   &  \num{2.618065E-5}   &  \num{3.291553E-4}   \\
        \(\mathrm{RQ6.R5B[12]/K1}\)   &  \num{-1.651558E-4}  &  \num{-7.698741E-5}  \\
        \(\mathrm{RQ7.L5B[12]/K1}\)   &  \num{6.293434E-5}   &  \num{-4.279221E-5}  \\
        \(\mathrm{RQ7.R5B[12]/K1}\)   &  \num{-1.308196E-5}  &  \num{1.068785E-5}   \\
        \(\mathrm{RQ8.L5B[12]/K1}\)   &  \num{1.700830E-5}   &  \num{-2.078329E-4}  \\
        \(\mathrm{RQ8.R5B[12]/K1}\)   &  \num{1.206262E-4}   &  \num{1.153268E-4}   \\
        \(\mathrm{RQ9.L5B[12]/K1}\)   &  \num{-9.633770E-5}  &  \num{4.971278E-5}   \\
        \(\mathrm{RQ9.R5B[12]/K1}\)   &  \num{-1.705114E-4}  &  \num{1.199293E-4}   \\
        \(\mathrm{RQ10.L5B[12]/K1}\)  &  \num{-8.920503E-5}  &  \num{-1.382229E-5}  \\
        \(\mathrm{RQ10.R5B[12]/K1}\)  &  \num{-6.293434E-5}  &  \num{4.006792E-5}   \\
        \hline
    \end{tblr}
    \caption{Definition of the optics rematching knob for \(\mathrm{IR5}\) as implemented in LSA. These settings rematch the optics for an applied rigid waist shift knob trimmed with a factor \num{-1}.}
    \label{table:lsa_ip5_neg_rematching_knob}
\end{table}





Spend mostly time in WHY we do things the way we do
talk we measure with AC dipole otherwise we would blow-up the beam
Good to know for me 