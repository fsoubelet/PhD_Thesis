\chapter{Element Naming Conventions in the LHC}
\label{appendix:naming_conventions}

%----------------------------------------------------------------------------------------

This appendix details the element naming convention in the LHC and HL-LHC.
\Cref{figure:lhc_segment_naming_scheme} below illustrates the established scheme using an example segment of the LHC.
A detailed listing of all LHC element and equipment names can be found at~\cite{CERN:Equipment_Codes}.

\begin{figure}[h]
    \centering
    \includegraphics*[width=0.9\linewidth]{Figures/Appendices/LHC_naming_scheme.pdf}
    \caption{Naming scheme in a segment of the LHC~\cite{CERN:Element_Naming}.}
    \label{figure:lhc_segment_naming_scheme}
\end{figure}

The general structure adheres to the following rules:
\begin{enumerate}
    \item Each octant is divided into two \intro{half-arcs} surrounding an \intro{insertion}.
    \item Each octant is also divided into a left side and a right side to said insertion.
    \item The center point of some octants is the \kl{Interaction Point} (IP), with their surrounding region sometimes also referred to as \kl{Interaction Region} (IR).
\end{enumerate}

From the standpoint of lattice definitions there are eight IPs, although this is merely for notational convenience. 
An \kl{Interaction Point} in the strict sense is a location where the two beams are made to collide, which only happens in octants \numlist{1;2;5;8} where experiments are run.
It is assumed through this document that when an IP or IR is referred, it refers to one of these octants.

Regardless of hosting an IP, all octants have in common that they host a long straight section in the middle as part of the insertion.
While the arcs can be considered generally uniform across LHC, the various long sections differ from octant to octant.

As the base pattern of the LHC arcs is a FODO lattice, the machine can be broken up into half-cells containing one quadrupole each.
As a result each half-cell is assigned a number such that the \(\mathrm{i^{th}}\) quadrupole away from the center of its octant is associated with the \(\mathrm{i^{th}}\) half-cell.
With this in mind, the general naming convention can be summarized as:

\begin{equation*}
        \langle \mathrm{TYPE} \rangle \langle \mathrm{SPECIAL} \rangle . \langle \mathrm{EXTRA} \rangle \langle \mathrm{HALF\_CELL} \rangle \langle \mathrm {LR} \rangle \langle \mathrm{OCTANT} \rangle . \mathrm{B} \langle \mathrm{12} \rangle
    \label{equation:lhc_naming_nomenclature}
\end{equation*}

The entries in the definition above are defined as follows:
\begin{itemize}
    \item \(\mathrm{TYPE}\): Entry specifying the type of element. Examples are given in \cref{table:element_prefix_examples}.
    \item \(\mathrm{SPECIAL}\): \textit{Optional} entry which can be used to subtype an element, e.g. \(\mathrm{H}\) or \(\mathrm{V}\) to signify that the element is acting on the horizontal or vertical plane.
    \item \(\mathrm{EXTRA}\): \textit{Optional} entry used to separate between otherwise identically named elements in regard to their type and number. E.g. \(\mathrm{A}\), \(\mathrm{B}\), \(\mathrm{C}\) to separate between three bending magnets in the same half-cell.
    \item \(\mathrm{LR}\): Entry specifying which side of the closest IP the element is on. The values for this entry are either \(\mathrm{L}\) (\textit{left}) or \(\mathrm{R}\) (\textit{right}).
    \item \(\mathrm{OCTANT}\): Entry specifying the octant the element is a part of. Valid entries are integers from \numrange{1}{8}.
    \item \(\mathrm{12}\): \textit{Optional} entry specifying which beam the element is part of. Either \num{1} or \num{2}, unless the element is shared between the two beams in which case the element name ends with the \(\mathrm{OCTANT}\) entry.
\end{itemize}

\begin{table}[!hbt]
    \centering
    \begin{tblr}{colspec={lr}}
        \hline
        \textbf{Element Type} & \textbf{Prefix}   \\
        \hline
        Bending Magnet    & \(\mathrm{MB}\)       \\
        Quadrupole        & \(\mathrm{MQ}\)       \\
        Orbit Corrector   & \(\mathrm{MCB}\)      \\
        BPM               & \(\mathrm{BPM}\)      \\
        Crab Cavity       & \(\mathrm{ACFCA}\)    \\
        Drift             & \(\mathrm{DRIFT}\)    \\
        \hline
    \end{tblr}
    \caption{Example prefixes for different LHC element types. An extensive list of all elements can be found at~\cite{CERN:Equipment_Codes}}.
    \label{table:element_prefix_examples}
 \end{table}

For instance, the element \(\mathrm{MQ.25L5.B1}\) is a quadrupole on the left side of \(\mathrm{IP5}\), in the \(25^{th}\) half-cell and for beam \(1\).
The special identifier can be used in multiple ways, for example \(\mathrm{MQML.10R1.B1}\) is a different type of quadrupole in half-cell \(10\), on the right side of \(\mathrm{IP1}\) for beam \(1\).
Here the special identifier describes the type of quadrupole.
For \(\mathrm{MCBH.21R5.B1}\), the special identifier \(\mathrm{H}\) signifies that it is a horizontal orbit corrector.
In the triplet quadrupoles one can notice for instance elements \(\mathrm{MQXB.A2L1}\) and \(\mathrm{MQXB.B2L1}\).
In this case the elements share type \(\mathrm{MQXB}\) (middle, single aperture inner triplet quadrupole), octant, side of \(\mathrm{IP}\) and half-cell, which is why they make use of the extra specifiers \(\mathrm{A}\) and \(\mathrm{B}\) to tell them apart.

Note that these elements skip the appendage of \(\mathrm{.B<12>}\).
These correspond to elements common to both beams, which can only happen in the \(\mathrm{IR}\).
This is due to the fact that when two beams are brought to collision they pass through the same equipment close to the point of collision.

% Full details can be found in: https://edms.cern.ch/ui/file/103369/3.2/LHC-PM-QA-204-32-00.pdf,  https://edms.cern.ch/ui/#!master/navigator/document?D:1929919921:1929919921:subDocs.