% Appendix B
\chapter{Resonance Driving Terms} % Main appendix title

\label{AppendixB} % For referencing this appendix elsewhere, use \Cref{AppendixB}

One can find in this Appendix the fully detailed list of LHC fills considered for the experimental campaign.

The derivation in this section follows the approach given in \cite{Tomas_thesis, Franchi_thesis}.
While the non-linear dynamics can not be described with matrices, it can be described by the transfer map formalism.
In the frame where the one turn map is represented by a pure rotation - in normal forms coordinates - it can be written as \cite{Tomas_thesis}:\\

\begin{equation}
    \mathcal{M} = 
    e^{:\eqnmarkbox[blue]{node1}{\tilde{h_{1}}}:}
    e^{:\eqnmarkbox[red]{node2}{\tilde{h_{2}}}:}
    \ldots
    e^{:\eqnmarkbox[green]{noden}{\tilde{h_{n}}}:}
    R
    \label{eq:norm_form_one_turn_map}
\end{equation}
\annotate[yshift=0.5em]{above, left}{node1}{First element}
\annotate[yshift=-0.75em]{below}{node2}{Second element}
\annotate[yshift=0.5em]{above, right}{noden}{Nth element}\\\\
% ------------------ %
where \(e^{:\tilde{h_{1}}:}\) is an exponential Lie operator describing a nonlinear element, and \(\mathbf{R}\) is the rotation matrix describing the linear motion.
This simplifies through the Campbell-Baker-Hausdorff theorem to:

\begin{equation}
    \mathcal{M} = e^{:h:} R
\end{equation}

In case the \(\tilde{h_{n}}\) are small, then \(h\) may be approximated by:

\begin{equation}
    h = \sum_{n=1}^{N} \tilde{h}_{n} + \sum_{n, m<n}^{N} \left[\tilde{h}_{m}, \tilde{h}_{n} \right] + \ldots
    \label{eq:h_expansion}
\end{equation}

Using only the first order in \(\tilde{h_{n}}\), \(h\) may be expanded according to \cref{eq:h_approximation_first_order} using the action-angle variables TODO.

\begin{equation}
    h = \sum_{j k l m} h_{j k l m} \left(2 J_{x}\right)^{\frac{j+k}{2}} \left(2 J_{y}\right)^{\frac{l+m}{2}} e^{i \left[(j-k)\left(\phi_{x}-\phi_{x_{0}}\right) + (l-m)\left(\phi_{y}-\phi_{y_{0}}\right) \right]}
    \label{eq:h_approximation_first_order}
\end{equation}

Here \(h_{j k l m}\) are Hamiltonian coefficients representing the contributions from multipoles of order \(n = j + k + l + m\).
A multipole of order \(n\) generates terms in the Hamiltonian \(\propto x^{j+k} y^{l+m}\), where again \(n = j + k + l + m\).

In the case of a skew quadrupole, such an element gives rise to terms in the Hamiltonian \(\propto xy\), meaning that it contributes to the Hamiltonian terms \(h_{1010}\), \(h_{1001}\), \(h_{0110}\) and \(h_{0101}\).
The idea behind normal form coordinates is to perform a transformation from a system with amplitude and phase dependence to a simpler form.
The simplest form is an amplitude dependent rotation, i.e. a rotation in phase space where the angle depends on the amplitude of the particle.
An excellent approach to normal forms formalism can be found in \cite{Carlier_thesis}.

The coordinate change is represented by a similarity transformation of the one turn map:

\begin{equation}
    e^{-: F:} e^{: h:} R e^{: F:}
\end{equation}
% ------------------ %
where \(F\) is the generating function for the transformation.
The formal solution to finding the generating function F is given in \cite{Forest_normal_forms} and the explicit expression is obtained in \cite{Tomas_thesis} as

\begin{equation}
    F = \sum_{j k l m} f_{j k l m} \left(2 I_{x}\right)^{\frac{j+k}{2}}\left(2 I_{y}\right)^{\frac{l+m}{2}} e^{i\left[(j-k)\left(\psi_{x}-\psi_{x_{0}}\right)+(l-m)\left(\psi_{y}-\psi_{y_{0}}\right)\right]}
    \label{eq:F_generating}
\end{equation}
% ------------------ %
where \(f_{jklm}\) are the resonance driving terms corresponding to the Hamiltonian terms \(h_{jklm}\) respectively.
They can be expressed according to \cref{eq:f_rdts} \cite{Tomas_thesis, Franchi_thesis}, where \(Q_x\) and \(Q_y\) are the unperturbed tunes.

\begin{equation}
    f_{jklm} = \frac{h_{jklm}}{1 - e^{i 2 \pi \left[(j-k) Q_{x} + (l-m) Q_{y} \right]}}
    \label{eq:f_rdts}
\end{equation}
% ------------------ %
\Cref{eq:f_rdts} diverges when \(j, k, l, m, Q_x\) and \(Q_y\) satisfy the condition:

\begin{equation}
    (j-k) Q_{x} + (l-m) Q_{y} = p \quad \text{ where } p \in \mathcal{Z}
    \label{eq:resonance_condition}
\end{equation}
% ------------------ %
Hence, the \(f_{jklm}\) terms are the driving terms of the resonances \([(j-k),(l-m)]\).
Every Hamiltonian term is associated with a resonance, which explains the term Resonance Driving Terms.
The normalized Courant-Snyder coordinates are related to the action-angle variable as

\begin{equation}
    \begin{aligned}
    z &= \sqrt{2 J_{z}} \cos (\phi_{z} - \phi_{z_{0}}) \\
    p_{z} &= -\sqrt{2 J_{z}} \sin (\phi_{z} - \phi_{z_{0}}) \quad \text { where } z=x, y
    \end{aligned}
    \label{eq:courant_snyder_to_action_angle}
\end{equation}
% ------------------ %
It is convenient to introduce the resonant basis h defined as

\begin{equation}
    \begin{aligned}
    h_{z}^{\pm} &= z \pm i p_{z} = \sqrt{2 J_{z}} e^{\mp i \left(\phi_{z}-\phi_{z_{0}}\right)} \quad \text { where } z=x, y \\
    \mathbf{h} &= \left( h_{x}^{+}, h_{x}^{-}, h_{y}^{+}, h_{y}^{-} \right)
    \end{aligned}
    \label{eq:resonant_basis_h}
\end{equation}

The transformation to a new set of Normal Form coordinates \(\left(\zeta_{x}^{+}, \zeta_{x}^{-}, \zeta_{y}^{+}, \zeta_{y}^{-}\right)\) is given by the operator \(e^{: -F :}\).
This is expressed as

\begin{equation}
    \zeta_{z}^{\pm} = \sqrt{2 I_{z}} e^{\pm i \left(\phi_{z}+\phi_{z 0} \right)} = e^{:-F:} h_{z}^{\pm}
    \label{eq:action_angle_to_normal_form}
\end{equation}
% ------------------ %
where \(I_{z}\) is the invariant of motion in the new frame.
The one-turn map in normal form coordinates is by construction an amplitude dependent rotation, and hence the motion in these coordinates as a function of the turn number N is given by

\begin{equation}
    \zeta_{z}^{-}(N) = \sqrt{2 I_{z}} e^{2 \pi \nu_{x} N + \phi z_{0}}
    \label{eq:normal_form_by_turn}
\end{equation}

The inverse transformation from the new action-angle variables to the linearly normalized variable is to first order written as

\begin{equation}
    h_{z}^{-} = e^{: F:} \zeta_{z}^{-} \simeq \zeta_{z}^{-} + \left[F, \zeta_{z}^{-}\right]
    \label{eq:inverse_normal_form_transform}
\end{equation}

and using \cref{eq:normal_form_by_turn} and \cref{eq:inverse_normal_form_transform} the normalized coordinates can be expressed in the form

\begin{equation}
    \begin{aligned}
    h_{x}^{-}(N) &= \sqrt{2 I_{x}} e^{i\left(2 \pi \nu_{x} N - \psi_{x_{0}}\right)} - \\
    & 2 i \sum_{jklm} j f_{jklm} \left(2 I_{x}\right)^{\frac{j+k-1}{2}} \left(2 I_{y}\right)^{\frac{l+m}{2}} e^{i \left[(1-j+k) \left(2 \pi \nu_{x} N-\psi_{x_{0}}\right) + (m-l) \left(2 \pi \nu_{y} N-\psi_{y_{0}}\right) \right]} \\
    h_{y}^{-}(N) &= \sqrt{2 I_{y}} e^{i\left(2 \pi \nu_{y} N - \psi_{y_{0}}\right)} - \\
    & 2 i \sum_{jklm} l f_{jklm} \left(2 I_{x}\right)^{\frac{j+k}{2}} \left(2 I_{y}\right)^{\frac{l+m-1}{2}} e^{i \left[(k-j) \left(2 \pi \nu_{x} N-\psi_{x_{0}}\right) + (1-l+m) \left(2 \pi \nu_{y} N-\psi_{y_{0}}\right) \right]}
    \end{aligned}
    \label{eq:normal_form_coordinates}
\end{equation}

%----------------------------------------------------------------------------------------